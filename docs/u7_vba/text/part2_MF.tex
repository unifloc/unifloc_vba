\section{Расчёт свойств потока}

В отличии от функций расчета PVT (физико-химических свойств флюидов) функции расчета свойства потока учитывают дополнительные параметры потока флюидов - $Q$ - дебит, объемный расход флюидов, $f_w$ - обводненность, $R_p$ - газовый фактор. В функциях свойств потока используется префикc \mintinline{vb.net}{MF_}.

Параметры потока, такие как расход ГЖС, доля газа в потоке, вязкость ГЖС важны для расчета и анализа работы скважин и скважинного оборудования.


\subsection{MF\_qmix\_m3day – расход газожидкостной смеси}

Функция позволяет рассчитать объемный расход газожидкостной смеси при заданных термобарических условиях. Объемный расход ГЖС важен например для подбора УЭЦН в скважине, так как именно определяет в какой точке характеристики УЭЦН будет работать. При наличии свободного газа в потоке расход ГЖС может быть значительно больше расхода жидкости на поверхности фиксируемого расходомером.
$$Q_{mix,rc} = Q_{w,sc} B_w(P,T) + Q_{o,sc} B_o(P,T)  + Q_{o,sc}  (R_p - R_s(P,T)) B_g(P,T) $$

Расход ГЖС определяется как сумма расходов отдельных фаз, приведенных к соответствующим термобарическим условиям, с учетом того, что часть газа будет растворена в нефти.

\putlisting{listings/MF_q_mix_rc_m3day.lst}

\subsection{MF\_rhomix\_kgm3 – плотность газожидкостной смеси}

Функция позволяет рассчитать плотность газожидкостной смеси при заданных термобарических условиях. 
$$\rho_{mix,rc} = \left( \frac{\rho_{w,sc}}{B_w} f_w + \frac{\rho_{o,sc} +r_s \rho_{g,sc} }{B_o}(1-f_w) \right) (1-f_g) + \frac{ \rho_{g,sc} }{B_g} f_g $$

\putlisting{listings/MF_Rhomix_kgm3.lst}

\subsection{MF\_gas\_fraction\_d – доля газа в потоке}

Функция расчёта доли свободного газа в потоке (без учёта проскальзывания) в зависимости от термобарических условий для заданного флюида. 
$$f_g = \frac{Q_{g,rc}}{Q_{mix,rc}} $$
Доля газа в потоке является одним из ключевых параметров ограничивающих производительность систем механизированной добычи - ЭЦН и других насосов.

\putlisting{listings/MF_gas_fraction_d.lst}

\subsection{MF\_p\_gas\_fraction\_atma – целевое давления для заданной доли газа в потоке}
Функция расчёта давления при котором достигается заданная доля свободного газа в потоке (без учёта проскальзывания). 
Значение давления при котором достигается определённая доля газа в потоке может быть найдено из решения уравнения, определяющего долю газа. 
$$f_g = \frac{Q_{g,rc}(P,T)}{Q_{mix,rc}(P,T)} $$
Решение в \unf реализовано итеративное, методом деления отрезка пополам (дихотомия). При вызове функции пересчитывается состояние смеси с различными термобарическими условиями. Поэтому расчёт проводится относительно медленно. 

\putlisting{listings/MF_p_gas_fraction_atma.lst}

\subsection{MF\_rp\_gas\_fraction\_m3m3 – целевой газовый фактор для заданной доли газа в потоке}
Функция расчёта газового фактора $R_p$ при котором достигается заданная доля свободного газа в потоке (без учёта проскальзывания) . 
Значение давления при котором достигается определённая доля газа в потоке может быть найдено из решения уравнения, определяющего долю газа. 
$$f_g = \frac{Q_{g,rc}(P,T,R_p)}{Q_{mix,rc}(P,T,R_p)} $$
Решение в \unf реализовано итеративное, методом деления отрезка пополам (дихотомия). При вызове функции пересчитывается состояние смеси с различными термобарическими условиями. Поэтому расчёт проводится относительно медленно. 

\putlisting{listings/MF_rp_gas_fraction_m3m3.lst}

\section{Сепарация газа в скважине}
В скважинах оборудованных системами механизированной добычи нефти важную роль играет процесс сепарации газа на приёме насоса. Под сепарацией газа понимается отделение части свободного газа из потока и перенаправление его по отдельному гидравлическому каналу на поверхность. В результате сепарации газа меняются свойства флюида, поступающего в насос и НКТ выше насоса. Оценка величины сепарации может быть проведена приведёнными ниже функциями.

\subsection{MF\_ksep\_natural\_d – естественная сепарация газа}
Функция рассчитывает естественную сепарацию газа на приёме насоса в скважине с использованием корреляции Маркеса \cite{Marquez_2003} . Результат - безразмерная величина в диапазоне от 0 до 1. 

\putlisting{listings/MF_ksep_natural_d.lst}

\subsection{MF\_ksep\_gasseparator\_d – сепарация газа роторным газосепаратором}
Функция рассчитывает сепарацию газа с использованием роторного газосепаратора, являющегося обычно частью компоновки УЭЦН. Данный расчет основан на результатах испытания характеристик роторных газосепараторов, выполненных в РГУ нефти и газа имени И.М.Губкина \cite{SPE_117415_2008}. 

Следует отметить, что несмотря на хорошее соответствие промысловых исследований и расчетов с использованием корреляции для естественной и искусственной сепарации \cite{SPE_117415_2008} к результатам стендовых исследований стоит относится с осторожностью. Основой осторожности могут быть следующие соображения: характеристики различных газосепараторов достаточно сильно отличаются друг от друга - есть удачные конструкции и не очень, при этом результаты стендовых испытаний доступны только для ограниченного набора конструкций, стендовые условия достаточно сильно отличаются от скважинных - ниже давление, другие модельные рабочие жидкости, точно оценить коэффициент сепарации газосепаратора в промысловых условиях затруднительно - набор таких данных для сравнения ограничен. 

Тем не менее изучение результатов стендовых испытаний полезно при проведении расчетов и развивает инженерную интуицию. 

\putlisting{listings/MF_ksep_gasseparator_d.lst}


\subsection{MF\_ksep\_total\_d – общая сепарация газа}

Функция рассчитывает полную сепарацию газа на приёме насосе в скважине по известным значениям естественной сепарации газа и коэффициента сепарации газосепаратора. Результат - безразмерная величина в диапазоне от 0 до 1. 

$$K_{sep\_total} = K_{sep\_nat} + (1-K_{sep\_nat}) K_{sep\_gassep}$$

\putlisting{listings/MF_ksep_total_d.lst}

\section{Расчёт многофазного потока в штуцере}


Штуцер или локальное гидравлическое сопротивление - элемент скважины или системы трубопроводов, применяемых для создания дополнительного перепада давления в системе и ограничения потока. 
Возможны различные варианты реализации штуцера - со штуцерной камерой, с угловым краном, позволяющим менять диаметр штуцера и другие.
Ключевым параметром штуцера является диаметр \(d_{choke} \) определяющий его способность к ограничению потока. 

\begin{figure}[h!]
	\begin{center}
	    		% https://www.mathcha.io/editor# использован для построения картинок



		
		\tikzset{every picture/.style={line width=0.75pt}} %set default line width to 0.75pt        
		
		\begin{tikzpicture}[x=0.75pt,y=0.75pt,yscale=-1,xscale=1]
		%uncomment if require: \path (0,300); %set diagram left start at 0, and has height of 300
		
		%Shape: Rectangle [id:dp8089540927658381] 
		\draw  [color={rgb, 255:red, 0; green, 0; blue, 0 }  ,draw opacity=1 ][fill={rgb, 255:red, 155; green, 155; blue, 155 }  ,fill opacity=1 ][line width=2.25]  (92,42) -- (570.83,42) -- (570.83,56.33) -- (92,56.33) -- cycle ;
		%Shape: Rectangle [id:dp7288541809010827] 
		\draw  [fill={rgb, 255:red, 155; green, 155; blue, 155 }  ,fill opacity=1 ][line width=2.25]  (92,227) -- (570.83,227) -- (570.83,241) -- (92,241) -- cycle ;
		%Shape: Rectangle [id:dp666453613189492] 
		\draw  [color={rgb, 255:red, 0; green, 0; blue, 0 }  ,draw opacity=1 ][fill={rgb, 255:red, 155; green, 155; blue, 155 }  ,fill opacity=1 ][line width=2.25]  (323.83,56.33) -- (341.17,56.33) -- (341.17,118.67) -- (323.83,118.67) -- cycle ;
		%Shape: Rectangle [id:dp015115451250117262] 
		\draw  [fill={rgb, 255:red, 155; green, 155; blue, 155 }  ,fill opacity=1 ][line width=2.25]  (323.83,165) -- (341.83,165) -- (341.83,226.83) -- (323.83,226.83) -- cycle ;
		%Right Arrow [id:dp058738740185342975] 
		\draw   (231,133.5) -- (274.56,133.5) -- (274.56,127) -- (289.83,140) -- (274.56,153) -- (274.56,146.5) -- (231,146.5) -- cycle ;
		%Straight Lines [id:da28021737295590965] 
		\draw    (341,119) -- (455,119) ;
		
		
		%Straight Lines [id:da8575303554097866] 
		\draw    (341,165) -- (455,165) ;
		
		
		%Straight Lines [id:da44299065539354565] 
		\draw    (440,120.89) -- (440,161.67) ;
		\draw [shift={(440,163.67)}, rotate = 270.28] [color={rgb, 255:red, 0; green, 0; blue, 0 }  ][line width=0.75]    (10.93,-3.29) .. controls (6.95,-1.4) and (3.31,-0.3) .. (0,0) .. controls (3.31,0.3) and (6.95,1.4) .. (10.93,3.29)   ;
		\draw [shift={(440.22,118.89)}, rotate = 90.28] [color={rgb, 255:red, 0; green, 0; blue, 0 }  ][line width=0.75]    (10.93,-3.29) .. controls (6.95,-1.4) and (3.31,-0.3) .. (0,0) .. controls (3.31,0.3) and (6.95,1.4) .. (10.93,3.29)   ;
		%Shape: Rectangle [id:dp8558237837917941] 
		\draw  [color={rgb, 255:red, 155; green, 155; blue, 155 }  ,draw opacity=1 ][fill={rgb, 255:red, 155; green, 155; blue, 155 }  ,fill opacity=1 ] (325.94,51) -- (339.28,51) -- (339.28,91) -- (325.94,91) -- cycle ;
		%Shape: Rectangle [id:dp8173981538013828] 
		\draw  [color={rgb, 255:red, 155; green, 155; blue, 155 }  ,draw opacity=1 ][fill={rgb, 255:red, 155; green, 155; blue, 155 }  ,fill opacity=1 ] (325.94,196) -- (339.94,196) -- (339.94,236) -- (325.94,236) -- cycle ;
		
		% Text Node
		\draw (207.67,141.04) node [scale=1.2,rotate=-0.61]  {$Q_{liq}$};
		% Text Node
		\draw (472,142.04) node [scale=1.2,rotate=-0.61]  {$d_{choke}$};
		% Text Node
		\draw (117.33,142) node [scale=1.44,rotate=-0.74]  {$P_{in}$};
		% Text Node
		\draw (540.67,140.37) node [scale=1.44,rotate=-0.74]  {$P_{out}$};
		
		
		\end{tikzpicture}
		\caption{Схема локального гидравлического сопротивления - штуцера}
		\label{ris:Pipe_choke}
	\end{center}
\end{figure}

Как и у любого элемента гидравлического потока есть три ключевых параметра - давление на входе \( P_{in} \), давление на выходе \(P_{out}\)  и расход газожидкостной смеси, обычно задаваемый в стандартных условиях \(Q_{liq} \). Задание любых двух элементов позволяет вычислить третий. При задании трех элементов модель штуцера может быть настроена на замеры за счёт подбора калибровочного параметра.

Следует обратить внимание, расчёт перепада давления в штуцере сильно зависит от направления расчета. При фиксированном давлении на выходе $P_{out}$, что для скважины и штуцера на устье соответствует заданному давлению в линии, для любого расхода ГЖС через штуцер можно найти соответствующее значение давления на входе \ref{ris:choke_out_curves}.
 
\begin{figure}[h!]
	
	\begin{center}
		
		\newcommand{\dPipeDataFile}{data/choke1.prn}
		\begin{tikzpicture}[scale=1]
		\begin{axis}[
		width=14cm,
		height=8cm,
		xlabel=$Q\; m^3 / day$,
		ylabel=$P_{in} \; atma$,
		legend pos=south east,
		title=Перепад давления в штуцере]
		\addplot table [y=Pout_1, x=Q]{\dPipeDataFile};
		\addlegendentry{$P_{out}=1$}
		\addplot table [y=Pout_5, x=Q]{\dPipeDataFile};
		\addlegendentry{$P_{out}=5$}
		\addplot table [y=Pout_10, x=Q]{\dPipeDataFile};
		\addlegendentry{$P_{out}=10$}
		\addplot table [y=Pout_15, x=Q]{\dPipeDataFile};
		\addlegendentry{$P_{out}=15$}
		\addplot table [y=Pout_20, x=Q]{\dPipeDataFile};
		\addlegendentry{$P_{out}=20$}
		\addplot table [y=Pout_30, x=Q]{\dPipeDataFile};
		\addlegendentry{$P_{out}=30$}
		\end{axis}
		\end{tikzpicture}
		
		
		\caption{Кривые зависимости давления на входе в штуцер от дебита при фиксированном давлении на выходе из штуцера $P_{out}$}
		\label{ris:choke_out_curves}
		
	\end{center}
\end{figure} 

А вот при фиксированном давлении на входе $P_{in}$ или фиксированном буферном давлении $P_{buf}$ не для всякого расхода ГЖС можно рассчитать давление на выходе \ref{ris:choke_in_curves}. При фиксированном давлении на входе $P_{in}$ существует максимальный расход ГЖС, который можно прокачать через штуцер с заданным диаметром проходного канала. Такой расход называется критическим. При критическом расходе в канале штуцера скорость потока достигает скорости звука и давление на входе перестает зависеть от давления за штуцером. Величина критического расхода через штуцер зависит от давления на входе, поскольку с повышением давления увеличивается скорость звука в среде.

Вертикальная линия на графике зависимости давления на выходе $P_{out}$ от дебита при критическом расходе показывает, что давление не определяется однозначно, а может принимать любое значение на вертикальной линии. Подобная неоднозначность расчетного давления на выходе штуцера может осложнять расчеты и должна учитываться инженером разрабатывающим расчетный модуль или проводящим расчёты.

\begin{figure}[h!]
	
	\begin{center}
		
		\newcommand{\dPipeDataFile}{data/choke2.prn}
		\begin{tikzpicture}[scale=1]
		\begin{axis}[
		width=14cm,
		height=8cm,
		xlabel=$Q\; m^3 / day$,
		ylabel=$P_{out} \; atma$,
		legend pos=south east,
		title=Перепад давления в штуцере]
		\addplot table [y=Pin_10, x=Q]{\dPipeDataFile};
		\addlegendentry{$P_{in}=10$}
		\addplot table [y=Pin_15, x=Q]{\dPipeDataFile};
		\addlegendentry{$P_{in}=15$}
		\addplot table [y=Pin_20, x=Q]{\dPipeDataFile};
		\addlegendentry{$P_{in}=20$}
		\addplot table [y=Pin_25, x=Q]{\dPipeDataFile};
		\addlegendentry{$P_{in}=25$}
		\addplot table [y=Pin_30, x=Q]{\dPipeDataFile};
		\addlegendentry{$P_{in}=30$}
		\addplot table [y=Pin_35, x=Q]{\dPipeDataFile};
		\addlegendentry{$P_{in}=35$}
		\end{axis}
		\end{tikzpicture}
		
		
		\caption{Кривые зависимости давления на выходе из штуцера от дебита при фиксированном давлении на входе в штуцер $P_{in}$}
		\label{ris:choke_in_curves}
		
	\end{center}
\end{figure} 

Функции расчета штуцера позволяют настроить модель штуцера на замерные данные. Настройка проводится за счет параметра калибровки $c_{calibr}$ \mintinline{vb.net}{c_calibr_fr}. 
Параметр калибровки $c_{calibr}$ применяется как множитель на дебит при расчете характеристики штуцера. 
$$Q_{real} = Q_{calc} * c_{calibr}$$
Таким образом $c_{calibr}=1$ отключает калибровку. А изменение $c_{calibr}$ позволит изменить характеристику штуцера для согласования с измерениями \ref{ris:choke_cal_curves}.

\begin{figure}[h!]
	
	\begin{center}
		
		\newcommand{\dPipeDataFile}{data/choke3.prn}
		\begin{tikzpicture}[scale=1]
		\begin{axis}[
		width=14cm,
		height=6cm,
		xlabel=$Q\; m^3 / day$,
		ylabel=$P_{out} \; atma$,
		legend pos=south west,
		title=Пример калибровки модели штуцера]
		\addplot table [y=cal_1, x=Q]{\dPipeDataFile};
		\addlegendentry{$c_{calibr}=1$}
		\addplot table [y=cal_1.2, x=Q]{\dPipeDataFile};
		\addlegendentry{$c_{calibr}=1.2$}
		\end{axis}
		\end{tikzpicture}
		
		
		\caption{Кривые зависимости давления на выходе из штуцера от дебита при фиксированном давлении на входе в штуцер $P_{in}$}
		\label{ris:choke_cal_curves}
		
	\end{center}
\end{figure}  

Все функции для расчета штуцера содержат в названии слово \mintinline{vb.net}{choke}.  

\subsection{MF\_p\_choke\_atma – Расчет давления на входе или на выходе штуцера}
Функция позволяет рассчитать давление на входе или выходе штуцера по известному давлению на противоположном конце при известных параметрах потока (дебите жидкости, обводнённости, газовому фактору). Расчёт проводится по корреляции Перкинса \cite{Perkins_1993} с учётом многофазного потока. 
 
\putlisting{listings/MF_p_choke_atma.lst}

%\subsection{MF\_dp\_choke\_atm – Расчёт перепада давления в штуцере}
%Функция позволяет рассчитать по известному линейному давлению и дебиту или по известному буферному давлению и дебиту перепад давления.  Расчет проводится по корреляции Перкинса \cite{Perkins_1993} с учетом многофазного потока.  
%Функция возвращает перепад давления и температуры в виде массива.
%\putlisting{listings/MF_dp_choke_atm.lst}


\subsection{MF\_qliq\_choke\_sm3day – функция расчёта дебита жидкости через штуцер}
Функция позволяет рассчитать по известному буферному давлению и линейному давлению дебит жидкости. Расчет проводится по корреляции Перкинса \cite{Perkins_1993} с учетом многофазного потока.  

\putlisting{listings/MF_qliq_choke_sm3day.lst}


\subsection{MF\_calibr\_choke\_fr – функция настройки модели штуцера}
Функция позволяет рассчитать корректирующий фактор для модели штуцера, позволяющий согласовать результаты замеров давления и дебита. Расчет проводится по корреляции Перкинса \cite{Perkins_1993} с учетом многофазного потока.  

\putlisting{listings/MF_calibr_choke_fr.lst}

\newpage
\section{Расчет многофазного потока в трубе}

Для расчета участка трубы с использованием пользовательских функций Унифлок применяется следующая схема - \ref{ris:Pipe_scheme_1}.

Участок трубы задается как прямой с постоянным наклоном $\theta$  длиной $L$, постоянного диаметра $d$. Поток движется под углом $\theta$ к горизонтальной плоскости. Угол  $\theta$ меняется от -90 до 90 градусов Цельсия. Отрицательная величина  $\theta < 0 $ означает, что поток движется вниз - например отрицательным будет угол наклона для нагнетательной скважины. Угол наклона $\theta = 0 $ соответствует потоку в горизонтальном участке трубопровода.

Труба имеет постоянную по всей длине шероховатость стенок. 

\begin{figure}[h!]
	\begin{center}
		% https://www.mathcha.io/editor# использован для построения картинок

\tikzset{every picture/.style={line width=0.75pt}} %set default line width to 0.75pt        

\begin{tikzpicture}[x=0.75pt,y=0.75pt,yscale=-1,xscale=1]
%uncomment if require: \path (0,395.3333282470703); %set diagram left start at 0, and has height of 395.3333282470703

%Shape: Can [id:dp2899696286091056] 
\draw  [fill={rgb, 255:red, 250; green, 245; blue, 184 }  ,fill opacity=1 ][line width=2.25]  (164.23,345.91) -- (471.15,94.25) .. controls (473.82,92.06) and (481.92,97.51) .. (489.23,106.43) .. controls (496.54,115.34) and (500.3,124.35) .. (497.63,126.55) -- (190.71,378.2)(164.23,345.91) .. controls (166.9,343.71) and (175,349.16) .. (182.31,358.08) .. controls (189.63,367) and (193.39,376.01) .. (190.71,378.2) .. controls (188.03,380.4) and (179.94,374.95) .. (172.62,366.03) .. controls (165.31,357.11) and (161.55,348.1) .. (164.23,345.91) -- cycle ;
%Shape: Arc [id:dp7673222576415257] 
\draw  [draw opacity=0] (213.54,358.73) .. controls (215.72,361.29) and (217.51,364.25) .. (218.76,367.57) .. controls (220.22,371.43) and (220.84,375.4) .. (220.7,379.27) -- (190.71,378.2) -- cycle ; \draw   (213.54,358.73) .. controls (215.72,361.29) and (217.51,364.25) .. (218.76,367.57) .. controls (220.22,371.43) and (220.84,375.4) .. (220.7,379.27) ;
%Straight Lines [id:da2539925089352497] 
\draw    (111.83,289.33) -- (164.23,345.91) ;


%Straight Lines [id:da27080386920459176] 
\draw    (426.06,41.14) -- (471.15,94.25) ;


%Straight Lines [id:da6784647335940455] 
\draw    (138.03,317.62) -- (448.6,67.7) ;


%Straight Lines [id:da42906958912244875] 
\draw    (343.67,226) -- (373,202.37) ;
\draw [shift={(374.56,201.11)}, rotate = 501.14] [color={rgb, 255:red, 0; green, 0; blue, 0 }  ][line width=0.75]    (10.93,-3.29) .. controls (6.95,-1.4) and (3.31,-0.3) .. (0,0) .. controls (3.31,0.3) and (6.95,1.4) .. (10.93,3.29)   ;

%Straight Lines [id:da31602361859897776] 
\draw    (89,379) -- (569.56,379) ;
\draw [shift={(571.56,379)}, rotate = 180] [color={rgb, 255:red, 0; green, 0; blue, 0 }  ][line width=0.75]    (10.93,-3.29) .. controls (6.95,-1.4) and (3.31,-0.3) .. (0,0) .. controls (3.31,0.3) and (6.95,1.4) .. (10.93,3.29)   ;


% Text Node
\draw (233.67,364.33) node   {$\theta $};
% Text Node
\draw (269.67,187.67) node [rotate=-2.44]  {$L$};
% Text Node
\draw (200.33,343.67) node [rotate=-0.74]  {$P_{in}$};
% Text Node
\draw (472.67,120) node [rotate=-0.74]  {$P_{out}$};
% Text Node
\draw (335.67,230.67) node [rotate=-0.61]  {$Q_{liq}$};


\end{tikzpicture}
		\caption{Схема трубы принятая для расчётов с использованием пользовательских функций}
		\label{ris:Pipe_scheme_1}
	\end{center}
\end{figure}

Для расчёта распределения давления в трубе необходимо задать граничное значение давления на одном из концов трубы. Граничное давление всегда задается параметром  \mintinline{vb.net}{Pcalc_atma}. Температура потока в точке, где задается давление, задается параметром  \mintinline{vb.net}{T_calc_C}.  Возможно два варианта задания условия - по потоку  \ref{ris:Pipe_scheme_2}  \mintinline{vb.net}{calc_along_flow=1}. и против потока  \ref{ris:Pipe_scheme_3} \mintinline{vb.net}{calc_along_flow=0}. 

\begin{figure}[h!]
	\begin{center}
				\tikzset{every picture/.style={line width=0.75pt}} %set default line width to 0.75pt        
		
		\begin{tikzpicture}[x=0.75pt,y=0.75pt,yscale=-1,xscale=1]
		%uncomment if require: \path (0,390); %set diagram left start at 0, and has height of 390
		
		%Shape: Can [id:dp7382807235009181] 
		\draw  [fill={rgb, 255:red, 250; green, 245; blue, 184 }  ,fill opacity=1 ][line width=2.25]  (176.23,350.28) -- (483.15,98.62) .. controls (485.82,96.43) and (493.92,101.88) .. (501.23,110.8) .. controls (508.54,119.72) and (512.3,128.72) .. (509.63,130.92) -- (202.71,382.57)(176.23,350.28) .. controls (178.9,348.08) and (187,353.54) .. (194.31,362.45) .. controls (201.63,371.37) and (205.39,380.38) .. (202.71,382.57) .. controls (200.03,384.77) and (191.94,379.32) .. (184.62,370.4) .. controls (177.31,361.48) and (173.55,352.47) .. (176.23,350.28) -- cycle ;
		%Shape: Arc [id:dp26711502457409386] 
		\draw  [draw opacity=0] (225.54,363.1) .. controls (227.72,365.66) and (229.51,368.62) .. (230.76,371.95) .. controls (232.22,375.8) and (232.84,379.77) .. (232.7,383.64) -- (202.71,382.57) -- cycle ; \draw   (225.54,363.1) .. controls (227.72,365.66) and (229.51,368.62) .. (230.76,371.95) .. controls (232.22,375.8) and (232.84,379.77) .. (232.7,383.64) ;
		%Straight Lines [id:da2708011557353165] 
		\draw    (123.83,293.7) -- (176.23,350.28) ;
		
		
		%Straight Lines [id:da09547020181005683] 
		\draw    (438.06,45.51) -- (483.15,98.62) ;
		
		
		%Straight Lines [id:da29893852101776] 
		\draw    (150.03,321.99) -- (460.6,72.07) ;
		
		
		%Straight Lines [id:da6573662742713218] 
		\draw    (355.67,230.37) -- (385,206.74) ;
		\draw [shift={(386.56,205.48)}, rotate = 501.14] [color={rgb, 255:red, 0; green, 0; blue, 0 }  ][line width=0.75]    (10.93,-3.29) .. controls (6.95,-1.4) and (3.31,-0.3) .. (0,0) .. controls (3.31,0.3) and (6.95,1.4) .. (10.93,3.29)   ;
		
		%Straight Lines [id:da5485107193914818] 
		\draw    (101,383.37) -- (581.56,383.37) ;
		\draw [shift={(583.56,383.37)}, rotate = 180] [color={rgb, 255:red, 0; green, 0; blue, 0 }  ][line width=0.75]    (10.93,-3.29) .. controls (6.95,-1.4) and (3.31,-0.3) .. (0,0) .. controls (3.31,0.3) and (6.95,1.4) .. (10.93,3.29)   ;
		
		%Right Arrow [id:dp6841435853741948] 
		\draw  [fill={rgb, 255:red, 245; green, 166; blue, 35 }  ,fill opacity=1 ] (135.99,250.8) -- (380.02,50.35) -- (378.16,48.09) -- (395.29,41.6) -- (385.59,57.14) -- (383.73,54.88) -- (139.71,255.32) -- cycle ;
		
		% Text Node
		\draw (245.67,368.7) node   {$\theta $};
		% Text Node
		\draw (281.67,192.04) node [rotate=-2.44]  {$L$};
		% Text Node
		\draw (207.33,354.04) node [rotate=-0.74]  {$P_{in}$};
		% Text Node
		\draw (487.67,117.37) node [rotate=-0.74]  {$P_{out}$};
		% Text Node
		\draw (347.67,235.04) node [rotate=-0.61]  {$Q_{liq}$};
		% Text Node
		\draw  [color={rgb, 255:red, 0; green, 0; blue, 0 }  ,draw opacity=1 ][fill={rgb, 255:red, 245; green, 166; blue, 35 }  ,fill opacity=1 ]  (107, 268.37) circle [x radius= 25.3, y radius= 25.3]   ;
		\draw (107,268.37) node [scale=1.2,rotate=-359.71]  {$P_{calc}$};
		% Text Node
		\draw  [fill={rgb, 255:red, 245; green, 166; blue, 35 }  ,fill opacity=1 ]  (417.67, 24.37) circle [x radius= 23.2, y radius= 23.2]   ;
		\draw (417.67,24.37) node [scale=1.2,rotate=-0.74]  {$P_{out}$};
		
		
		\end{tikzpicture}		
		\caption{Схема расчёта распределения давления по потоку \mintinline{vb.net}{calc_along_flow=1}}
		\label{ris:Pipe_scheme_2}
	\end{center}
\end{figure} 

Схема расчета распределения давления по потоку для случая вертикальной добывающей скважины соответствует расчету распределения давления "снизу вверх" - от забойного давления к устьевому.

\begin{figure}[h!]
	\begin{center}
			
	\tikzset{every picture/.style={line width=0.75pt}} %set default line width to 0.75pt        
	
	\begin{tikzpicture}[x=0.75pt,y=0.75pt,yscale=-1,xscale=1]
	%uncomment if require: \path (0,453); %set diagram left start at 0, and has height of 453
	
	%Shape: Can [id:dp7385102204739014] 
	\draw  [fill={rgb, 255:red, 250; green, 245; blue, 184 }  ,fill opacity=1 ][line width=2.25]  (198.23,370.28) -- (505.15,118.62) .. controls (507.82,116.43) and (515.92,121.88) .. (523.23,130.8) .. controls (530.54,139.72) and (534.3,148.72) .. (531.63,150.92) -- (224.71,402.57)(198.23,370.28) .. controls (200.9,368.08) and (209,373.54) .. (216.31,382.45) .. controls (223.63,391.37) and (227.39,400.38) .. (224.71,402.57) .. controls (222.03,404.77) and (213.94,399.32) .. (206.62,390.4) .. controls (199.31,381.48) and (195.55,372.47) .. (198.23,370.28) -- cycle ;
	%Shape: Arc [id:dp4602786709186524] 
	\draw  [draw opacity=0] (247.54,383.1) .. controls (249.72,385.66) and (251.51,388.62) .. (252.76,391.95) .. controls (254.22,395.8) and (254.84,399.77) .. (254.7,403.64) -- (224.71,402.57) -- cycle ; \draw   (247.54,383.1) .. controls (249.72,385.66) and (251.51,388.62) .. (252.76,391.95) .. controls (254.22,395.8) and (254.84,399.77) .. (254.7,403.64) ;
	%Straight Lines [id:da5353213855148988] 
	\draw    (145.83,313.7) -- (198.23,370.28) ;
	
	
	%Straight Lines [id:da6371642797081225] 
	\draw    (460.06,65.51) -- (505.15,118.62) ;
	
	
	%Straight Lines [id:da5708425902799406] 
	\draw    (172.03,341.99) -- (482.6,92.07) ;
	
	
	%Straight Lines [id:da5694938455522887] 
	\draw    (377.67,250.37) -- (407,226.74) ;
	\draw [shift={(408.56,225.48)}, rotate = 501.14] [color={rgb, 255:red, 0; green, 0; blue, 0 }  ][line width=0.75]    (10.93,-3.29) .. controls (6.95,-1.4) and (3.31,-0.3) .. (0,0) .. controls (3.31,0.3) and (6.95,1.4) .. (10.93,3.29)   ;
	
	%Straight Lines [id:da47656732513988986] 
	\draw    (123,403.37) -- (603.56,403.37) ;
	\draw [shift={(605.56,403.37)}, rotate = 180] [color={rgb, 255:red, 0; green, 0; blue, 0 }  ][line width=0.75]    (10.93,-3.29) .. controls (6.95,-1.4) and (3.31,-0.3) .. (0,0) .. controls (3.31,0.3) and (6.95,1.4) .. (10.93,3.29)   ;
	
	%Right Arrow [id:dp49730840239995544] 
	\draw  [fill={rgb, 255:red, 245; green, 166; blue, 35 }  ,fill opacity=1 ] (419.38,64.16) -- (174.9,264.05) -- (176.75,266.31) -- (159.61,272.77) -- (169.34,257.26) -- (171.19,259.52) -- (415.67,59.63) -- cycle ;
	
	% Text Node
	\draw (267.67,388.7) node   {$\theta $};
	% Text Node
	\draw (303.67,212.04) node [rotate=-2.44]  {$L$};
	% Text Node
	\draw (229.33,374.04) node [rotate=-0.74]  {$P_{in}$};
	% Text Node
	\draw (509.67,137.37) node [rotate=-0.74]  {$P_{out}$};
	% Text Node
	\draw (369.67,255.04) node [rotate=-0.61]  {$Q_{liq}$};
	% Text Node
	\draw  [color={rgb, 255:red, 0; green, 0; blue, 0 }  ,draw opacity=1 ][fill={rgb, 255:red, 245; green, 166; blue, 35 }  ,fill opacity=1 ]  (444, 43.37) circle [x radius= 25.3, y radius= 25.3]   ;
	\draw (444,43.37) node [scale=1.2,rotate=-359.71]  {$P_{calc}$};
	% Text Node
	\draw  [fill={rgb, 255:red, 245; green, 166; blue, 35 }  ,fill opacity=1 ]  (129.67, 288.37) circle [x radius= 21.57, y radius= 21.57]   ;
	\draw (129.67,288.37) node [scale=1.2,rotate=-0.74]  {$P_{in}$};
	
	
	\end{tikzpicture}
		\caption{Схема расчета распределения давления против потока \mintinline{vb.net}{calc_along_flow=0}}
		\label{ris:Pipe_scheme_3}
	\end{center}
\end{figure} 

Схема расчета распределения давления против потока для случая вертикальной добывающей скважины соответствует расчету распределения давления "сверху вниз" - от устьевого давления к забойному.

\subsection{MF\_dp\_pipe\_atm – расчёт перепада давления в трубе}

Функция позволяет рассчитать перепад давления в участке трубопровода. 
Функция возвращает давление и температуру в виде массива.

\putlisting{listings/MF_dp_pipe_atm.lst}

Ниже на рисунке приведены результаты расчёта кривой оттока (перепада давления в вертикальной трубе) для различных корреляций, реализованных в \unf{}.

\begin{figure}[h!]
	
\begin{center}

\newcommand{\dPipeDataFile}{data/dPipe.txt}
\begin{tikzpicture}[scale=1]
\begin{axis}[
width=14cm,
height=10cm,
xlabel=$Q\; m^3 / day$,
ylabel=$P_{wf} \; atma$,
legend pos=south east,
title=Pipe Pressure Drop]
\addplot table [y=P_0, x=Q]{\dPipeDataFile};
\addlegendentry{Beggs Brill}
\addplot table [y=P_1, x=Q]{\dPipeDataFile};
\addlegendentry{Ansari}
\addplot table [y=P_2, x=Q]{\dPipeDataFile};
\addlegendentry{Unified}
\addplot table [y=P_3, x=Q]{\dPipeDataFile};
\addlegendentry{Gray}
\addplot table [y=P_4, x=Q]{\dPipeDataFile};
\addlegendentry{Hagedorn Brown}
\addplot table [y=P_5, x=Q]{\dPipeDataFile};
\addlegendentry{Sakharov Mokhov}
\end{axis}
\end{tikzpicture}


	\caption{Кривые характеристики многофазного потока для вертикальных труб рассчитанные с использованием различных корреляций }
\label{ris:VLP_curves}

\end{center}
\end{figure}

\subsection{MF\_p\_pipe\_atma – функция расчета давления на конце трубы}  

\putlisting{listings/MF_p_pipe_atma.lst}

\subsection{MF\_p\_pipe\_znlf\_atma – функция расчета давления на конце трубы при барботаже}  

\putlisting{listings/MF_p_pipe_znlf_atma.lst}

\subsection{MF\_dpdl\_atmm – функция расчета градиента давления по многофазной корреляции}  

\putlisting{listings/MF_dpdl_atmm.lst}

\newpage

