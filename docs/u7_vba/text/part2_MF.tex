\section{Расчёт свойств потока}

\subsection{MF\_qmix\_m3day – расход газожидкостной смеси}

Функция позволяет рассчитать объемный расход газожидкостной смеси при заданных термобарических условиях. 

$$Q_{mix} = Q_w B_w(P,T) + Q_o B_o(P,T)  + Q_o  (R_p - R_s(P,T)) B_g(P,T) $$

\putlisting{listings/MF_q_mix_rc_m3day.lst}

\subsection{MF\_rhomix\_kgm3 – плотность газожидкостной смеси}

Функция позволяет рассчитать плотность газожидкостной смеси при заданных термобарических условиях. 

\putlisting{listings/MF_rhomix_kgm3.lst}

\subsection{MF\_gas\_fraction\_d – доля газа в потоке}
Функция расчёта доли свободного газа в потоке (без учёта проскальзывания) в зависимости от термобарических условий для заданного флюида. 
В отличии от функций PVT учитывается обводнённость.
\putlisting{listings/MF_gas_fraction_d.lst}

\subsection{MF\_p\_gas\_fraction\_atma – целевое давления для заданной доли газа в потоке}
Функция расчёта давления при котором достигается заданная доля свободного газа в потоке (без учёта проскальзывания) . 
В отличии от функций PVT учитывается обводнённость.
Следует учитывать, что при вызове функции пересчитывается состояние смеси с различными термобарическими условиями.
\putlisting{listings/MF_p_gas_fraction_atma.lst}

\subsection{MF\_rp\_gas\_fraction\_m3m3 – целевой газовый фактор для заданной доли газа в потоке}
Функция расчёта давления при котором достигается заданная доля свободного газа в потоке (без учёта проскальзывания) . 
В отличии от функций PVT учитывается обводнённость.
Следует учитывать, что при вызове функции пересчитывается состояние смеси с различными термобарическими условиями.
\putlisting{listings/MF_rp_gas_fraction_m3m3.lst}

\section{Сепарация газа в скважине}

В скважинах оборудованных системами механизированной добычи нефти важную роль играет процесс сепарации газа на приёме насоса. Под сепарацией газа понимается отделение части свободного газа из потока и перенаправление его по отдельному гидравлическому каналу на поверхность. В результате сепарации газа меняются свойства флюида, поступающего в насос и НКТ выше насоса. Оценка величины сепарации может быть проведена приведёнными ниже функциями.

\subsection{MF\_ksep\_natural\_d – естественная сепарация газа}

Функция рассчитывает естественную сепарацию газа на приёме насоса в скважине с использованием корреляции Маркеса \cite{Marquez_2003} . Результат - безразмерная величина в диапазоне от 0 до 1. 

\putlisting{listings/MF_ksep_natural_d.lst}


\subsection{MF\_ksep\_gasseparator\_d – сепарация газа роторным газосепаратором}

Функция рассчитывает сепарацию газа с использованием роторного газосепаратора, являющегося обычно частью компоновки УЭЦН. Данный расчет основан на результатах испытания характеристик роторных газосепараторов, выполненных в РГУ нефти и газа имени И.М.Губкина \cite{SPE_117415_2008}. 

Следует отметить, что несмотря на хорошее соответствие промысловых исследований и расчетов с использованием корреляции для естественной и искусственной сепарации \cite{SPE_117415_2008} к результатам стендовых исследований стоит относится с осторожностью. Основой осторожности могут быть следующие соображения: характеристики различных газосепараторов достаточно сильно отличаются друг от друга - есть удачные конструкции и не очень, при этом результаты стендовых испытаний доступны только для ограниченного набора конструкций, стендовые условия достаточно сильно отличаются от скважинных - ниже давление, другие модельные рабочие жидкости, точно оценить коэффициент сепарации газосепаратора в промысловых условиях затруднительно - набор таких данных для сравнения ограничен. 

Тем не менее изучение результатов стендовых испытаний полезно при проведении расчетов и развивает инженерную интуицию. 

\putlisting{listings/MF_ksep_gasseparator_d.lst}


\subsection{MF\_ksep\_total\_d – общая сепарация газа}

Функция рассчитывает полную сепарацию газа на приёме насосе в скважине по известным значениям естественной сепарации газа и коэффициента сепарации газосепаратора. Результат - безразмерная величина в диапазоне от 0 до 1. 

\putlisting{listings/MF_ksep_total_d.lst}

\section{Расчёт многофазного потока в штуцере}

\subsection{Модель потока через штуцер}

%Тут надо будет нарисовать схему штуцера и пояснить что и как называется в коде. 

Штуцер или локальное гидравлическое сопротивление - элемент скважины или системы трубопроводов, применяемых для создания дополнительного перепада давления в системе и ограничения потока. 
Возможны различные варианты реализации штуцера - со штуцерной камерой, с угловым краном, позволяющим менять диаметр штуцера и другие.
Ключевым параметром штуцера является диаметр \(d_{choke} \) определяющий его способность к ограничению потока. 

Как и у любого элемента гидравлического потока есть три ключевых параметра - давление на входе \( P_{in} \) , \( P_0\), давление на выходе \(P_{out}\) , \( P_1\) и расход газожидкостной смеси, обычно задаваемый в стандартных условиях \(Q_{liq} \). Задание любых двух элементов позволяет вычислить третий. При задании трех элементов модель штуцера может быть настроена на замеры.
 

\subsection{MF\_p\_choke\_atma – Расчет давления на входе или на выходе штуцера}
Функция позволяет рассчитать давление на входе или выходе штуцера по известному давлению на противоположном конце при известных параметрах потока (дебите жидкости, обводненности, газовому фактору). Расчет проводится по корреляции Перкинса \cite{Perkins_1993} с учетом многофазного потока.  
\putlisting{listings/MF_p_choke_atma.lst}

\subsection{MF\_dp\_choke\_atm – Расчёт перепада давления в штуцере}
Функция позволяет рассчитать по известному линейному давлению и дебиту или по известному буферному давлению и дебиту перепад давления.  Расчет проводится по корреляции Перкинса \cite{Perkins_1993} с учетом многофазного потока.  
Функция возвращает перепад давления и температуры в виде массива.
\putlisting{listings/MF_dp_choke_atm.lst}


\subsection{MF\_qliq\_choke\_sm3day – функция расчёта дебита жидкости через штуцер}
Функция позволяет рассчитать по известному буферному давлению и линейному давлению дебит жидкости. Расчет проводится по корреляции Перкинса \cite{Perkins_1993} с учетом многофазного потока.  

\putlisting{listings/MF_qliq_choke_sm3day.lst}


\subsection{MF\_cf\_choke\_fr – функция настройки модели штуцера}
Функция позволяет рассчитать корректирующий фактор для модели штуцера, позволяющий согласовать результаты замеров давления и дебита. Расчет проводится по корреляции Перкинса \cite{Perkins_1993} с учетом многофазного потока.  

\putlisting{listings/MF_cf_choke_fr.lst}

\newpage
\section{Расчет многофазного потока в трубе}



\subsection{MF\_dp\_pipe\_atm – расчёт перепада давления в трубе}

Функция позволяет рассчитать перепад давления в участке трубопровода. 

Функция возвращает давление и температуру в виде массива.

\putlisting{listings/MF_dp_pipe_atm.lst}

Ниже на рисунке приведены результаты расчёта кривой оттока (перепада давления в вертикальной трубе) для различных корреляций, реализованных в \unf{}.

\newcommand{\dPipeDataFile}{data/dPipe.txt}
\begin{tikzpicture}[scale=1]
\begin{axis}[
width=14cm,
height=10cm,
xlabel=$Q\; m^3 / day$,
ylabel=$P_{wf} \; atma$,
legend pos=south east,
title=Pipe Pressure Drop]
\addplot table [y=P_0, x=Q]{\dPipeDataFile};
\addlegendentry{Beggs Brill}
\addplot table [y=P_1, x=Q]{\dPipeDataFile};
\addlegendentry{Ansari}
\addplot table [y=P_2, x=Q]{\dPipeDataFile};
\addlegendentry{Unified}
\addplot table [y=P_3, x=Q]{\dPipeDataFile};
\addlegendentry{Gray}
\addplot table [y=P_4, x=Q]{\dPipeDataFile};
\addlegendentry{Hagedorn Brown}
\addplot table [y=P_5, x=Q]{\dPipeDataFile};
\addlegendentry{Sakharov Mokhov}
\end{axis}
\end{tikzpicture}

\subsection{MF\_p\_pipe\_atma – функция расчета давления на конце трубы}  

\putlisting{listings/MF_p_pipe_atma.lst}

\subsection{MF\_p\_pipe\_znlf\_atma – функция расчета давления на конце трубы при барботаже}  

\putlisting{listings/MF_p_pipe_znlf_atma.lst}

\subsection{MF\_dpdl\_atmm – функция расчета градиента давления по многофазной корреляции}  

\putlisting{listings/MF_dpdl_atmm.lst}

\newpage

