
\section{Расчет многофазного потока в пласте}
Для анализа работы скважины и скважинного оборудования в большинстве случаев достаточно простейшего подхода для описания производительности пласта. На текущий момент в \unf{} используется линейная индикаторная кривая с поправкой Вогеля для учета разгазирования в призабойной зоне пласта с учетом обводненности \cite{KBrown_AL_methods_vol4}. 

Пользовательские функции для расчета производительности пласта начинаются с префикса  \mintinline{vb.net}{IPR_}. 

\subsection{IPR\_pi\_sm3dayatm – расчёт продуктивности}
Функция позволяет рассчитать коэффициент продуктивности скважины по данным тестовой эксплуатации. Особенность линейной модели притока к скважине с поправкой Волегя заключается в минимальном наборе исходных данных, необходимых для построения индикаторной кривой. Достаточно знать пластовое давление и дебит и забойное давление в одной точке.

\putlisting{listings/IPR_pi_sm3dayatm.lst}


\subsection{IPR\_pwf\_atm – расчёт забойного давления по дебиту и продуктивности}
Функция позволяет рассчитать забойное давление скважины по известным значениям дебита и продуктивности.

\putlisting{listings/IPR_pwf_atma.lst}

\subsection{IPR\_qliq\_sm3day – расчёт дебита по забойному давлению и продуктивности}
Функция позволяет рассчитать дебит жидкости скважины на поверхности по забойному давлению и продуктивности.

\putlisting{listings/IPR_qliq_sm3day.lst}



\newpage
