
\section{Расчет многофазного потока в пласте}
Для анализа работы скважины и скважинного оборудования в большинстве случаев достаточно простейшего подхода для описания производительности пласта. На текущий момент в \unf{} используется линейная индикаторная кривая с поправкой Вогеля для учета разгазирования в призабойной зоне пласта с учетом обводненности \cite{KBrown_AL_methods_vol4}. 

Пользовательские функции для расчета производительности пласта начинаются с префикса  \mintinline{vb.net}{IPR_}. 

Для расчета притока из пласта необходимо определить связь между дебитом жидкости $Q_{liq}$ (притоком) и забойным давлением работающей скважины $P_{wf}$.
Линейная индикаторная кривая на основе закона Дарси задает такую связь через коэффициент продуктивности скважины, который определяется как 
\begin{equation}\label{PI_def} 
 PI = \frac{Q_{liq}}{P_{res} - P_{wf}} 
\end{equation}

где $P_{res}$ - пластовое давление - давление на контуре питания скважины. Закон Дарси описывает установившийся приток несжимаемой жидкости в однородном пласте. 

Соответственно уравнение притока будет иметь вид

$$ Q_{liq} = PI \left( P_{res} - P_{wf} \right( $$

Для линейного притока по закону Дарси коэффициент продуктивности может быть оценен либо по данным эксплуатации из уравнения \ref{PI_def} либо по аналитической зависимости по характеристикам пласта и системы заканчивания. Например для радиального притока к вертикальной скважине широко известна формула Дюпюи согласно которой 
\begin{equation}\label{eq_Dupui} 
PI = f\frac{kh}{\mu B}\frac{1}{ \ln \frac{r_e}{r_w} -A + S }  
\end{equation}

здесь $f$ - размерный коэффициент, зависящий от выбранной системы единиц для остальных параметров. Так для системы единиц



\begin{table}[]
	\begin{tabular}{|c|c|c|c|c|}
		\hline
		обозначение & Параметр                                                         & СИ          & \begin{tabular}[c]{@{}c@{}}Практические \\ метрические\end{tabular} & \begin{tabular}[c]{@{}c@{}}Американские\\ промысловые\end{tabular} \\ \hline
		$f$         & \begin{tabular}[c]{@{}c@{}}размерный \\ коэффициент\end{tabular} & $2\pi$      & $\dfrac{1}{18.41}$                                                  & $\dfrac{1}{141.2}$                                                 \\ \hline
		$k$         & проницаемость                                                    & $\text{м}^2$         & мД                                                                  & mD                                                                 \\ \hline
		$h$         & \begin{tabular}[c]{@{}c@{}}мощность \\ пласта\end{tabular}       & м           & м                                                                   & ft                                                                 \\ \hline
		
		$B$         & \begin{tabular}[c]{@{}c@{}}объемный \\ коэффициент\end{tabular}  & $\text{м}^3/\text{м}^3$    & $\text{м}^3/\text{м}^3$                                                                  & scf / bbl                                                                 \\ \hline
		$\mu$      & вязкость                                                         & Па $\cdot$ с   & сП                                                                  & cP                                                                 \\ \hline
	\end{tabular}
\end{table}


\subsection{IPR\_pi\_sm3dayatm – расчёт продуктивности}
Функция позволяет рассчитать коэффициент продуктивности скважины по данным тестовой эксплуатации. Особенность линейной модели притока к скважине с поправкой Волегя заключается в минимальном наборе исходных данных, необходимых для построения индикаторной кривой. Достаточно знать пластовое давление и дебит и забойное давление в одной точке.

\putlisting{listings/IPR_pi_sm3dayatm.lst}


\subsection{IPR\_pwf\_atm – расчёт забойного давления по дебиту и продуктивности}
Функция позволяет рассчитать забойное давление скважины по известным значениям дебита и продуктивности.

\putlisting{listings/IPR_pwf_atma.lst}

\subsection{IPR\_qliq\_sm3day – расчёт дебита по забойному давлению и продуктивности}
Функция позволяет рассчитать дебит жидкости скважины на поверхности по забойному давлению и продуктивности.

\putlisting{listings/IPR_qliq_sm3day.lst}



\newpage
