%&preformat-disser
\RequirePackage[l2tabu,orthodox]{nag} % Раскомментировав, можно в логе получать рекомендации относительно правильного использования пакетов и предупреждения об устаревших и нерекомендуемых пакетах
% Формат А4, 14pt (ГОСТ Р 7.0.11-2011, 5.3.6)
\documentclass[a4paper,14pt,oneside,openany]{memoir}


% файл с настройками документов, для того чтобы транслировать разные настройки по нескольким документам
% основан на шаблоне диссертации  https://github.com/AndreyAkinshin/Russian-Phd-LaTeX-Dissertation-Template
% Хабибуллин Ринат 2020

% version definition
\newcommand{\unf}{Unifloc 7.16 VBA}



%настройки из Russian-Phd-LaTeX-Dissertation-Template
%% Режим черновика
\makeatletter
\@ifundefined{c@draft}{
  \newcounter{draft}
  \setcounter{draft}{0}  % 0 --- чистовик (максимальное соблюдение ГОСТ)
                         % 1 --- черновик (отклонения от ГОСТ, но быстрая сборка итоговых PDF)
}{}
\makeatother


%%% Использование в xelatex и lualatex семейств шрифтов %%%
\makeatletter
\@ifundefined{c@fontfamily}{
  \newcounter{fontfamily}
  \setcounter{fontfamily}{1}  % 0 --- CMU семейство. Используется как fallback;
                              % 1 --- Шрифты от MS (Times New Roman и компания)
                              % 2 --- Семейство Liberation
}{}
\makeatother

%% Библиография


%%% Предкомпиляция tikz рисунков для ускорения работы %%%
\makeatletter
\@ifundefined{c@imgprecompile}{
  \newcounter{imgprecompile}
  \setcounter{imgprecompile}{0}   % 0 --- без предкомпиляции;
                                  % 1 --- пользоваться предварительно скомпилированными pdf вместо генерации заново из tikz
}{}
\makeatother
            % общие настройки шаблона  
%%% Проверка используемого TeX-движка %%%
\RequirePackage{ifxetex, ifluatex}
\newif\ifxetexorluatex   % определяем новый условный оператор (http://tex.stackexchange.com/a/47579)
\ifxetex
    \xetexorluatextrue
\else
    \ifluatex
        \xetexorluatextrue
    \else
        \xetexorluatexfalse
    \fi
\fi

\RequirePackage{etoolbox}[2015/08/02]               % Для продвинутой проверки разных условий

%%% Поля и разметка страницы %%%
\usepackage{pdflscape}                              % Для включения альбомных страниц
\usepackage{geometry}                               % Для последующего задания полей

%%% Математические пакеты %%%
\usepackage{amsthm,amsmath,amscd}   % Математические дополнения от AMS
\usepackage{amsfonts,amssymb}       % Математические дополнения от AMS
\usepackage{mathtools}              % Добавляет окружение multlined
\usepackage{unicode-math}           % использование Unicode-шрифтов для формул

%%%% Установки для размера шрифта 14 pt %%%%
%% Формирование переменных и констант для сравнения (один раз для всех подключаемых файлов)%%
%% должно располагаться до вызова пакета fontspec или polyglossia, потому что они сбивают его работу
\newlength{\curtextsize}
\newlength{\bigtextsize}
\setlength{\bigtextsize}{13.9pt}

\makeatletter
%\show\f@size                                       % неплохо для отслеживания, но вызывает стопорение процесса, если документ компилируется без команды  -interaction=nonstopmode 
\setlength{\curtextsize}{\f@size pt}
\makeatother

%%% Кодировки и шрифты %%%

\usepackage{polyglossia}[2014/05/21]            % Поддержка многоязычности (fontspec подгружается автоматически)


%%% Оформление абзацев %%%
\usepackage{indentfirst}                            % Красная строка

%%% Цвета %%%
\usepackage[dvipsnames, table, hyperref, cmyk]{xcolor} % Совместимо с tikz. Конвертация всех цветов в cmyk заложена как удовлетворение возможного требования типографий. Возможно конвертирование и в rgb.

%%% Таблицы %%%
\usepackage{longtable,ltcaption}                    % Длинные таблицы
\usepackage{multirow,makecell}                      % Улучшенное форматирование таблиц
\usepackage{pbox}

%%% Общее форматирование
\usepackage{soulutf8}                               % Поддержка переносоустойчивых подчёркиваний и зачёркиваний
\usepackage{icomma}                                 % Запятая в десятичных дробях

%%% Оптимизация расстановки переносов и длины последней строки абзаца
\ifluatex
    \ifnumequal{\value{draft}}{1}{% Черновик
        \usepackage[hyphenation, lastparline, nosingleletter, homeoarchy,
        rivers, draft]{impnattypo}
    }{% Чистовик
        \usepackage[hyphenation, lastparline, nosingleletter]{impnattypo}
    }
\else
    \usepackage[hyphenation, lastparline]{impnattypo}
\fi

%%% Гиперссылки %%%
\usepackage{hyperref}[2012/11/06]

%%% Изображения %%%
\usepackage{graphicx}[2014/04/25]                   % Подключаем пакет работы с графикой

%%% Списки %%%
\usepackage{enumitem}

%%% Счётчики %%%
\usepackage[figure,table]{totalcount}               % Счётчик рисунков и таблиц
\usepackage{totcount}                               % Пакет создания счётчиков на основе последнего номера подсчитываемого элемента (может требовать дважды компилировать документ)
\usepackage{totpages}                               % Счётчик страниц, совместимый с hyperref (ссылается на номер последней страницы). Желательно ставить последним пакетом в преамбуле

%%% Продвинутое управление групповыми ссылками (пока только формулами) %%%
\usepackage{cleveref}                              % cleveref корректно считывает язык из настроек polyglossia

\creflabelformat{equation}{#2#1#3}                  % Формат по умолчанию ставил круглые скобки вокруг каждого номера ссылки, теперь просто номера ссылок без какого-либо дополнительного оформления
\crefrangelabelformat{equation}{#3#1#4\cyrdash#5#2#6}   % Интервалы в русском языке принято делать через тире, если иное не оговорено


\ifnumequal{\value{draft}}{1}{% Черновик
    \usepackage[firstpage]{draftwatermark}
    \SetWatermarkText{DRAFT}
    \SetWatermarkFontSize{14pt}
    \SetWatermarkScale{15}
    \SetWatermarkAngle{45}
}{}

%%% Цитата, не приводимая в автореферате:
% возможно, актуальна только для biblatex
%\newcommand{\citeinsynopsis}[1]{\ifsynopsis\else ~\cite{#1} \fi}


%%% Прикладные пакеты %%% 
%\usepackage{calc}               % Пакет для расчётов параметров, например длины

%%% Для добавления Стр. над номерами страниц в оглавлении
%%% http://tex.stackexchange.com/a/306950
\usepackage{afterpage}

\usepackage{tikz}                   % Продвинутый пакет векторной графики
\usetikzlibrary{chains}             % Для примера tikz рисунка
\usetikzlibrary{shapes.geometric}   % Для примера tikz рисунка
\usetikzlibrary{shapes.symbols}     % Для примера tikz рисунка
\usetikzlibrary{arrows}             % Для примера tikz рисунка
\ifnumequal{\value{imgprecompile}}{1}{% Только если у нас включена предкомпиляция
	\usetikzlibrary{external}   % подключение возможности предкомпиляции
	\tikzexternalize[prefix=Dissertation/images/] % activate! % здесь можно указать отдельную папку для скомпилированных файлов
	\ifxetex
	\tikzset{external/up to date check={diff}}
	\fi
}{}


\usepackage{tabu, tabulary}  %таблицы с автоматически подбирающейся шириной столбцов
\usepackage{fr-longtable}    %ради \endlasthead



% Русская традиция начертания греческих букв
\usepackage{upgreek} % прямые греческие ради русской традиции

%%% Микротипографика
%\ifnumequal{\value{draft}}{0}{% Только если у нас режим чистовика
%    \usepackage[final, babel, shrink=45]{microtype}[2016/05/14] % улучшает представление букв и слов в строках, может помочь при наличии отдельно висящих слов
%}{}

% Отметка о версии черновика на каждой странице
% Чтобы работало надо в своей локальной копии по инструкции
% https://www.ctan.org/pkg/gitinfo2 создать небходимые файлы в папке
% ./git/hooks
% If you’re familiar with tweaking git, you can probably work it out for
% yourself. If not, I suggest you follow these steps:
% 1. First, you need a git repository and working tree. For this example,
% let’s suppose that the root of the working tree is in ~/compsci
% 2. Copy the file post-xxx-sample.txt (which is in the same folder of
% your TEX distribution as this pdf) into the git hooks directory in your
% working copy. In our example case, you should end up with a file called
% ~/compsci/.git/hooks/post-checkout
% 3. If you’re using a unix-like system, don’t forget to make the file executable.
% Just how you do this is outside the scope of this manual, but one
% possible way is with commands such as this:
% chmod g+x post-checkout.
% 4. Test your setup with “git checkout master” (or another suitable branch
% name). This should generate copies of gitHeadInfo.gin in the directories
% you intended.
% 5. Now make two more copies of this file in the same directory (hooks),
% calling them post-commit and post-merge, and you’re done. As before,
% users of unix-like systems should ensure these files are marked as
% executable.
\ifnumequal{\value{draft}}{1}{% Черновик
	\IfFileExists{.git/gitHeadInfo.gin}{                                        
		\usepackage[mark,pcount]{gitinfo2}
		\renewcommand{\gitMark}{rev.\gitAbbrevHash\quad\gitCommitterEmail\quad\gitAuthorIsoDate}
		\renewcommand{\gitMarkFormat}{\rmfamily\color{Gray}\small\bfseries}
	}{}
}{}


% для раскраски листингов кода
\usepackage{minted}
\usemintedstyle{vs}
\usepackage{tcolorbox}

\usepackage{pgfplotstable}
\usepackage{pgfplots}
\pgfplotsset{compat=1.12}

\tcbuselibrary{breakable,skins,minted}

\renewcommand{\listingscaption}{Листинг}

\newcommand{\putlisting}[1]{
	\tcbinputlisting{
		listing file=#1,
		minted language=vb.net,
		minted options={breaklines,fontsize=\footnotesize},% <-- put other minted options inside the brackets
		breakable,enhanced,% <-- put other tcolorbox options here
		listing only
	}
}


         % Пакеты общие для диссертации и автореферата 
\input{../setup/common/dispackages}      % Пакеты общие для диссертации и автореферата
\input{../setup/common/userpackages}     % Пакеты общие для диссертации и автореферата
\input{../setup/common/setup_template}   % Упрощённые настройки шаблона  
\input{../setup/common/newnames}         % Новые переменные, для всего проекта
\input{../setup/common/fonts}            % Определение шрифтов (частичное)
\input{../setup/common/styles}           % Стили общие для диссертации и автореферата

%%% Изображения %%%
\graphicspath{{pics/}}

\input{../setup/common/disstyles}  		 % Стили для диссертации
% для вертикального центрирования ячеек в tabulary
\def\zz{\ifx\[$\else\aftergroup\zzz\fi}
%$ \] % <-- чиним подсветку синтаксиса в некоторых редакторах
\def\zzz{\setbox0\lastbox
\dimen0\dimexpr\extrarowheight + \ht0-\dp0\relax
\setbox0\hbox{\raise-.5\dimen0\box0}%
\ht0=\dimexpr\ht0+\extrarowheight\relax
\dp0=\dimexpr\dp0+\extrarowheight\relax 
\box0
}




%Общие счётчики окружений листингов
%http://tex.stackexchange.com/questions/145546/how-to-make-figure-and-listing-share-their-counter
% Если смешивать плавающие и не плавающие окружения, то могут быть проблемы с нумерацией
\makeatletter
\AtBeginDocument{%
    \let\c@ListingEnv\c@lstlisting
    \let\theListingEnv\thelstlisting
    \let\ftype@lstlisting\ftype@ListingEnv % give the floats the same precedence
}
\makeatother

% значок С++ — используйте команду \cpp
\newcommand{\cpp}{%
    C\nolinebreak\hspace{-.05em}%
    \raisebox{.2ex}{+}\nolinebreak\hspace{-.10em}%
    \raisebox{.2ex}{+}%
}

%%%  Чересстрочное форматирование таблиц
%% http://tex.stackexchange.com/questions/278362/apply-italic-formatting-to-every-other-row
\newcounter{rowcnt}
\newcommand\altshape{\ifnumodd{\value{rowcnt}}{\color{red}}{\vspace*{-1ex}\itshape}}
% \AtBeginEnvironment{tabular}{\setcounter{rowcnt}{1}}
% \AtEndEnvironment{tabular}{\setcounter{rowcnt}{0}}

%%% Ради примера во второй главе
\let\originalepsilon\epsilon
\let\originalphi\phi
\let\originalkappa\kappa
\let\originalle\le
\let\originalleq\leq
\let\originalge\ge
\let\originalgeq\geq
\let\originalemptyset\emptyset
\let\originaltan\tan
\let\originalcot\cot
\let\originalcsc\csc

%%% Русская традиция начертания математических знаков
\renewcommand{\le}{\ensuremath{\leqslant}}
\renewcommand{\leq}{\ensuremath{\leqslant}}
\renewcommand{\ge}{\ensuremath{\geqslant}}
\renewcommand{\geq}{\ensuremath{\geqslant}}
\renewcommand{\emptyset}{\varnothing}

%%% Русская традиция начертания математических функций (на случай копирования из зарубежных источников)
\renewcommand{\tan}{\operatorname{tg}}
\renewcommand{\cot}{\operatorname{ctg}}
\renewcommand{\csc}{\operatorname{cosec}}

%%% Русская традиция начертания греческих букв (греческие буквы вертикальные, через пакет upgreek)
\renewcommand{\epsilon}{\ensuremath{\upvarepsilon}}   %  русская традиция записи
\renewcommand{\phi}{\ensuremath{\upvarphi}}
%\renewcommand{\kappa}{\ensuremath{\varkappa}}
\renewcommand{\alpha}{\upalpha}
\renewcommand{\beta}{\upbeta}
\renewcommand{\gamma}{\upgamma}
\renewcommand{\delta}{\updelta}
\renewcommand{\varepsilon}{\upvarepsilon}
\renewcommand{\zeta}{\upzeta}
\renewcommand{\eta}{\upeta}
\renewcommand{\theta}{\uptheta}
\renewcommand{\vartheta}{\upvartheta}
\renewcommand{\iota}{\upiota}
\renewcommand{\kappa}{\upkappa}
\renewcommand{\lambda}{\uplambda}
\renewcommand{\mu}{\upmu}
\renewcommand{\nu}{\upnu}
\renewcommand{\xi}{\upxi}
\renewcommand{\pi}{\uppi}
\renewcommand{\varpi}{\upvarpi}
\renewcommand{\rho}{\uprho}
%\renewcommand{\varrho}{\upvarrho}
\renewcommand{\sigma}{\upsigma}
%\renewcommand{\varsigma}{\upvarsigma}
\renewcommand{\tau}{\uptau}
\renewcommand{\upsilon}{\upupsilon}
\renewcommand{\varphi}{\upvarphi}
\renewcommand{\chi}{\upchi}
\renewcommand{\psi}{\uppsi}
\renewcommand{\omega}{\upomega}
 		 % Стили для специфических пользовательских задач


%%% Библиография. Выбор движка для реализации %%%
% Здесь только проверка установленного ключа. Сама настройка выбора движка
% размещена в common/setup.tex
\ifnumequal{\value{bibliosel}}{0}{%
	\input{../biblio/predefined}   % Встроенная реализация с загрузкой файла через движок bibtex8
}{
	\input{../biblio/biblatex}     % Реализация пакетом biblatex через движок biber
}

% дополнительный набор пакетов (не входит в Russian-Phd-LaTeX-Dissertation-Template)
% для раскраски листингов кода
\usepackage{minted}
\usemintedstyle{vs}
\usepackage{tcolorbox}

\usepackage{pgfplotstable}
\usepackage{pgfplots}
\pgfplotsset{compat=1.12}

\tcbuselibrary{breakable,skins,minted}

\renewcommand{\listingscaption}{Листинг}

\newcommand{\putlisting}[1]{
	\tcbinputlisting{
		listing file=#1,
		minted language=vb.net,
		minted options={breaklines,fontsize=\footnotesize},% <-- put other minted options inside the brackets
		breakable,enhanced,% <-- put other tcolorbox options here
		listing only
	}
}     % Пакеты дополнительные для раскраски кода 

% Вывести информацию о выбранных опциях в лог сборки
\typeout{Selected options:}
\typeout{Draft mode: \arabic{draft}}
\typeout{Font: \arabic{fontfamily}}
\typeout{AltFont: \arabic{usealtfont}}
\typeout{Bibliography backend: \arabic{bibliosel}}
\typeout{Precompile images: \arabic{imgprecompile}}
% Вывести информацию о версиях используемых библиотек в лог сборки
\listfiles







% version definition
\newcommand{\unf}{Unifloc 7.7 VBA}
\usemintedstyle{vs}

\graphicspath{{pics/}}

\begin{document}
	%\input{../setup/common/names}                 % Переопределение именований
	%%% Структура диссертации (ГОСТ Р 7.0.11-2011, 4)
	% Титульный лист (ГОСТ Р 7.0.11-2001, 5.1)
\thispagestyle{empty}
\begin{center}

\end{center}
%
\vspace{0pt plus4fill} %число перед fill = кратность относительно некоторого расстояния fill, кусками которого заполнены пустые места
\IfFileExists{images/logo.pdf}{
  \begin{minipage}[b]{0.5\linewidth}
    \begin{flushleft}
%      \includegraphics[height=3.5cm]{logo}
    \end{flushleft}
  \end{minipage}%
  \begin{minipage}[b]{0.5\linewidth}
    \begin{flushright}
      На правах рукописи\\
%      \textsl {УДК \thesisUdk}
    \end{flushright}
  \end{minipage}
}{
\begin{flushright}
На правах рукописи

%\textsl {УДК \thesisUdk}
\end{flushright}
}
%
\vspace{0pt plus6fill} %число перед fill = кратность относительно некоторого расстояния fill, кусками которого заполнены пустые места
\begin{center}
{\large \thesisTitle}
\end{center}
%
\vspace{0pt plus1fill} %число перед fill = кратность относительно некоторого расстояния fill, кусками которого заполнены пустые места
\begin{center}
\textbf {\large %\MakeUppercase
Unifloc 7 VBA}

\vspace{0pt plus2fill} %число перед fill = кратность относительно некоторого расстояния fill, кусками которого заполнены пустые места
{%\small

}

\vspace{0pt plus2fill} %число перед fill = кратность относительно некоторого расстояния fill, кусками которого заполнены пустые места
\unf


\end{center}
%
\vspace{0pt plus4fill} %число перед fill = кратность относительно некоторого расстояния fill, кусками которого заполнены пустые места
\begin{flushright}



\end{flushright}
%
\vspace{0pt plus4fill} %число перед fill = кратность относительно некоторого расстояния fill, кусками которого заполнены пустые места
{\centering Москва 2020\par}
           % Титульный лист
	\include{text/contents}        % Оглавление
	\chapter*{Введение}                         % Заголовок
\addcontentsline{toc}{chapter}{Введение}    % Добавляем его в оглавление

Документ описывает расчётный модуль \unf реализованный в Excel VBA. Модуль предназначен для изучения математических моделей систем нефтедобычи и развития навыков проведения инженерных расчётов.

Расчётный модуль охватывают основные элементы математических моделей систем нефтедобычи - модель физико-химических свойств пластовых флюидов, модели многофазного потока в трубах, в пласте, задачи узлового анализа, модели скважинного оборудования в частности УЭЦН.  

Для использования \unf требуются навыки уверенного пользователя MS Excel, желательно знание основ программирования и основ теории добычи нефти. 

Алгоритмы реализованные в расчётном модуле не претендуют на полноту и достоверность и ориентированы на учебные задачи и проведение простых расчётов. Руководство пользователя также не претендует на полноту описания системы (часто получается, что описание отстаёт от текущего состояния \unf). Все приводится как есть. Более надёжным способом получения достоверной информации о работе макросов \unf является изучение непосредственно расчётного кода в редакторе VBE.

По всем вопросам можно обращаться к автору расчётных модулей - Хабибуллину Ринату Альфредовичу (khabibullin.ra@gubkin.ru)  

    % Введение
	\chapter{Макросы VBA для проведения расчётов}

Расчёты \unf{} выполняются с использованием макросов, написанных на языке программирования Visual Basic for Application (VBA), встроенном в Excel [\href{https://ru.wikipedia.org/wiki/Visual_Basic_for_Applications}{wikipedia VBA}]. 

Макросы \unf{} могут быть использованы различными способами. В самом простом варианте для использования \unf{} не требуется программировать, достаточно уметь вызывать необходимые функции из рабочей книги Excel, создавая расчетные модули. В более сложном и мощном варианте использования на основе функций \unf{} можно создавать свои макросы, которые могут быть вызваны, например, по нажатию кнопки. Это упрощает проведение больших и массовых расчетов, но требует базовых навыков программирования. Самый продвинутый вариант подразумевает создание собственных программ на основе объектной модели \unf{}. 


Исходный код расчётных модулей находится в отдельном файле - надстройке Excel - файле с расширением.xlam. Для использования макросов данная надстройка должна быть запущена в программе Excel при проведении расчетов. Ее можно каждый раз запускать вручную или установить для автоматического запуска при старте Excel. Подробное описание процедуры установки надстройки можно найти на сайте Microsoft по ключевым словам  \href{https://support.office.com/ru-ru/article/%D0%94%D0%BE%D0%B1%D0%B0%D0%B2%D0%BB%D0%B5%D0%BD%D0%B8%D0%B5-%D0%B8-%D1%83%D0%B4%D0%B0%D0%BB%D0%B5%D0%BD%D0%B8%D0%B5-%D0%BD%D0%B0%D0%B4%D1%81%D1%82%D1%80%D0%BE%D0%B5%D0%BA-%D0%B2-excel-0af570c4-5cf3-4fa9-9b88-403625a0b460}{"добавление и удаление надстроек в Excel"}.


\section{Работа с VBA}

\section{Установка надстройки для автоматического запуска}
\begin{enumerate}
	\item На вкладке Файл выберите команду Параметры, а затем — категорию Надстройки.
	\item В поле Управление выберите пункт Надстройки Excel, а затем нажмите кнопку Перейти. Откроется диалоговое окно Надстройки.
	\item Чтобы установить и активировать надстройку Унифлок 7.1, нажмите кнопку Обзор (в диалоговом окне Надстройки), найдите надстройку, а затем нажмите кнопку ОК.
	\item Надстройка появится в списке надстроек. Галочка активации надстройки должна быть установлена
\end{enumerate}	

После установки и активации надстройки, встроенными в нее макросами можно будет пользоваться в любой книге Excel на данным компьюетере. При переносе расчётных файлов на другой комппьютер для сохранения их работоспособности должна быть передана и установлена и надстройка. 

\section{Ручной запуск надстройки}
В некоторых случаях может быть удобен альтернативный способ работы с надстройкой, не требующий ее установки на компьютере. Это бывает удобно, когда версия настройки часто меняется. Для этого необходимо открыть файл надстройки непосредственно в Excel, например двойным щелчком по файлу с расширением.xlam в проводнике. При этом Excel откроется, но никаких документов в нем не появится, а сама надстройка будет загружена и готова к использованию. Следует обратить внимание, что при таком варианте работы с надстройкой при открытии файла использующего макросы \unf{} сохраненных на другом коспьютере может возникать сообщение, что связанный файл надстройки не найден на новом компьютере. В этом случае в окне запроса следует выбрать кнопку "изменить"\ и указать правильное положение файла надстройки.

\section{Редактор VBE}
Чтобы получить доступ к макросам в текущей версии расчётного модуля для выполнения упражнений необходимо:
\begin{itemize}
	\item Запустить Excel запустив рабочую книгу для выполнения упражнений
	\item Нажать комбинацию клавиш <Alt-F11>
	\item Откроется новое окно c редактором макросов VBA (Рис. \ref{ris:VBA_overview}). Иногда в литературе окно редактирования макросов обозначают как VBE (Visual Basic Enviroment)
	\item Окне VBE можно изучить структуру проекта (набора макросов и других элементов). Раздел со структурой проекта можно открыть из меню <Вид – Обозреватель проекта>. Макросы располагаются в ветках «модули» и «модули классов»
	 
\end{itemize}

\begin{figure}[ht]
	\center{\includegraphics[width=1\linewidth]{VBA_overview}}
	\caption{Окно редактора VBE}
	\label{ris:VBA_overview}
\end{figure}


\section{Особенности VBA и соглашения \unf{}}
Строки, начинающиеся со знака ‘ являются комментариями. В VBE они выделяются зелёным цветом. На исполнение макросов не влияют.

Для многих макросов не обязательно задавать все параметры. Некоторые значения параметров могут не задаваться – тогда будут использованы значения параметров, принятые по умолчанию. Параметры, допускающие задание по умолчанию, помечены в исходном коде ключевым словом \mintinline{vb.net}{Optional}.

При создании макросов в основном использовались международные обозначения переменных, принятые в монографиях общества инженеров нефтяников SPE. Список наиболее употребимых обозначений приведен в приложении. 

При создании макросов для обозначения переменных разработчики старались придерживаться следующих соглашений (не всегда успешно впрочем)
\begin{itemize}
	\item название переменной или функции отражает физический смысл 
	\item лучше длинное и понятное название, чем короткое и непонятное, разделители слов в названиях - знаки подчеркивания (там, где это возможно)
	\item для расчетных функций название может содержать (последовательно) - префикс, указывающий группу функций, расчетное значение, ключевые параметры, на основе которых проводится расчет, размерность результата
	\item для минимизации путаницы с размерностями физических величин все размерные переменные в названии содержат явное указание размерности
\end{itemize}

           % Глава 1
	\chapter{Функции модуля «u7\_Excel\_functions»}
\section{Расчёт физико-химических свойств флюидов (PVT)}
Для расчёта физико-химических свойств пластовых флюидов используется модель нелетучей нефти. Для всех функций, реализующих расчёт с учётом PVT свойств необходимо задавать одинаковый полный набор параметров, описывающих нефть, газ и воду.  При этом для некоторых частных функций не все параметры будут влиять на результат расчёта, тем не менее эти параметры необходимо задавать. Это сделано для унификации методик расчёта – при любом вызове функции проводится расчёт всех свойств модели нелетучей нефти, но возвращаются только необходимые данные. Это обстоятельности может замедлить расчёты с использованием функций Excel.
 
\subsection{Обозначения PVT параметров}
Типовой набор параметров приведён ниже:

%\putlisting{listings/uf7_vars.lst}


\begin{itemize}
	
\item	$\gamma_g$  - \mintinline{vb.net}{gamma_gas} - удельная плотность газа, по воздуху. Стандартное обозначение переменной \mintinline{vb.net}{gamma_gas}. Безразмерная величина. Следует обратить внимание, что удельная плотность газа по воздуху не совпадает с плотностью воздуха в г/см3, поскольку плотность воздуха при стандартных условиях \mintinline[breaklines]{vb.net}{Const const_rho_air = 1.205} при температуре 20 °С и давлении 101325 Па для сухого воздуха. По умолчанию задается значение \mintinline[breaklines]{vb.net}{const_gg_default = 0.6}

\item $\gamma_o$  - \mintinline{vb.net}{gamma_oil} - удельная плотность нефти, по воде. Стандартное обозначение переменной \mintinline{vb.net}{gamma_oil}. Безразмерная величина, но по значению совпадает с плотность в г/см3. По умолчанию задаётся значение \mintinline{vb.net}{const_go_default = 0.86}

\item $\gamma_w$  - \mintinline{vb.net}{gamma_wat}- удельная плотность воды, по воде. Стандартное обозначение переменной \mintinline{vb.net}{gamma_wat}. Безразмерная величина, но по значению совпадает с плотность в г/см3. По умолчанию задаётся значение \mintinline{vb.net}{const_gw_default = 1} Плотность воды может отличаться от задаваемой по умолчанию, например для воды с большой минерализацией.  

\item $R_{sb}$- газосодержание при давлении насыщения, м3/м3. Стандартное обозначение в коде \mintinline{vb.net}{Rsb_m3m3}. Значение по умолчанию \mintinline{vb.net}{const_Rsb_default = 100}

\item $R_p$-  замерной газовый фактор, м3/м3. Стандартное обозначение в коде \mintinline{vb.net}{Rp_m3m3}. Калибровочный параметр. По умолчанию используется значение равное газосодержанию при давлении насыщения. Если задаётся значение меньшее чем газосодержание при давлении насыщения, то последнее принимается равным газовому фактору (приоритет у газового фактора, потому что как правило это замерное значение в отличии от газосодержания определяемого по результатам лабораторных исследований проб нефти).

\item $P_b$ - давление насыщения, атм. Стандартное обозначение в коде \mintinline{vb.net}{Pb_atm}. Калибровочный параметр. По умолчанию не задаётся, рассчитывается по корреляции. Если задан, то все расчёты по корреляциям корректируются с учётом заданного параметра. При задании давления насыщения обязательно должна быть задана температура пласта – температура при которой было определено давление насыщения. 

\item $T_{res}$- пластовая температура, \textcelsius. Стандартное обозначение в коде \mintinline{vb.net}{Tres_C}. Учитывается при расчёте давления насыщения. По умолчанию принято значение 90 \textcelsius.

\item $B_{ob}$ - объёмный коэффициент нефти, м3/м3. Стандартное обозначение в коде \mintinline{vb.net}{Bob_m3m3}. Калибровочный параметр. По умолчанию рассчитывается по корреляции. Если задан, то все расчёты по корреляциям корректируются с учётом заданного параметра.

\item $\mu_{ob}$ - вязкость нефти при давлении насыщения, сП. Стандартное обозначение \mintinline{vb.net}{Muob_cP}. Калибровочный параметр. По умолчанию рассчитывается по корреляции. Если задан, то все расчёты по корреляциям корректируются с учётом заданного параметра.

\item PVTcorr - номер набора PVT корреляций используемых для расчёта. 
\begin{itemize}	
	\item 	StandingBased = 0 - на основе корреляции Стендинга
	\item 	McCainBased = 1 - на основе корреляции Маккейна
	\item 	StraigthLine = 2 - на основе упрощённых зависимостей
\end{itemize}

\item PVTstr - закодированная строка с параметрами PVT. Если задана - перекрывает другие значения. Позволяет задать PVT параметры ссылкой всего на одну ячейку в Excel. Введена для удобства использования функций с большим числом параметров из Excel. Может быть сгенерирована вызовом функции \mintinline{vb.net}{PVT_Encode_string}.

\item $K_s$ – коэффициент сепарации газа. Определяет изменение свойств флюида после отделения части газа из потока в результате сепарации при определённых давлении и температуре. По умолчанию предполагается, что сепарации нет $K_s$=0. Для корректного задания свойств флюида после сепарации части газа необходимо также задать параметры $P_{ksep}$, $T_{ksep}$

\item $P_{ksep}$ - Давление при которой произошла сепарация части газа. Необходимо для расчёта свойств флюида с учётом сепарации. 

\item $T_{ksep}$ - Температура при которой произошла сепарация части газа. Необходимо для расчёта свойств флюида с учётом сепарации. 

\end{itemize}

\subsection{Стандартные условия} 
Многие параметры нефти, газа и воды существенно зависят от давления и температуры. Например объем занимаемый определённым количеством газа примерно в два раза снизится при повышении давления в два раза. 

Поэтому для удобства фиксации и сравнения параметров они часто приводятся к \href{https://ru.wikipedia.org/wiki/%D0%A1%D1%82%D0%B0%D0%BD%D0%B4%D0%B0%D1%80%D1%82%D0%BD%D1%8B%D0%B5_%D1%83%D1%81%D0%BB%D0%BE%D0%B2%D0%B8%D1%8F}{стандартным или нормальным условиям} - определённым давлениям и температуре. 
	
Принятые в разных дисциплинах и разных организациях точные значения давления и температуры в стандартных условиях могут различаться (смотри например \url{https://en.wikipedia.org/wiki/Standard_conditions_for_temperature_and_pressure}), поэтому указание значений физических величин без уточнения условий, в которых они приводятся, может приводить к ошибкам. Наряду с термином «стандартные условия» применяется термин «нормальные условия». «Нормальные условия» обычно отличаются от «стандартных» тем, что под нормальным давлением принимается давление равное 101 325 Па = 1 атм = 760 мм рт. ст.

Обычно в монографиях SPE принято, что стандартное давление для газов, жидкостей и твёрдых тел, равное $10^5$ Па (100 кПа, 1 бар); стандартная температура для газов, равная 15.6 °С соответствующая 60 °F. 

В Российском ГОСТ 2939-63  принято, что стандартное давление для газов, жидкостей и твёрдых тел, равное $10.13^5$ Па (101325 Па, 1 атм); стандартная температура для газов, равная 20 °С соответствующая 68 °F. 

В \unf приняты следующие значения стандартных условий
%\putlisting{listings/standard_cond.lst}



\begin{listing}[H]
	\begin{minted}[fontsize=\small]{vb.net}
Public Const const_Psc_atma As Double = 1
Public Const const_Tsc_C = 20
Public Const const_convert_atma_Pa = 101325
	\end{minted}
	\caption{Принятые параметры стандартных условий в расчетах}
	\label{lst:code_standard_cond}
\end{listing}

\subsection{PVT\_Pb\_atma – давление насыщения}
Функция рассчитывает давление насыщения по известным данным газосодержания при давлении насыщения, $\gamma_g, \gamma_o, T_r$.

При проведении расчётов с использованием значения давления насыщения, следует помнить, что давление насыщения является функцией температуры. В частности при калибровки результатов расчётов на известное значение давления насыщения $P_b$ следует указывать значение пластовой температуры $T_r$ при котором давление насыщения было получено. 

В наборе корреляций на основе корреляции Стендинга расчет давления насыщения проводится по корреляции Стендинга \cite{Yukos_PVT_2002}

\putlisting{listings/PVT_Pb_atma.lst}

Пример расчёта с использованием функции \mintinline{vb.net}{PVT_Pb_atma} для различных наборов PVT корреляций приведён на рисунке ниже. Видно, что результаты расчетов по различным корреляциях дают качественно схожие результаты, но не совпадают друг с другом.  Отличия, по всей видимости,  обусловленные применением различных наборов исходных данных использовавшихся авторами. Поэтому при проведении расчетов для конкретного месторождения актуальной является задача выбора адекватного набора корреляций. Макросы \unf позволяют провести расчет с использованием различных подходов, но при этом выбор корреляции остается за пользователем. 

\begin{tikzpicture}[scale=0.8]
\begin{axis}[
xlabel=$R_{sb} \;  m^3/m^3$,
ylabel=$P_b\; atma$,
legend pos=north west,
title=Standing]
\addplot table [y=T20, x=Rs]{data/Pb_T_data.txt};
\addlegendentry{$T = 20$ С}
\addplot table [y=T60, x=Rs]{data/Pb_T_data.txt};
\addlegendentry{$T = 60$ С}
\addplot table [y=T100, x=Rs]{data/Pb_T_data.txt};
\addlegendentry{$T = 100$ С}
\addplot table [y=T140, x=Rs]{data/Pb_T_data.txt};
\addlegendentry{$T = 140$ С}
\end{axis}
\end{tikzpicture}
\begin{tikzpicture}[scale=0.8]
\begin{axis}[
xlabel=$R_{sb} \;  m^3/m^3$,
ylabel=$P_b\; atma$,
legend pos=north west,
title = McCain]
\addplot table [y=T20, x=Rs]{data/Pb_T_data1.txt};
\addlegendentry{$T = 20$ С}
\addplot table [y=T60, x=Rs]{data/Pb_T_data1.txt};
\addlegendentry{$T = 60$ С}
\addplot table [y=T100, x=Rs]{data/Pb_T_data1.txt};
\addlegendentry{$T = 100$ С}
\addplot table [y=T140, x=Rs]{data/Pb_T_data1.txt};
\addlegendentry{$T = 140$ С}
\end{axis}
\end{tikzpicture}


При проведении расчётов с использованием набора корреляций на основе корреляций МакКейна следует учитывать, что они работают только для температур более 18 градусов Цельсия. При более низких значениях температуры расчёт будет проводиться для 18 градусов Цельсия. 

\subsection{PVT\_Rs\_m3m3 – газосодержание}

Газосодержание это отношения объёма газа растворенного в нефти к объёму нефти приведённые к стандартным условиям. 

$$R_s = \frac{(V_g)_{sc}}{(V_o)_{sc}}$$

Газосодержание является одним из ключевых свойств нефти при расчётах производительности скважин и работы скважинного оборудования. Динамика изменения газосодержания во многом определяет количество свободного газа в потоке и должна учитываться при проведении расчётов. 

При задании PVT свойств нефти часто используют значение газосодержания при давлении насыщения $r_{sb}$ - определяющее объем газа растворенного в нефти в пластовых условиях. В модели флюида \unf газосодержание при давлении насыщения является исходным параметров нефти и должно быть обязательно задано. 

Следует отличать газосодержание в нефти при давлении насыщения $R_sb$ и газовый фактор $R_p$.

$$R_p = \frac{(Q_g)_{sc}}{(Q_o)_{sc}}$$

Газовый фактор $R_{p}$  в отличии от газосодержания $R_{sb}$  является, вообще говоря, параметром скважины - показывает отношение объёма добытого газа из скважины к объёму добытой нефти приведённые к стандартным условиям. Газосодержание же является свойством нефти - показывает сколько газа растворено в нефти. Если газ добываемой из скважины это газ который выделился из нефти в процессе подъёма, что характерно для недонасыщенных нефтей, то значения газового фактора и газосодержания будут совпадать. Если газ поступает в скважину не непосредственно из добываемой нефти, а например фильтруется из газовой шапки или поступает через негерметичность ствола скважины - то в такой скважине газовый фактор может значительно превышать значение газосодержания. Такая ситуация может быть смоделирована в \unf. Для этого необходимо наряду с газосодержанием при давлении насыщения $R_{sb}$ задать значение газового фактора $R_p$. В этом случае газосодержание при давлении насыщения $R_{sb}$  будет определять динамику выделения попутного газа из нефти при снижении давления, а газовый фактор $R_p$ определять общее количество газа в потоке. 

При определённых условиях газовый фактор может быть меньше газосодержания. Это происходит, когда газ выделяется в призабойной зоне и скапливается в ней не поступая в скважину вместе с нефтью. Но такие условия возникают достаточно редко, существуют на скважине ограниченное время и представляют интерес больше для разработчиков нежели чем для технологов. С точки зрения анализа работы скважины и скважинного оборудования можно считать, что значение газового фактора не может быть меньше газосодержания при давлении насыщения. Такой предположение реализовано в \unf. При этом значение газового фактора технически легче измерить чем газосодержание - поэтому при противоречии значений газового фактора и газосодержания при давлении насыщения приоритет отдаётеся газовому фактору. 


\putlisting{listings/PVT_Rs_m3m3.lst}

Примеры расчёта с использованием функции \mintinline{vb.net}{PVT_Rs_m3m3} для различных наборов PVT корреляций приведён на рисунке ниже.

\newcommand{\RsDataFile}{data/Rs_P_data.txt}
\begin{tikzpicture}[scale=0.8]
\begin{axis}[
xlabel=$R_{sb} \;  m^3/m^3$,
ylabel=$P_b\; atma$,
legend pos=south east,
title=Standing]
\addplot table [y=T_0_20, x=P]{\RsDataFile};
\addlegendentry{$T = 20$ С}
\addplot table [y=T_0_60, x=P]{\RsDataFile};
\addlegendentry{$T = 60$ С}
\addplot table [y=T_0_100, x=P]{\RsDataFile};
\addlegendentry{$T = 100$ С}
\addplot table [y=T_0_140, x=P]{\RsDataFile};
\addlegendentry{$T = 140$ С}
\end{axis}
\end{tikzpicture}
\begin{tikzpicture}[scale=0.8]
\begin{axis}[
xlabel=$R_{sb} \;  m^3/m^3$,
ylabel=$P_b\; atma$,
legend pos=south east,
title = McCain]
\addplot table [y=T_1_20, x=P]{\RsDataFile};
\addlegendentry{$T = 20$ С}
\addplot table [y=T_1_60, x=P]{\RsDataFile};
\addlegendentry{$T = 60$ С}
\addplot table [y=T_1_100, x=P]{\RsDataFile};
\addlegendentry{$T = 100$ С}
\addplot table [y=T_1_140, x=P]{\RsDataFile};
\addlegendentry{$T = 140$ С}
\end{axis}
\end{tikzpicture}


\subsection{PVT\_Bo\_m3m3 – объёмный коэффициент нефти}

Функция рассчитывает объёмный коэффициент нефти для произвольных термобарических условий. 
Объёмный коэффициент нефти определяется как отношение объёма занимаемого нефтью в пластовых условиях к объёму занимаемому нефтью при стандартных условиях. 

$$B_o = \frac{(V_o)_{rc}}{(V_o)_{sc}}$$

Нефть в пласте занимает больший объем чем на поверхности за счёт растворенного в ней газа. Соответственно объёмный коэффициент нефти обычно имеет значение больше 1 при давлениях больше чем стандартное.

Для калибровки значения объёмного коэффициента можно использовать значение объёмного коэффициента нефти при давлении насыщения $B_{ob}$. 

Следует отметить, что вообще говоря значение объёмного коэффициента нефти при давлении насыщения не является значением при пластовых условиях (при давлении выше давления насыщения играет роль сжимаемость нефти), однако при анализе производительности скважины и скважинного оборудования можно условно считать, что значение объёмного коэффициента при давлении насыщения соответствует значению  объёмного коэффициента в пластовых условиях.  

\putlisting{listings/PVT_Bo_m3m3.lst}

Примеры расчёта с использованием функции \mintinline{vb.net}{PVT_Bo_m3m3} для различных наборов PVT корреляций приведён на рисунке ниже.

Объёмный коэффициент нефти хорошо коррелирует со значением газосодержания. Поэтому различный вид кривых на рисунке ниже связан с первую очередь с различным газосодержанием при проведении расчётов.

\newcommand{\BoDataFile}{data/Bo_P_data.txt}
\begin{tikzpicture}[scale=0.8]
\begin{axis}[
xlabel=$P\; atma$,
ylabel=$B_o\;  m^3/m^3$,
legend pos=south east,
title=Standing]
\addplot table [y=T_0_20, x=P]{\BoDataFile};
\addlegendentry{$T = 20$ С}
\addplot table [y=T_0_60, x=P]{\BoDataFile};
\addlegendentry{$T = 60$ С}
\addplot table [y=T_0_100, x=P]{\BoDataFile};
\addlegendentry{$T = 100$ С}
\addplot table [y=T_0_140, x=P]{\BoDataFile};
\addlegendentry{$T = 140$ С}
\end{axis}
\end{tikzpicture}
\begin{tikzpicture}[scale=0.8]
\begin{axis}[
xlabel=$P\; atma$,
ylabel=$B_o\;  m^3/m^3$,
legend pos=south east,
title = McCain]
\addplot table [y=T_1_20, x=P]{\BoDataFile};
\addlegendentry{$T = 20$ С}
\addplot table [y=T_1_60, x=P]{\BoDataFile};
\addlegendentry{$T = 60$ С}
\addplot table [y=T_1_100, x=P]{\BoDataFile};
\addlegendentry{$T = 100$ С}
\addplot table [y=T_1_140, x=P]{\BoDataFile};
\addlegendentry{$T = 140$ С}
\end{axis}
\end{tikzpicture}

\subsection{PVT\_Bg\_m3m3 – объёмный коэффициент газа}
Функция рассчитывает объёмный коэффициент нефтяного газа для произвольных термобарических условий. 

Объёмный коэффициент газа определяется как отношение объема занимаемого газом для произвольных термобарических условий (при определенном давлении и температуре) к объёму занимаемому газом при стандартных условиях. 

$$B_g = \frac{V_g(P,T)}{(V_g)_{sc}}$$

Значение объемного коэффиента газа может быть определено исходя из уравнения состояния газа

$$ PV = z \nu RT  $$

откуда можно получить 

$$ B_g = z \frac{P_{sc}}{P} \frac{T}{T_{sc}} $$

где $P_{sc}, T_{sc}$ давление (атм) и температура (К) при стандартных условиях, $P,T$ давление (атм) и температура (K) при расчетных условиях, $z$ коэффициент сверхсжимаемости газа, который вообще говоря зависит от давления и температуры $z = z(P,T)$. 

\putlisting{listings/PVT_Bg_m3m3.lst}

\newcommand{\DataFile}{data/Bg_P_data.txt}
\begin{tikzpicture}[scale=0.8]
\begin{axis}[
ymode=log, 
xlabel=$P\; atma$,
ylabel=$B_g\\;  m^3/m^3$,
legend pos=north east,
title=Standing]
\addplot table [y=T_0_20, x=P]{\DataFile};
\addlegendentry{$T = 20$ С}
\addplot table [y=T_0_60, x=P]{\DataFile};
\addlegendentry{$T = 60$ С}
\addplot table [y=T_0_100, x=P]{\DataFile};
\addlegendentry{$T = 100$ С}
\addplot table [y=T_0_140, x=P]{\DataFile};
\addlegendentry{$T = 140$ С}
\end{axis}
\end{tikzpicture}
\begin{tikzpicture}[scale=0.8]
\begin{axis}[
ymode=log, 
xlabel=$P\; atma$,
ylabel=$B_g\;  m^3/m^3$,
legend pos=north east,
title = McCain]
\addplot table [y=T_1_20, x=P]{\DataFile};
\addlegendentry{$T = 20$ С}
\addplot table [y=T_1_60, x=P]{\DataFile};
\addlegendentry{$T = 60$ С}
\addplot table [y=T_1_100, x=P]{\DataFile};
\addlegendentry{$T = 100$ С}
\addplot table [y=T_1_140, x=P]{\DataFile};
\addlegendentry{$T = 140$ С}
\end{axis}
\end{tikzpicture}

\subsection{PVT\_Bw\_m3m3 – объёмный коэффициент воды}
Функция рассчитывает объёмный коэффициент воды для произвольных термобарических условий. 

Объёмный коэффициент воды определяется как отношение объёма занимаемого водой для произвольных термобарических условий (при определённом давлении и температуре) к объёму занимаемому водой при стандартных условиях. 

$$B_w = \frac{V_w(P,T)}{(V_w)_{sc}}$$

\putlisting{listings/PVT_Bw_m3m3.lst}

\subsection{PVT\_Muo\_cP – вязкость нефти}
Функция рассчитывает вязкость нефти при заданных термобарических условиях по корреляции. Расчёт может быть откалиброван на известное значение вязкости нефти при давлении равном давлению насыщения и при пластовой температуре за счёт задания калибровочного параметра \mintinline{vb.net}{Muob_cP}. При калибровке динамика изменения будет соответствовать расчету по корреляции, но значения будут масштабированы таким образом, чтобы при давлении насыщения удовлетворить калибровочному параметру.

При расчёте следует обратить внимание, что значение вязкости коррелирует со значением плотности нефти. Как правило вязкость тяжёлых нефтей выше чем для легких.

При расчёте с использованием набора корреляций на основе корреляции Стендинга - вязкость как дегазированной нефти и нефти с учетом растворенного газа рассчитывается по корреляции Беггса Робинсона \cite{Yukos_PVT_2002}. 
Корреляции для расчета вязкости разгазированной и газонасыщенной нефти, разработанные Beggs \& Robinson, основаны на 2000 замерах 600 различных нефтей.
Диапазоны значений основных свойств, использованных для разработки данной корреляции, приведены в таблице ниже.
\begin{center}
	\begin{tabular}{ccc}
		давление, atma & \textbf{8.96…483.} \\
		температура, °C & \textbf{37…127}  \\
		газосодержание, $R_s \; m^3 /m^3$ & \textbf{3.6…254}\\
		относительная плотность нефти по воде,, $\gamma_o$ & \textbf{0.725…0.956} \\
	\end{tabular}
\end{center}
   
\putlisting{listings/PVT_Muo_cP.lst}

\newcommand{\MuDataFile}{data/Muo_P_data.txt}
\begin{tikzpicture}[scale=0.8]
\begin{axis}[
xlabel=$P\; atma$,
ylabel=$\mu_o\; cP$,
legend pos=north east,
title=Standing]
\addplot table [y=T_0_20, x=P]{\MuDataFile};
\addlegendentry{$T = 20$ С}
\addplot table [y=T_0_60, x=P]{\MuDataFile};
\addlegendentry{$T = 60$ С}
\addplot table [y=T_0_100, x=P]{\MuDataFile};
\addlegendentry{$T = 100$ С}
\addplot table [y=T_0_140, x=P]{\MuDataFile};
\addlegendentry{$T = 140$ С}
\end{axis}
\end{tikzpicture}
\begin{tikzpicture}[scale=0.8]
\begin{axis}[
xlabel=$P\; atma$,
ylabel=$\mu_o\; cP$,
legend pos=north east,
title = McCain]
\addplot table [y=T_1_20, x=P]{\MuDataFile};
\addlegendentry{$T = 20$ С}
\addplot table [y=T_1_60, x=P]{\MuDataFile};
\addlegendentry{$T = 60$ С}
\addplot table [y=T_1_100, x=P]{\MuDataFile};
\addlegendentry{$T = 100$ С}
\addplot table [y=T_1_140, x=P]{\MuDataFile};
\addlegendentry{$T = 140$ С}
\end{axis}
\end{tikzpicture}

\subsection{PVT\_Mug\_cP – вязкость газа}

Функция рассчитывает вязкость газа при заданных термобарических условиях. Результат расчета в сП.  Используется подход предложенный Lee  \cite{Lee_1966}, который хорошо подходит для большинства натуральных газов. 
В отличии от нефти и других жидкостей вязкость газа, как правило, значительно ниже, что определяет высокую подвижность газа. 
Более подробное описание методов расчета вязкости газа можно найти на странице  \href{http://petrowiki.org/Gas_viscosity}{http://petrowiki.org/gas\_viscosity}


\putlisting{listings/PVT_Mug_cP.lst}

\newcommand{\MugDataFile}{data/Mug_P_data.txt}
\begin{tikzpicture}[scale=0.8]
\begin{axis}[
xlabel=$P\; atma$,
ylabel=$\mu_g\; cP$,
legend pos=north west,
title=Standing]
\addplot table [y=T_0_20, x=P]{\MugDataFile};
\addlegendentry{$T = 20$ С}
\addplot table [y=T_0_60, x=P]{\MugDataFile};
\addlegendentry{$T = 60$ С}
\addplot table [y=T_0_100, x=P]{\MugDataFile};
\addlegendentry{$T = 100$ С}
\addplot table [y=T_0_140, x=P]{\MugDataFile};
\addlegendentry{$T = 140$ С}
\end{axis}
\end{tikzpicture}
\begin{tikzpicture}[scale=0.8]
\begin{axis}[
xlabel=$P\; atma$,
ylabel=$\mu_g\; cP$,
legend pos=north west,
title = McCain]
\addplot table [y=T_1_20, x=P]{\MugDataFile};
\addlegendentry{$T = 20$ С}
\addplot table [y=T_1_60, x=P]{\MugDataFile};
\addlegendentry{$T = 60$ С}
\addplot table [y=T_1_100, x=P]{\MugDataFile};
\addlegendentry{$T = 100$ С}
\addplot table [y=T_1_140, x=P]{\MugDataFile};
\addlegendentry{$T = 140$ С}
\end{axis}
\end{tikzpicture}

\subsection{PVT\_Muw\_cP – вязкость воды}

Функция рассчитывает вязкость воды при заданных термобарических условиях. Результат расчета выдается в сП. 
Вязкость воды зависит от давления, температуры и наличия растворенных примесей. В общем вязкость аоды растет при росте давления, снижении температуры, повышении солености. 
Растворение газа почти не влияет на вязкость воды и в расчетах не учитывается. 
Расчет проводится по корреляции McCain \cite{McCain_1991}

Более подробное описание методов расчета вязкости газа можно найти на странице  \href{http://petrowiki.org/Produced_water_properties}{http://petrowiki.org/Produced\_water\_properties}


\putlisting{listings/PVT_Muw_cP.lst}


Следует отметить, что вязкость воды достаточно сильно зависит от температуры, в то время как зависимость от давления менее выражена.

\newcommand{\MuwDataFile}{data/Muw_P_data.txt}
\begin{tikzpicture}[scale=0.8]
\begin{axis}[
xlabel=$P\; atma$,
ylabel=$\mu_w\; cP$,
legend pos=north west,
title=Standing]
\addplot table [y=T_0_20, x=P]{\MuwDataFile};
\addlegendentry{$T = 20$ С}
\addplot table [y=T_0_60, x=P]{\MuwDataFile};
\addlegendentry{$T = 60$ С}
\addplot table [y=T_0_100, x=P]{\MuwDataFile};
\addlegendentry{$T = 100$ С}
\addplot table [y=T_0_140, x=P]{\MuwDataFile};
\addlegendentry{$T = 140$ С}
\end{axis}
\end{tikzpicture}
\begin{tikzpicture}[scale=0.8]
\begin{axis}[
xlabel=$P\; atma$,
ylabel=$\mu_w\; cP$,
legend pos=north west,
title = McCain]
\addplot table [y=T_1_20, x=P]{\MuwDataFile};
\addlegendentry{$T = 20$ С}
\addplot table [y=T_1_60, x=P]{\MuwDataFile};
\addlegendentry{$T = 60$ С}
\addplot table [y=T_1_100, x=P]{\MuwDataFile};
\addlegendentry{$T = 100$ С}
\addplot table [y=T_1_140, x=P]{\MuwDataFile};
\addlegendentry{$T = 140$ С}
\end{axis}
\end{tikzpicture}

\subsection{PVT\_Rhoo\_kgm3 – плотность нефти}
Функция вычисляет значение плотности нефти при заданных термобарических условиях. Результат расчета имеет размерность кг/м3. 


\putlisting{listings/PVT_Rhoo_kgm3.lst}

\newcommand{\RhooDataFile}{data/Rhoo_P_data.txt}
\begin{tikzpicture}[scale=0.8]
\begin{axis}[
xlabel=$P\; atma$,
ylabel=$\rho_o\; kg/m^3$,
legend pos=north east,
title=Standing]
\addplot table [y=T_0_20, x=P]{\RhooDataFile};
\addlegendentry{$T = 20$ С}
\addplot table [y=T_0_60, x=P]{\RhooDataFile};
\addlegendentry{$T = 60$ С}
\addplot table [y=T_0_100, x=P]{\RhooDataFile};
\addlegendentry{$T = 100$ С}
\addplot table [y=T_0_140, x=P]{\RhooDataFile};
\addlegendentry{$T = 140$ С}
\end{axis}
\end{tikzpicture}
\begin{tikzpicture}[scale=0.8]
\begin{axis}[
xlabel=$P\; atma$,
ylabel=$\rho_o \; kg/m^3$,
legend pos=north east,
title = McCain]
\addplot table [y=T_1_20, x=P]{\RhooDataFile};
\addlegendentry{$T = 20$ С}
\addplot table [y=T_1_60, x=P]{\RhooDataFile};
\addlegendentry{$T = 60$ С}
\addplot table [y=T_1_100, x=P]{\RhooDataFile};
\addlegendentry{$T = 100$ С}
\addplot table [y=T_1_140, x=P]{\RhooDataFile};
\addlegendentry{$T = 140$ С}
\end{axis}
\end{tikzpicture}



\subsection{PVT\_Rhog\_kgm3 – плотность газа}
\putlisting{listings/PVT_Rhog_kgm3.lst}
	
\newcommand{\RhogDataFile}{data/Rhog_P_data.txt}
\begin{tikzpicture}[scale=0.8]
\begin{axis}[
xlabel=$P\; atma$,
ylabel=$\rho_g\; kg/m^3$,
legend pos=north west,
title=Standing]
\addplot table [y=T_0_20, x=P]{\RhogDataFile};
\addlegendentry{$T = 20$ С}
\addplot table [y=T_0_60, x=P]{\RhogDataFile};
\addlegendentry{$T = 60$ С}
\addplot table [y=T_0_100, x=P]{\RhogDataFile};
\addlegendentry{$T = 100$ С}
\addplot table [y=T_0_140, x=P]{\RhogDataFile};
\addlegendentry{$T = 140$ С}
\end{axis}
\end{tikzpicture}
\begin{tikzpicture}[scale=0.8]
\begin{axis}[
xlabel=$P\; atma$,
ylabel=$\rho_g \; kg/m^3$,
legend pos=north west,
title = McCain]
\addplot table [y=T_1_20, x=P]{\RhogDataFile};
\addlegendentry{$T = 20$ С}
\addplot table [y=T_1_60, x=P]{\RhogDataFile};
\addlegendentry{$T = 60$ С}
\addplot table [y=T_1_100, x=P]{\RhogDataFile};
\addlegendentry{$T = 100$ С}
\addplot table [y=T_1_140, x=P]{\RhogDataFile};
\addlegendentry{$T = 140$ С}
\end{axis}
\end{tikzpicture}

\subsection{PVT\_Rhow\_kgm3 – плотность воды}
\putlisting{listings/PVT_Rhow_kgm3.lst}

\newcommand{\RhowDataFile}{data/Rhow_P_data.txt}
\begin{tikzpicture}[scale=0.8]
\begin{axis}[
xlabel=$P\; atma$,
ylabel=$\rho_w\; kg/m^3$,
legend pos=north east,
title=Standing]
\addplot table [y=T_0_20, x=P]{\RhowDataFile};
\addlegendentry{$T = 20$ С}
\addplot table [y=T_0_60, x=P]{\RhowDataFile};
\addlegendentry{$T = 60$ С}
\addplot table [y=T_0_100, x=P]{\RhowDataFile};
\addlegendentry{$T = 100$ С}
\addplot table [y=T_0_140, x=P]{\RhowDataFile};
\addlegendentry{$T = 140$ С}
\end{axis}
\end{tikzpicture}
\begin{tikzpicture}[scale=0.8]
\begin{axis}[
xlabel=$P\; atma$,
ylabel=$\rho_w \; kg/m^3$,
legend pos=north east,
title = McCain]
\addplot table [y=T_1_20, x=P]{\RhowDataFile};
\addlegendentry{$T = 20$ С}
\addplot table [y=T_1_60, x=P]{\RhowDataFile};
\addlegendentry{$T = 60$ С}
\addplot table [y=T_1_100, x=P]{\RhowDataFile};
\addlegendentry{$T = 100$ С}
\addplot table [y=T_1_140, x=P]{\RhowDataFile};
\addlegendentry{$T = 140$ С}
\end{axis}
\end{tikzpicture}

\subsection{PVT\_Z – коэффициент сверхсжимаемости газа}

Функция позволяет рассчитать коэффициент сверхсжимаемости газа. 


$$ PV = z \nu RT  $$



\putlisting{listings/PVT_Z.lst}

\newcommand{\zDataFile}{data/Z_P_data.txt}
\begin{tikzpicture}[scale=0.8]
\begin{axis}[
xlabel=$P\; atma$,
ylabel=$Z$,
legend pos=south west,
title=Standing]
\addplot table [y=T_0_20, x=P]{\zDataFile};
\addlegendentry{$T = 20$ С}
\addplot table [y=T_0_60, x=P]{\zDataFile};
\addlegendentry{$T = 60$ С}
\addplot table [y=T_0_100, x=P]{\zDataFile};
\addlegendentry{$T = 100$ С}
\addplot table [y=T_0_140, x=P]{\zDataFile};
\addlegendentry{$T = 140$ С}
\end{axis}
\end{tikzpicture}
\begin{tikzpicture}[scale=0.8]
\begin{axis}[
xlabel=$P\; atma$,
ylabel=$Z$,
legend pos=south west,
title = McCain]
\addplot table [y=T_1_20, x=P]{\zDataFile};
\addlegendentry{$T = 20$ С}
\addplot table [y=T_1_60, x=P]{\zDataFile};
\addlegendentry{$T = 60$ С}
\addplot table [y=T_1_100, x=P]{\zDataFile};
\addlegendentry{$T = 100$ С}
\addplot table [y=T_1_140, x=P]{\zDataFile};
\addlegendentry{$T = 140$ С}
\end{axis}
\end{tikzpicture}

\section{Расчёт свойств потока}

\subsection{MF\_Qmix\_m3day – расход газожидкостной смеси}

Функция позволяет рассчитать объемный расход газожидкостной смеси при заданных термобарических условиях. 

$$Q_{mix} = Q_w B_w(P,T) + Q_o B_o(P,T)  + Q_o  (R_p - R_s(P,T)) B_g(P,T) $$

\putlisting{listings/MF_Qmix_m3day.lst}

\subsection{MF\_Rhomix\_kgm3 – плотность газожидкостной смеси}

Функция позволяет рассчитать плотность газожидкостной смеси при заданных термобарических условиях. 

\putlisting{listings/MF_Rhomix_kgm3.lst}

\subsection{MF\_GasFraction\_d – доля газа в потоке}
Функция расчёта доли свободного газа в потоке (без учёта проскальзывания) в зависимости от термобарических условий для заданного флюида. 
В отличии от функций PVT учитывается обводнённость.
\putlisting{listings/MF_GasFraction_d.lst}

\subsection{MF\_PGasFraction\_atma – целевое давления для заданной доли газа в потоке}
Функция расчёта давления при котором достигается заданная доля свободного газа в потоке (без учёта проскальзывания) . 
В отличии от функций PVT учитывается обводнённость.
Следует учитывать, что при вызове функции пересчитывается состояние смеси с различными термобарическими условиями.
\putlisting{listings/MF_PGasFraction_atma.lst}

\subsection{MF\_RpGasFraction\_m3m3 – целевой газовый фактор для заданной доли газа в потоке}
Функция расчёта давления при котором достигается заданная доля свободного газа в потоке (без учёта проскальзывания) . 
В отличии от функций PVT учитывается обводнённость.
Следует учитывать, что при вызове функции пересчитывается состояние смеси с различными термобарическими условиями.
\putlisting{listings/MF_RpGasFraction_m3m3.lst}

\section{Сепарация газа в скважине}

В скважинах оборудованных системами механизированной добычи нефти важную роль играет процесс сепарации газа на приёме насоса. Под сепарацией газа понимается отделение части свободного газа из потока и перенаправление его по отдельному гидравлическому каналу на поверхность. В результате сепарации газа меняются свойства флюида поступающего в насос и НКТ выше насоса. Оценка величины сепарации может быть проведена приведёнными ниже функциями.

\subsection{MF\_SeparNat\_d – естественная сепарация газа}

Функция рассчитывает естественную сепарацию газа на приёме насоса в скважине с использованием корреляции Маркеса \cite{Marquez_2003} . Результат - безразмерная величина в диапазоне от 0 до 1. 

\putlisting{listings/MF_SeparNat_d.lst}

\subsection{MF\_SeparTotal\_d – естественная сепарация газа}

Функция рассчитывает полную сепарацию газа на приёме насосе в скважине по известным значениям естественной сепарации газа и коэффициента сепарации газосепаратора. Результат - безразмерная величина в диапазоне от 0 до 1. 

\putlisting{listings/MF_SeparTotal_d.lst}


\section{Расчёт многофазного потока в штуцере}

\subsection{Модель потока через штуцер}

%Тут надо будет нарисовать схему штуцера и пояснить что и как называется в коде. 

Штуцер или локальное гидравлическое сопротивление - элемент скважины или системы трубопроводов применяемых для создания дополнительного перепада давления в системе и ограничения потока. 
Возможны различные варианты реализации штуцера - со штуцерное камерой, с угловым краном позволяющим менять диаметр штуцера и другие.
Ключевым параметром штуцера является диаметр \(d_{choke} \) определяющий его способность к ограничению потока. 

Как и у любого элемента гидравлического потока есть три ключевых параметра - давление на входе \( P_{in} \) , \( P_0\), давление на выходе \(P_{out}\) , \( P_1\) и расход газожидкостной смеси, обычно задаваемый в стандартных условиях \(Q_{liq} \).
 

\subsection{MF\_PChoke\_atm – Расчет давления на входе и на выходе штуцера}
Функция позволяет рассчитать давление на входе или выходе штуцера по известному давлению на противоположном конце при известных параметрах потока (дебите жидкости, обводненности, газовому фактору). Расчет проводится по корреляции Перкинса \cite{Perkins_1993} с учетом многофазного потока.  
\putlisting{listings/MF_Pchoke_atm.lst}

\subsection{MF\_dPChoke\_atm – Расчёт перепада давления в штуцере}
Функция позволяет рассчитать по известному линейному давлению и дебиту или по известному буферному давлению и дебиту перепад давления.  Расчет проводится по корреляции Перкинса \cite{Perkins_1993} с учетом многофазного потока.  
Функция возвращает перепад давления и температуры в виде массива.
\putlisting{listings/MF_dPchoke_atm.lst}



\newpage
\subsection{MF\_QChoke\_m3day – функция расчёта дебита жидкости через штуцер}
Функция позволяет рассчитать по известному буферному давлению и линейному давлению дебит жидкости. Расчет проводится по корреляции Перкинса \cite{Perkins_1993} с учетом многофазного потока.  

\putlisting{listings/MF_QChoke_m3day.lst}

\newpage
\section{Расчет многофазного потока в трубе}



\subsection{MF\_dPpipe\_atma – расчёт перепада давления в трубе}

Функция позволяет рассчитать перепад давления в участке трубопровода. 

Функция возвращает давление и температуру в виде массива.

\putlisting{listings/MF_dPpipe_atm.lst}

Ниже на рисунке приведены результаты расчёта кривой оттока (перепада давления в вертикальной трубе) для различных корреляций реализованных в \unf.

\newcommand{\dPipeDataFile}{data/dPipe.txt}
\begin{tikzpicture}[scale=1]
\begin{axis}[
width=14cm,
height=10cm,
xlabel=$Q\; m^3 / day$,
ylabel=$P_{wf} \; atma$,
legend pos=south east,
title=Pipe Pressure Drop]
\addplot table [y=P_0, x=Q]{\dPipeDataFile};
\addlegendentry{Beggs Brill}
\addplot table [y=P_1, x=Q]{\dPipeDataFile};
\addlegendentry{Ansari}
\addplot table [y=P_2, x=Q]{\dPipeDataFile};
\addlegendentry{Unified}
\addplot table [y=P_3, x=Q]{\dPipeDataFile};
\addlegendentry{Gray}
\addplot table [y=P_4, x=Q]{\dPipeDataFile};
\addlegendentry{Hagedorn Brown}
\addplot table [y=P_5, x=Q]{\dPipeDataFile};
\addlegendentry{Sakharov Mokhov}
\end{axis}
\end{tikzpicture}



\newpage
\section{Расчет многофазного потока в пласте}
\newpage
\subsection{IPR\_PI\_sm3dayatm – расчёт продуктивности}
Функция позволяет рассчитать коэффициент продуктивности скважины.
\begin{listing}[H]
	\begin{minted}{vb.net}
' расчёт продуктивности
Public Function IPR_PI_sm3dayatm(Qtest_m3day, Pwftest_atm, Pr_atm, _
Optional WCT_perc As Double = 0, Optional Pb_atm As Double = -1)
	'
	' Qtest_m3day   - тестовый дебит скважины
	' Pwftest_atm   - тестовое забойное давление
	' Pr_atm        - пластовое давление, атм
	'
	' необязательные параметры
	' WCT_perc      - обводненность
	' Pb_atm        - давление насыщения
	'
	\end{minted}
	\caption{Объявление функции расчёта продуктивности}
	\label{lst:codedIPR_PI}
\end{listing}
\newpage
\subsection{IPR\_Pwf\_atm – расчёт дебита по давлению и продуктивности}
Функция позволяет рассчитать дебит жидкости скважины по известным значениям давления и продуктивности.

\begin{listing}[H]
	\begin{minted}{vb.net}
' расчёт дебита по давлению и продуктивности
Public Function IPR_Pwf_atm(PI_m3dayatm, Pr_atm, Ql_m3day, _
Optional WCT_perc As Double = 0, Optional Pb_atm As Double = -1)
'
' PI_m3dayatm   - коэффициент продуктивности
' Pr_atm        - пластовое давление, атм
' Ql_m3day      - дебит жидкости скважины на поверхности
'
' необязательные параметры
' WCT_perc      - обводненность
' Pb_atm        - давление насыщения
'
	\end{minted}
	\caption{Объявление функции расчёта дебита по давлению и продуктивности}
	\label{lst:codedIPR_Pwf}
\end{listing}
\newpage
\subsection{IPR\_Ql\_sm3Day – расчёт дебита по давлению и продуктивности}
Функция позволяет рассчитать дебита по давлению и продуктивности.
\begin{listing}[H]
	\begin{minted}{vb.net}
' расчёт дебита по давлению и продуктивности
Public Function IPR_Ql_sm3Day(PI_m3dayatm, Pr_atm, Pwf_atm, _
Optional WCT_perc As Double = 0, Optional Pb_atm As Double = -1)
'
' PI_m3dayatm   - коэффициент продуктивности
' Pr_atm        - пластовое давление, атм
' Pwf_atm       - забойное давление
'
' необязательные параметры
' WCT_perc      - обводненность
' Pb_atm        - давление насыщения
'

	\end{minted}
	\caption{Объявление функции расчёта дебита по давлению и продуктивности}
	\label{lst:codedIPR_Ql}
\end{listing}


\newpage
           % Глава 2 PVT 
	\section{Расчёт свойств потока}

В отличии от функций расчета PVT (физико-химических свойств флюидов) функции расчета свойства потока учитывают дополнительные параметры потока флюидов - $Q$ - дебит, объемный расход флюидов, $f_w$ - обводненность, $R_p$ - газовый фактор. В функциях свойств потока используется префикc \mintinline{vb.net}{MF_}.

Параметры потока, такие как расход ГЖС, доля газа в потоке, вязкость ГЖС важны для расчета и анализа работы скважин и скважинного оборудования.


\subsection{MF\_qmix\_m3day – расход газожидкостной смеси}

Функция позволяет рассчитать объемный расход газожидкостной смеси при заданных термобарических условиях. Объемный расход ГЖС важен например для подбора УЭЦН в скважине, так как именно определяет в какой точке характеристики УЭЦН будет работать. При наличии свободного газа в потоке расход ГЖС может быть значительно больше расхода жидкости на поверхности фиксируемого расходомером.
$$Q_{mix,rc} = Q_{w,sc} B_w(P,T) + Q_{o,sc} B_o(P,T)  + Q_{o,sc}  (R_p - R_s(P,T)) B_g(P,T) $$

Расход ГЖС определяется как сумма расходов отдельных фаз, приведенных к соответствующим термобарическим условиям, с учетом того, что часть газа будет растворена в нефти.

\putlisting{listings/MF_q_mix_rc_m3day.lst}

\subsection{MF\_rhomix\_kgm3 – плотность газожидкостной смеси}

Функция позволяет рассчитать плотность газожидкостной смеси при заданных термобарических условиях. 
$$\rho_{mix,rc} = \left( \frac{\rho_{w,sc}}{B_w} f_w + \frac{\rho_{o,sc} +r_s \rho_{g,sc} }{B_o}(1-f_w) \right) (1-f_g) + \frac{ \rho_{g,sc} }{B_g} f_g $$

\putlisting{listings/MF_Rhomix_kgm3.lst}

\subsection{MF\_gas\_fraction\_d – доля газа в потоке}

Функция расчёта доли свободного газа в потоке (без учёта проскальзывания) в зависимости от термобарических условий для заданного флюида. 
$$f_g = \frac{Q_{g,rc}}{Q_{mix,rc}} $$
Доля газа в потоке является одним из ключевых параметров ограничивающих производительность систем механизированной добычи - ЭЦН и других насосов.

\putlisting{listings/MF_gas_fraction_d.lst}

\subsection{MF\_p\_gas\_fraction\_atma – целевое давления для заданной доли газа в потоке}
Функция расчёта давления при котором достигается заданная доля свободного газа в потоке (без учёта проскальзывания). 
Значение давления при котором достигается определённая доля газа в потоке может быть найдено из решения уравнения, определяющего долю газа. 
$$f_g = \frac{Q_{g,rc}(P,T)}{Q_{mix,rc}(P,T)} $$
Решение в \unf реализовано итеративное, методом деления отрезка пополам (дихотомия). При вызове функции пересчитывается состояние смеси с различными термобарическими условиями. Поэтому расчёт проводится относительно медленно. 

\putlisting{listings/MF_p_gas_fraction_atma.lst}

\subsection{MF\_rp\_gas\_fraction\_m3m3 – целевой газовый фактор для заданной доли газа в потоке}
Функция расчёта газового фактора $R_p$ при котором достигается заданная доля свободного газа в потоке (без учёта проскальзывания) . 
Значение давления при котором достигается определённая доля газа в потоке может быть найдено из решения уравнения, определяющего долю газа. 
$$f_g = \frac{Q_{g,rc}(P,T,R_p)}{Q_{mix,rc}(P,T,R_p)} $$
Решение в \unf реализовано итеративное, методом деления отрезка пополам (дихотомия). При вызове функции пересчитывается состояние смеси с различными термобарическими условиями. Поэтому расчёт проводится относительно медленно. 

\putlisting{listings/MF_rp_gas_fraction_m3m3.lst}

\section{Сепарация газа в скважине}
В скважинах оборудованных системами механизированной добычи нефти важную роль играет процесс сепарации газа на приёме насоса. Под сепарацией газа понимается отделение части свободного газа из потока и перенаправление его по отдельному гидравлическому каналу на поверхность. В результате сепарации газа меняются свойства флюида, поступающего в насос и НКТ выше насоса. Оценка величины сепарации может быть проведена приведёнными ниже функциями.

\subsection{MF\_ksep\_natural\_d – естественная сепарация газа}
Функция рассчитывает естественную сепарацию газа на приёме насоса в скважине с использованием корреляции Маркеса \cite{Marquez_2003} . Результат - безразмерная величина в диапазоне от 0 до 1. 

\putlisting{listings/MF_ksep_natural_d.lst}

\subsection{MF\_ksep\_gasseparator\_d – сепарация газа роторным газосепаратором}
Функция рассчитывает сепарацию газа с использованием роторного газосепаратора, являющегося обычно частью компоновки УЭЦН. Данный расчет основан на результатах испытания характеристик роторных газосепараторов, выполненных в РГУ нефти и газа имени И.М.Губкина \cite{SPE_117415_2008}. 

Следует отметить, что несмотря на хорошее соответствие промысловых исследований и расчетов с использованием корреляции для естественной и искусственной сепарации \cite{SPE_117415_2008} к результатам стендовых исследований стоит относится с осторожностью. Основой осторожности могут быть следующие соображения: характеристики различных газосепараторов достаточно сильно отличаются друг от друга - есть удачные конструкции и не очень, при этом результаты стендовых испытаний доступны только для ограниченного набора конструкций, стендовые условия достаточно сильно отличаются от скважинных - ниже давление, другие модельные рабочие жидкости, точно оценить коэффициент сепарации газосепаратора в промысловых условиях затруднительно - набор таких данных для сравнения ограничен. 

Тем не менее изучение результатов стендовых испытаний полезно при проведении расчетов и развивает инженерную интуицию. 

\putlisting{listings/MF_ksep_gasseparator_d.lst}


\subsection{MF\_ksep\_total\_d – общая сепарация газа}

Функция рассчитывает полную сепарацию газа на приёме насосе в скважине по известным значениям естественной сепарации газа и коэффициента сепарации газосепаратора. Результат - безразмерная величина в диапазоне от 0 до 1. 

$$K_{sep\_total} = K_{sep\_nat} + (1-K_{sep\_nat}) K_{sep\_gassep}$$

\putlisting{listings/MF_ksep_total_d.lst}

\section{Расчёт многофазного потока в штуцере}


Штуцер или локальное гидравлическое сопротивление - элемент скважины или системы трубопроводов, применяемых для создания дополнительного перепада давления в системе и ограничения потока. 
Возможны различные варианты реализации штуцера - со штуцерной камерой, с угловым краном, позволяющим менять диаметр штуцера и другие.
Ключевым параметром штуцера является диаметр \(d_{choke} \) определяющий его способность к ограничению потока. 

\begin{figure}[h!]
	\begin{center}
	    		% https://www.mathcha.io/editor# использован для построения картинок



		
		\tikzset{every picture/.style={line width=0.75pt}} %set default line width to 0.75pt        
		
		\begin{tikzpicture}[x=0.75pt,y=0.75pt,yscale=-1,xscale=1]
		%uncomment if require: \path (0,300); %set diagram left start at 0, and has height of 300
		
		%Shape: Rectangle [id:dp8089540927658381] 
		\draw  [color={rgb, 255:red, 0; green, 0; blue, 0 }  ,draw opacity=1 ][fill={rgb, 255:red, 155; green, 155; blue, 155 }  ,fill opacity=1 ][line width=2.25]  (92,42) -- (570.83,42) -- (570.83,56.33) -- (92,56.33) -- cycle ;
		%Shape: Rectangle [id:dp7288541809010827] 
		\draw  [fill={rgb, 255:red, 155; green, 155; blue, 155 }  ,fill opacity=1 ][line width=2.25]  (92,227) -- (570.83,227) -- (570.83,241) -- (92,241) -- cycle ;
		%Shape: Rectangle [id:dp666453613189492] 
		\draw  [color={rgb, 255:red, 0; green, 0; blue, 0 }  ,draw opacity=1 ][fill={rgb, 255:red, 155; green, 155; blue, 155 }  ,fill opacity=1 ][line width=2.25]  (323.83,56.33) -- (341.17,56.33) -- (341.17,118.67) -- (323.83,118.67) -- cycle ;
		%Shape: Rectangle [id:dp015115451250117262] 
		\draw  [fill={rgb, 255:red, 155; green, 155; blue, 155 }  ,fill opacity=1 ][line width=2.25]  (323.83,165) -- (341.83,165) -- (341.83,226.83) -- (323.83,226.83) -- cycle ;
		%Right Arrow [id:dp058738740185342975] 
		\draw   (231,133.5) -- (274.56,133.5) -- (274.56,127) -- (289.83,140) -- (274.56,153) -- (274.56,146.5) -- (231,146.5) -- cycle ;
		%Straight Lines [id:da28021737295590965] 
		\draw    (341,119) -- (455,119) ;
		
		
		%Straight Lines [id:da8575303554097866] 
		\draw    (341,165) -- (455,165) ;
		
		
		%Straight Lines [id:da44299065539354565] 
		\draw    (440,120.89) -- (440,161.67) ;
		\draw [shift={(440,163.67)}, rotate = 270.28] [color={rgb, 255:red, 0; green, 0; blue, 0 }  ][line width=0.75]    (10.93,-3.29) .. controls (6.95,-1.4) and (3.31,-0.3) .. (0,0) .. controls (3.31,0.3) and (6.95,1.4) .. (10.93,3.29)   ;
		\draw [shift={(440.22,118.89)}, rotate = 90.28] [color={rgb, 255:red, 0; green, 0; blue, 0 }  ][line width=0.75]    (10.93,-3.29) .. controls (6.95,-1.4) and (3.31,-0.3) .. (0,0) .. controls (3.31,0.3) and (6.95,1.4) .. (10.93,3.29)   ;
		%Shape: Rectangle [id:dp8558237837917941] 
		\draw  [color={rgb, 255:red, 155; green, 155; blue, 155 }  ,draw opacity=1 ][fill={rgb, 255:red, 155; green, 155; blue, 155 }  ,fill opacity=1 ] (325.94,51) -- (339.28,51) -- (339.28,91) -- (325.94,91) -- cycle ;
		%Shape: Rectangle [id:dp8173981538013828] 
		\draw  [color={rgb, 255:red, 155; green, 155; blue, 155 }  ,draw opacity=1 ][fill={rgb, 255:red, 155; green, 155; blue, 155 }  ,fill opacity=1 ] (325.94,196) -- (339.94,196) -- (339.94,236) -- (325.94,236) -- cycle ;
		
		% Text Node
		\draw (207.67,141.04) node [scale=1.2,rotate=-0.61]  {$Q_{liq}$};
		% Text Node
		\draw (472,142.04) node [scale=1.2,rotate=-0.61]  {$d_{choke}$};
		% Text Node
		\draw (117.33,142) node [scale=1.44,rotate=-0.74]  {$P_{in}$};
		% Text Node
		\draw (540.67,140.37) node [scale=1.44,rotate=-0.74]  {$P_{out}$};
		
		
		\end{tikzpicture}
		\caption{Схема локального гидравлического сопротивления - штуцера}
		\label{ris:Pipe_choke}
	\end{center}
\end{figure}

Как и у любого элемента гидравлического потока есть три ключевых параметра - давление на входе \( P_{in} \), давление на выходе \(P_{out}\)  и расход газожидкостной смеси, обычно задаваемый в стандартных условиях \(Q_{liq} \). Задание любых двух элементов позволяет вычислить третий. При задании трех элементов модель штуцера может быть настроена на замеры за счёт подбора калибровочного параметра.

Следует обратить внимание, расчёт перепада давления в штуцере сильно зависит от направления расчета. При фиксированном давлении на выходе $P_{out}$, что для скважины и штуцера на устье соответствует заданному давлению в линии, для любого расхода ГЖС через штуцер можно найти соответствующее значение давления на входе \ref{ris:choke_out_curves}.
 
\begin{figure}[h!]
	
	\begin{center}
		
		\newcommand{\dPipeDataFile}{data/choke1.prn}
		\begin{tikzpicture}[scale=1]
		\begin{axis}[
		width=14cm,
		height=8cm,
		xlabel=$Q\; m^3 / day$,
		ylabel=$P_{in} \; atma$,
		legend pos=south east,
		title=Перепад давления в штуцере]
		\addplot table [y=Pout_1, x=Q]{\dPipeDataFile};
		\addlegendentry{$P_{out}=1$}
		\addplot table [y=Pout_5, x=Q]{\dPipeDataFile};
		\addlegendentry{$P_{out}=5$}
		\addplot table [y=Pout_10, x=Q]{\dPipeDataFile};
		\addlegendentry{$P_{out}=10$}
		\addplot table [y=Pout_15, x=Q]{\dPipeDataFile};
		\addlegendentry{$P_{out}=15$}
		\addplot table [y=Pout_20, x=Q]{\dPipeDataFile};
		\addlegendentry{$P_{out}=20$}
		\addplot table [y=Pout_30, x=Q]{\dPipeDataFile};
		\addlegendentry{$P_{out}=30$}
		\end{axis}
		\end{tikzpicture}
		
		
		\caption{Кривые зависимости давления на входе в штуцер от дебита при фиксированном давлении на выходе из штуцера $P_{out}$}
		\label{ris:choke_out_curves}
		
	\end{center}
\end{figure} 

А вот при фиксированном давлении на входе $P_{in}$ или фиксированном буферном давлении $P_{buf}$ не для всякого расхода ГЖС можно рассчитать давление на выходе \ref{ris:choke_in_curves}. При фиксированном давлении на входе $P_{in}$ существует максимальный расход ГЖС, который можно прокачать через штуцер с заданным диаметром проходного канала. Такой расход называется критическим. При критическом расходе в канале штуцера скорость потока достигает скорости звука и давление на входе перестает зависеть от давления за штуцером. Величина критического расхода через штуцер зависит от давления на входе, поскольку с повышением давления увеличивается скорость звука в среде.

Вертикальная линия на графике зависимости давления на выходе $P_{out}$ от дебита при критическом расходе показывает, что давление не определяется однозначно, а может принимать любое значение на вертикальной линии. Подобная неоднозначность расчетного давления на выходе штуцера может осложнять расчеты и должна учитываться инженером разрабатывающим расчетный модуль или проводящим расчёты.

\begin{figure}[h!]
	
	\begin{center}
		
		\newcommand{\dPipeDataFile}{data/choke2.prn}
		\begin{tikzpicture}[scale=1]
		\begin{axis}[
		width=14cm,
		height=8cm,
		xlabel=$Q\; m^3 / day$,
		ylabel=$P_{out} \; atma$,
		legend pos=south east,
		title=Перепад давления в штуцере]
		\addplot table [y=Pin_10, x=Q]{\dPipeDataFile};
		\addlegendentry{$P_{in}=10$}
		\addplot table [y=Pin_15, x=Q]{\dPipeDataFile};
		\addlegendentry{$P_{in}=15$}
		\addplot table [y=Pin_20, x=Q]{\dPipeDataFile};
		\addlegendentry{$P_{in}=20$}
		\addplot table [y=Pin_25, x=Q]{\dPipeDataFile};
		\addlegendentry{$P_{in}=25$}
		\addplot table [y=Pin_30, x=Q]{\dPipeDataFile};
		\addlegendentry{$P_{in}=30$}
		\addplot table [y=Pin_35, x=Q]{\dPipeDataFile};
		\addlegendentry{$P_{in}=35$}
		\end{axis}
		\end{tikzpicture}
		
		
		\caption{Кривые зависимости давления на выходе из штуцера от дебита при фиксированном давлении на входе в штуцер $P_{in}$}
		\label{ris:choke_in_curves}
		
	\end{center}
\end{figure} 

Функции расчета штуцера позволяют настроить модель штуцера на замерные данные. Настройка проводится за счет параметра калибровки $c_{calibr}$ \mintinline{vb.net}{c_calibr_fr}. 
Параметр калибровки $c_{calibr}$ применяется как множитель на дебит при расчете характеристики штуцера. 
$$Q_{real} = Q_{calc} * c_{calibr}$$
Таким образом $c_{calibr}=1$ отключает калибровку. А изменение $c_{calibr}$ позволит изменить характеристику штуцера для согласования с измерениями \ref{ris:choke_cal_curves}.

\begin{figure}[h!]
	
	\begin{center}
		
		\newcommand{\dPipeDataFile}{data/choke3.prn}
		\begin{tikzpicture}[scale=1]
		\begin{axis}[
		width=14cm,
		height=6cm,
		xlabel=$Q\; m^3 / day$,
		ylabel=$P_{out} \; atma$,
		legend pos=south west,
		title=Пример калибровки модели штуцера]
		\addplot table [y=cal_1, x=Q]{\dPipeDataFile};
		\addlegendentry{$c_{calibr}=1$}
		\addplot table [y=cal_1.2, x=Q]{\dPipeDataFile};
		\addlegendentry{$c_{calibr}=1.2$}
		\end{axis}
		\end{tikzpicture}
		
		
		\caption{Кривые зависимости давления на выходе из штуцера от дебита при фиксированном давлении на входе в штуцер $P_{in}$}
		\label{ris:choke_cal_curves}
		
	\end{center}
\end{figure}  

Все функции для расчета штуцера содержат в названии слово \mintinline{vb.net}{choke}.  

\subsection{MF\_p\_choke\_atma – Расчет давления на входе или на выходе штуцера}
Функция позволяет рассчитать давление на входе или выходе штуцера по известному давлению на противоположном конце при известных параметрах потока (дебите жидкости, обводнённости, газовому фактору). Расчёт проводится по корреляции Перкинса \cite{Perkins_1993} с учётом многофазного потока. 
 
\putlisting{listings/MF_p_choke_atma.lst}

%\subsection{MF\_dp\_choke\_atm – Расчёт перепада давления в штуцере}
%Функция позволяет рассчитать по известному линейному давлению и дебиту или по известному буферному давлению и дебиту перепад давления.  Расчет проводится по корреляции Перкинса \cite{Perkins_1993} с учетом многофазного потока.  
%Функция возвращает перепад давления и температуры в виде массива.
%\putlisting{listings/MF_dp_choke_atm.lst}


\subsection{MF\_qliq\_choke\_sm3day – функция расчёта дебита жидкости через штуцер}
Функция позволяет рассчитать по известному буферному давлению и линейному давлению дебит жидкости. Расчет проводится по корреляции Перкинса \cite{Perkins_1993} с учетом многофазного потока.  

\putlisting{listings/MF_qliq_choke_sm3day.lst}


\subsection{MF\_calibr\_choke\_fr – функция настройки модели штуцера}
Функция позволяет рассчитать корректирующий фактор для модели штуцера, позволяющий согласовать результаты замеров давления и дебита. Расчет проводится по корреляции Перкинса \cite{Perkins_1993} с учетом многофазного потока.  

\putlisting{listings/MF_calibr_choke_fr.lst}

\newpage
\section{Расчет многофазного потока в трубе}

Для расчета участка трубы с использованием пользовательских функций Унифлок применяется следующая схема - \ref{ris:Pipe_scheme_1}.

Участок трубы задается как прямой с постоянным наклоном $\theta$  длиной $L$, постоянного диаметра $d$. Поток движется под углом $\theta$ к горизонтальной плоскости. Угол  $\theta$ меняется от -90 до 90 градусов Цельсия. Отрицательная величина  $\theta < 0 $ означает, что поток движется вниз - например отрицательным будет угол наклона для нагнетательной скважины. Угол наклона $\theta = 0 $ соответствует потоку в горизонтальном участке трубопровода.

Труба имеет постоянную по всей длине шероховатость стенок. 

\begin{figure}[h!]
	\begin{center}
		% https://www.mathcha.io/editor# использован для построения картинок

\tikzset{every picture/.style={line width=0.75pt}} %set default line width to 0.75pt        

\begin{tikzpicture}[x=0.75pt,y=0.75pt,yscale=-1,xscale=1]
%uncomment if require: \path (0,395.3333282470703); %set diagram left start at 0, and has height of 395.3333282470703

%Shape: Can [id:dp2899696286091056] 
\draw  [fill={rgb, 255:red, 250; green, 245; blue, 184 }  ,fill opacity=1 ][line width=2.25]  (164.23,345.91) -- (471.15,94.25) .. controls (473.82,92.06) and (481.92,97.51) .. (489.23,106.43) .. controls (496.54,115.34) and (500.3,124.35) .. (497.63,126.55) -- (190.71,378.2)(164.23,345.91) .. controls (166.9,343.71) and (175,349.16) .. (182.31,358.08) .. controls (189.63,367) and (193.39,376.01) .. (190.71,378.2) .. controls (188.03,380.4) and (179.94,374.95) .. (172.62,366.03) .. controls (165.31,357.11) and (161.55,348.1) .. (164.23,345.91) -- cycle ;
%Shape: Arc [id:dp7673222576415257] 
\draw  [draw opacity=0] (213.54,358.73) .. controls (215.72,361.29) and (217.51,364.25) .. (218.76,367.57) .. controls (220.22,371.43) and (220.84,375.4) .. (220.7,379.27) -- (190.71,378.2) -- cycle ; \draw   (213.54,358.73) .. controls (215.72,361.29) and (217.51,364.25) .. (218.76,367.57) .. controls (220.22,371.43) and (220.84,375.4) .. (220.7,379.27) ;
%Straight Lines [id:da2539925089352497] 
\draw    (111.83,289.33) -- (164.23,345.91) ;


%Straight Lines [id:da27080386920459176] 
\draw    (426.06,41.14) -- (471.15,94.25) ;


%Straight Lines [id:da6784647335940455] 
\draw    (138.03,317.62) -- (448.6,67.7) ;


%Straight Lines [id:da42906958912244875] 
\draw    (343.67,226) -- (373,202.37) ;
\draw [shift={(374.56,201.11)}, rotate = 501.14] [color={rgb, 255:red, 0; green, 0; blue, 0 }  ][line width=0.75]    (10.93,-3.29) .. controls (6.95,-1.4) and (3.31,-0.3) .. (0,0) .. controls (3.31,0.3) and (6.95,1.4) .. (10.93,3.29)   ;

%Straight Lines [id:da31602361859897776] 
\draw    (89,379) -- (569.56,379) ;
\draw [shift={(571.56,379)}, rotate = 180] [color={rgb, 255:red, 0; green, 0; blue, 0 }  ][line width=0.75]    (10.93,-3.29) .. controls (6.95,-1.4) and (3.31,-0.3) .. (0,0) .. controls (3.31,0.3) and (6.95,1.4) .. (10.93,3.29)   ;


% Text Node
\draw (233.67,364.33) node   {$\theta $};
% Text Node
\draw (269.67,187.67) node [rotate=-2.44]  {$L$};
% Text Node
\draw (200.33,343.67) node [rotate=-0.74]  {$P_{in}$};
% Text Node
\draw (472.67,120) node [rotate=-0.74]  {$P_{out}$};
% Text Node
\draw (335.67,230.67) node [rotate=-0.61]  {$Q_{liq}$};


\end{tikzpicture}
		\caption{Схема трубы принятая для расчётов с использованием пользовательских функций}
		\label{ris:Pipe_scheme_1}
	\end{center}
\end{figure}

Для расчёта распределения давления в трубе необходимо задать граничное значение давления на одном из концов трубы. Граничное давление всегда задается параметром  \mintinline{vb.net}{Pcalc_atma}. Температура потока в точке, где задается давление, задается параметром  \mintinline{vb.net}{T_calc_C}.  Возможно два варианта задания условия - по потоку  \ref{ris:Pipe_scheme_2}  \mintinline{vb.net}{calc_along_flow=1}. и против потока  \ref{ris:Pipe_scheme_3} \mintinline{vb.net}{calc_along_flow=0}. 

\begin{figure}[h!]
	\begin{center}
				\tikzset{every picture/.style={line width=0.75pt}} %set default line width to 0.75pt        
		
		\begin{tikzpicture}[x=0.75pt,y=0.75pt,yscale=-1,xscale=1]
		%uncomment if require: \path (0,390); %set diagram left start at 0, and has height of 390
		
		%Shape: Can [id:dp7382807235009181] 
		\draw  [fill={rgb, 255:red, 250; green, 245; blue, 184 }  ,fill opacity=1 ][line width=2.25]  (176.23,350.28) -- (483.15,98.62) .. controls (485.82,96.43) and (493.92,101.88) .. (501.23,110.8) .. controls (508.54,119.72) and (512.3,128.72) .. (509.63,130.92) -- (202.71,382.57)(176.23,350.28) .. controls (178.9,348.08) and (187,353.54) .. (194.31,362.45) .. controls (201.63,371.37) and (205.39,380.38) .. (202.71,382.57) .. controls (200.03,384.77) and (191.94,379.32) .. (184.62,370.4) .. controls (177.31,361.48) and (173.55,352.47) .. (176.23,350.28) -- cycle ;
		%Shape: Arc [id:dp26711502457409386] 
		\draw  [draw opacity=0] (225.54,363.1) .. controls (227.72,365.66) and (229.51,368.62) .. (230.76,371.95) .. controls (232.22,375.8) and (232.84,379.77) .. (232.7,383.64) -- (202.71,382.57) -- cycle ; \draw   (225.54,363.1) .. controls (227.72,365.66) and (229.51,368.62) .. (230.76,371.95) .. controls (232.22,375.8) and (232.84,379.77) .. (232.7,383.64) ;
		%Straight Lines [id:da2708011557353165] 
		\draw    (123.83,293.7) -- (176.23,350.28) ;
		
		
		%Straight Lines [id:da09547020181005683] 
		\draw    (438.06,45.51) -- (483.15,98.62) ;
		
		
		%Straight Lines [id:da29893852101776] 
		\draw    (150.03,321.99) -- (460.6,72.07) ;
		
		
		%Straight Lines [id:da6573662742713218] 
		\draw    (355.67,230.37) -- (385,206.74) ;
		\draw [shift={(386.56,205.48)}, rotate = 501.14] [color={rgb, 255:red, 0; green, 0; blue, 0 }  ][line width=0.75]    (10.93,-3.29) .. controls (6.95,-1.4) and (3.31,-0.3) .. (0,0) .. controls (3.31,0.3) and (6.95,1.4) .. (10.93,3.29)   ;
		
		%Straight Lines [id:da5485107193914818] 
		\draw    (101,383.37) -- (581.56,383.37) ;
		\draw [shift={(583.56,383.37)}, rotate = 180] [color={rgb, 255:red, 0; green, 0; blue, 0 }  ][line width=0.75]    (10.93,-3.29) .. controls (6.95,-1.4) and (3.31,-0.3) .. (0,0) .. controls (3.31,0.3) and (6.95,1.4) .. (10.93,3.29)   ;
		
		%Right Arrow [id:dp6841435853741948] 
		\draw  [fill={rgb, 255:red, 245; green, 166; blue, 35 }  ,fill opacity=1 ] (135.99,250.8) -- (380.02,50.35) -- (378.16,48.09) -- (395.29,41.6) -- (385.59,57.14) -- (383.73,54.88) -- (139.71,255.32) -- cycle ;
		
		% Text Node
		\draw (245.67,368.7) node   {$\theta $};
		% Text Node
		\draw (281.67,192.04) node [rotate=-2.44]  {$L$};
		% Text Node
		\draw (207.33,354.04) node [rotate=-0.74]  {$P_{in}$};
		% Text Node
		\draw (487.67,117.37) node [rotate=-0.74]  {$P_{out}$};
		% Text Node
		\draw (347.67,235.04) node [rotate=-0.61]  {$Q_{liq}$};
		% Text Node
		\draw  [color={rgb, 255:red, 0; green, 0; blue, 0 }  ,draw opacity=1 ][fill={rgb, 255:red, 245; green, 166; blue, 35 }  ,fill opacity=1 ]  (107, 268.37) circle [x radius= 25.3, y radius= 25.3]   ;
		\draw (107,268.37) node [scale=1.2,rotate=-359.71]  {$P_{calc}$};
		% Text Node
		\draw  [fill={rgb, 255:red, 245; green, 166; blue, 35 }  ,fill opacity=1 ]  (417.67, 24.37) circle [x radius= 23.2, y radius= 23.2]   ;
		\draw (417.67,24.37) node [scale=1.2,rotate=-0.74]  {$P_{out}$};
		
		
		\end{tikzpicture}		
		\caption{Схема расчёта распределения давления по потоку \mintinline{vb.net}{calc_along_flow=1}}
		\label{ris:Pipe_scheme_2}
	\end{center}
\end{figure} 

Схема расчета распределения давления по потоку для случая вертикальной добывающей скважины соответствует расчету распределения давления "снизу вверх" - от забойного давления к устьевому.

\begin{figure}[h!]
	\begin{center}
			
	\tikzset{every picture/.style={line width=0.75pt}} %set default line width to 0.75pt        
	
	\begin{tikzpicture}[x=0.75pt,y=0.75pt,yscale=-1,xscale=1]
	%uncomment if require: \path (0,453); %set diagram left start at 0, and has height of 453
	
	%Shape: Can [id:dp7385102204739014] 
	\draw  [fill={rgb, 255:red, 250; green, 245; blue, 184 }  ,fill opacity=1 ][line width=2.25]  (198.23,370.28) -- (505.15,118.62) .. controls (507.82,116.43) and (515.92,121.88) .. (523.23,130.8) .. controls (530.54,139.72) and (534.3,148.72) .. (531.63,150.92) -- (224.71,402.57)(198.23,370.28) .. controls (200.9,368.08) and (209,373.54) .. (216.31,382.45) .. controls (223.63,391.37) and (227.39,400.38) .. (224.71,402.57) .. controls (222.03,404.77) and (213.94,399.32) .. (206.62,390.4) .. controls (199.31,381.48) and (195.55,372.47) .. (198.23,370.28) -- cycle ;
	%Shape: Arc [id:dp4602786709186524] 
	\draw  [draw opacity=0] (247.54,383.1) .. controls (249.72,385.66) and (251.51,388.62) .. (252.76,391.95) .. controls (254.22,395.8) and (254.84,399.77) .. (254.7,403.64) -- (224.71,402.57) -- cycle ; \draw   (247.54,383.1) .. controls (249.72,385.66) and (251.51,388.62) .. (252.76,391.95) .. controls (254.22,395.8) and (254.84,399.77) .. (254.7,403.64) ;
	%Straight Lines [id:da5353213855148988] 
	\draw    (145.83,313.7) -- (198.23,370.28) ;
	
	
	%Straight Lines [id:da6371642797081225] 
	\draw    (460.06,65.51) -- (505.15,118.62) ;
	
	
	%Straight Lines [id:da5708425902799406] 
	\draw    (172.03,341.99) -- (482.6,92.07) ;
	
	
	%Straight Lines [id:da5694938455522887] 
	\draw    (377.67,250.37) -- (407,226.74) ;
	\draw [shift={(408.56,225.48)}, rotate = 501.14] [color={rgb, 255:red, 0; green, 0; blue, 0 }  ][line width=0.75]    (10.93,-3.29) .. controls (6.95,-1.4) and (3.31,-0.3) .. (0,0) .. controls (3.31,0.3) and (6.95,1.4) .. (10.93,3.29)   ;
	
	%Straight Lines [id:da47656732513988986] 
	\draw    (123,403.37) -- (603.56,403.37) ;
	\draw [shift={(605.56,403.37)}, rotate = 180] [color={rgb, 255:red, 0; green, 0; blue, 0 }  ][line width=0.75]    (10.93,-3.29) .. controls (6.95,-1.4) and (3.31,-0.3) .. (0,0) .. controls (3.31,0.3) and (6.95,1.4) .. (10.93,3.29)   ;
	
	%Right Arrow [id:dp49730840239995544] 
	\draw  [fill={rgb, 255:red, 245; green, 166; blue, 35 }  ,fill opacity=1 ] (419.38,64.16) -- (174.9,264.05) -- (176.75,266.31) -- (159.61,272.77) -- (169.34,257.26) -- (171.19,259.52) -- (415.67,59.63) -- cycle ;
	
	% Text Node
	\draw (267.67,388.7) node   {$\theta $};
	% Text Node
	\draw (303.67,212.04) node [rotate=-2.44]  {$L$};
	% Text Node
	\draw (229.33,374.04) node [rotate=-0.74]  {$P_{in}$};
	% Text Node
	\draw (509.67,137.37) node [rotate=-0.74]  {$P_{out}$};
	% Text Node
	\draw (369.67,255.04) node [rotate=-0.61]  {$Q_{liq}$};
	% Text Node
	\draw  [color={rgb, 255:red, 0; green, 0; blue, 0 }  ,draw opacity=1 ][fill={rgb, 255:red, 245; green, 166; blue, 35 }  ,fill opacity=1 ]  (444, 43.37) circle [x radius= 25.3, y radius= 25.3]   ;
	\draw (444,43.37) node [scale=1.2,rotate=-359.71]  {$P_{calc}$};
	% Text Node
	\draw  [fill={rgb, 255:red, 245; green, 166; blue, 35 }  ,fill opacity=1 ]  (129.67, 288.37) circle [x radius= 21.57, y radius= 21.57]   ;
	\draw (129.67,288.37) node [scale=1.2,rotate=-0.74]  {$P_{in}$};
	
	
	\end{tikzpicture}
		\caption{Схема расчета распределения давления против потока \mintinline{vb.net}{calc_along_flow=0}}
		\label{ris:Pipe_scheme_3}
	\end{center}
\end{figure} 

Схема расчета распределения давления против потока для случая вертикальной добывающей скважины соответствует расчету распределения давления "сверху вниз" - от устьевого давления к забойному.

\subsection{MF\_dp\_pipe\_atm – расчёт перепада давления в трубе}

Функция позволяет рассчитать перепад давления в участке трубопровода. 
Функция возвращает давление и температуру в виде массива.

\putlisting{listings/MF_dp_pipe_atm.lst}

Ниже на рисунке приведены результаты расчёта кривой оттока (перепада давления в вертикальной трубе) для различных корреляций, реализованных в \unf{}.

\begin{figure}[h!]
	
\begin{center}

\newcommand{\dPipeDataFile}{data/dPipe.txt}
\begin{tikzpicture}[scale=1]
\begin{axis}[
width=14cm,
height=10cm,
xlabel=$Q\; m^3 / day$,
ylabel=$P_{wf} \; atma$,
legend pos=south east,
title=Pipe Pressure Drop]
\addplot table [y=P_0, x=Q]{\dPipeDataFile};
\addlegendentry{Beggs Brill}
\addplot table [y=P_1, x=Q]{\dPipeDataFile};
\addlegendentry{Ansari}
\addplot table [y=P_2, x=Q]{\dPipeDataFile};
\addlegendentry{Unified}
\addplot table [y=P_3, x=Q]{\dPipeDataFile};
\addlegendentry{Gray}
\addplot table [y=P_4, x=Q]{\dPipeDataFile};
\addlegendentry{Hagedorn Brown}
\addplot table [y=P_5, x=Q]{\dPipeDataFile};
\addlegendentry{Sakharov Mokhov}
\end{axis}
\end{tikzpicture}


	\caption{Кривые характеристики многофазного потока для вертикальных труб рассчитанные с использованием различных корреляций }
\label{ris:VLP_curves}

\end{center}
\end{figure}

\subsection{MF\_p\_pipe\_atma – функция расчета давления на конце трубы}  

\putlisting{listings/MF_p_pipe_atma.lst}

\subsection{MF\_p\_pipe\_znlf\_atma – функция расчета давления на конце трубы при барботаже}  

\putlisting{listings/MF_p_pipe_znlf_atma.lst}

\subsection{MF\_dpdl\_atmm – функция расчета градиента давления по многофазной корреляции}  

\putlisting{listings/MF_dpdl_atmm.lst}

\newpage

        % Глава 2 многофазный поток
	
\section{Расчет многофазного потока в пласте}
Для анализа работы скважины и скважинного оборудования в большинстве случаев достаточно простейшего подхода для описания производительности пласта. На текущий момент в \unf{} используется линейная индикаторная кривая с поправкой Вогеля для учета разгазирования в призабойной зоне пласта с учетом обводненности \cite{KBrown_AL_methods_vol4}. 

Пользовательские функции для расчета производительности пласта начинаются с префикса  \mintinline{vb.net}{IPR_}. 

Для расчета притока из пласта необходимо определить связь между дебитом жидкости $Q_{liq}$ (притоком) и забойным давлением работающей скважины $P_{wf}$.
Линейная индикаторная кривая на основе закона Дарси задает такую связь через коэффициент продуктивности скважины, который определяется как 
\begin{equation}\label{PI_def} 
 PI = \frac{Q_{liq}}{P_{res} - P_{wf}} 
\end{equation}

где $P_{res}$ - пластовое давление - давление на контуре питания скважины. Закон Дарси описывает установившийся приток несжимаемой жидкости в однородном пласте. 

Соответственно уравнение притока будет иметь вид

$$ Q_{liq} = PI \left( P_{res} - P_{wf} \right) $$

Для линейного притока по закону Дарси коэффициент продуктивности может быть оценен либо по данным эксплуатации из уравнения \ref{PI_def} либо по аналитической зависимости по характеристикам пласта и системы заканчивания. Например для радиального притока к вертикальной скважине широко известна формула Дюпюи согласно которой 
\begin{equation}\label{eq_Dupui} 
PI = f \cdot \frac{kh}{\mu B}\frac{1}{ \ln \dfrac{r_e}{r_w} + S }  
\end{equation}

здесь $f$ - размерный коэффициент, зависящий от выбранной системы единиц для остальных параметров. Так для системы единиц

\newcommand{\rnttab}[1]{
	\begin{tabular}[c]{@{}c@{}}#1\end{tabular}	
}

\begin{table}[]
	\centering
	\caption{Размерности параметров выражения \ref{eq_Dupui}} \label{tab:dim_Dupui}
	\begin{tabular}{|c|c|c|c|c|}
		\hline
		Обозначение & Параметр   			        	& СИ           & \rnttab{Практические \\ метрические}     & \rnttab{Американские\\ промысловые} \\ \hline
		$f$        & \rnttab{размерный \\ коэффициент} & $2\pi$       & $\dfrac{1}{18.41}$     			      & $\dfrac{1}{141.2}$                      \\ \hline
		$k$        & проницаемость           		    & $\text{м}^2$ & мД                    					  & mD   							    \\ \hline
		$h$        & \rnttab{мощность \\ пласта}       & м            & м                      				  & ft   								    \\ \hline		
		$B$        & \rnttab{объемный \\ коэффициент}  & $\text{м}^3/\text{м}^3$    & $\text{м}^3/\text{м}^3$    & $scf/bbl$    						\\ \hline
		$\mu$      & вязкость                           & Па $\cdot$ с & сП                                       & сP                                  \\ \hline
		$r_e$      & \rnttab{радиус зоны \\ дренирования} & м & м                                       & ft                                  \\ \hline
		$r_w$      & \rnttab{радиус  \\ скважины} & м & м                                       & ft                                  \\ \hline
		$S$        & скин фактор 				   & \multicolumn{3}{c|}{безразмерный}                     \\ \hline
	\end{tabular}
\end{table}
 
 При снижении забойного давления добывающей скважины ниже давления насыщения, оценка дебита жидкости по закону Дарси  оказывается завышенной. Газ выделяющийся в призабойной зоне пласта создает дополнительное гидравлическое сопротивление.  В \unf{} поправка на снижение забойного давления ниже давления насыщения реализована на основе поправки Вогеля. Для безводной нефти по Вогелю продуктивность скважины по данным тестовой эксплуатации - дебите жидкости $Q_{liq}$ и соответствующем забойном давлении $P_{wf}$ может быть оценен по выражению \ref{eq_Vogel}.
 
 \begin{equation}\label{eq_Vogel} 
 PI = \frac{Q_{liq}}{P_{res} - P_{b} + \dfrac{P_{b}}{1.8} \left[ 1.0 - 0.2  \dfrac{P_{wf}}{P_{b}}- 0.8 \left( \dfrac{P_{wf}}{P_{b}} \right)^2 \right] }   
 \end{equation}
 
 При наличии обводненности зависимость усложняется.
 
 В \unf{} реализована модель определения коэффициента продуктивности по данным эксплуатации. Сравнение индикаторных кривых, построенных по тестовым данным $Q_{liq} = 100$ и $P_{wf} = 150$ при наличии и отсутствии воды, приведено на рисунке \ref{ris:IPR_curves}. 
 
 \begin{figure}[h!]
 	\begin{center}
		 \newcommand{\IPRFile}{data/IPR_fw_data.txt}
		 \begin{tikzpicture}[scale=1]
		 \begin{axis}[
		 width=14cm,
		 height=10cm,
		 xlabel={$Q, m^3 / day$},
		 ylabel={$P_{wf},  atma$},
		 legend pos=north east,
		 title= Индикаторные кривые IPR]
		 \addplot table [y=Pwf_fw0, x=Q_fw0]{\IPRFile};
		 \addlegendentry{$f_w = 0$}
		 \addplot table [y=Pwf_fw95, x=Q_fw95]{\IPRFile};
		\addlegendentry{$f_w = 95$}
		 \end{axis}
		 \end{tikzpicture}
 	\caption{Сравнение индикаторных кривых для заданных тестовых параметров $Q_{liq} = 100$ и $P_{wf} = 150$ при наличии и отсутствии воды в потоке }
 \label{ris:IPR_curves}
\end{center}
\end{figure}
 

\subsection{IPR\_pi\_sm3dayatm – расчёт продуктивности}
Функция позволяет рассчитать коэффициент продуктивности скважины по данным тестовой эксплуатации. Особенность линейной модели притока к скважине с поправкой Вогеля заключается в минимальном наборе исходных данных, необходимых для построения индикаторной кривой. Достаточно знать пластовое давление, дебит и забойное давление в одной точке.

\putlisting{listings/IPR_PI_sm3dayatm.lst}


\subsection{IPR\_pwf\_atm – расчёт забойного давления по дебиту и продуктивности}
Функция позволяет рассчитать забойное давление скважины по известным значениям дебита и продуктивности.

\putlisting{listings/IPR_Pwf_atma.lst}

\subsection{IPR\_qliq\_sm3day – расчёт дебита по забойному давлению и продуктивности}
Функция позволяет рассчитать дебит жидкости скважины на поверхности по забойному давлению и продуктивности.

\putlisting{listings/IPR_Qliq_sm3Day.lst}



\newpage
       % Глава 2 индикаторная кривая
	\chapter{Функции модуля  «u7\_Excel\_functions\_ESP»}
В этом модули приведены интерфейсные функции Excel (функции, которые можно вызывать непосредственно с листа Excel) для расчёта параметров работы УЭЦН - установки электрического центробежного насоса. 

УЭЦН состоит из следующих основных конструктивных элементов:
\begin{itemize}
	\item ЦН - центробежный насос. Модуль обеспечивающий перекачку жидкости.
	\item ПЭД - погружной электрический двигатель. Модуль обеспечивающий преобразование электрической энергии, поступающий к УЭЦН по кабелю в механическую энергию вращения вала.
	\item ГС - газосепаратор или приемный модуль. Модуль обеспечивающий забор пластовой жидкости из скважины и подачу ее в насос. При этом центробежный газосепаратор способе отделить часть свободного газа в потоке и направить его в межтрубное пространство скважины.
	\item вал - узел передающий энергию от погружного электрического двигателя (ПЭД) к остальным узлам установки, в том числе к центробежному насосу.
\end{itemize}

Задача расчета УЭЦН обычно сводится к следующим:
\begin{itemize}
	\item Прямая задача - по заданным значения дебита жидкости скважины,  давлению на приеме, напряжению питания УЭЦН на поверхности найти давление на выкиде насоса, потребляему электрическую мощность, потребляемый ток установки, КПД всей системы и отдельных узлов системы
	\item Обратная задача - по данным контроля параметров работы УЭЦН на поверхности - потребляемому току, напряжению питания частоте подаваемого напряжения, данным по конструкции УЭЦН и скважины найти дебит жидкости и обводнённость по скважине, давление на приеме и забойное давление.
	\item Задача узлового анализа - по данным конструкции скважины, параметров работы погружного оборудования оценить дебит по жидкости скважины при заданным параметрах работы УЭЦН или при из изменении. К этому типу задач относится задача подбора погружного оборудования для достижения заданных условий эксплуатации 
	
\end{itemize}

Для расчёта УЭЦН требуется рассчитать гидравлические параметры работы ЦН и электромеханические параметры ПЭД

\section{Гидравлический расчет центробежного насоса (ЦН)}

Расчет выполняется на основе паспортных характеристик ЦН. 

\section{Электромеханический расчёт погружного электрического двигателя ПЭД}
Рассматривается асинхронный электрический двигатель. 

Погружные асинхронные электрические двигатели для добычи нефти выполяются трехфазными. 

Впервые конструкция трёхфазного асинхронного двигателя была разработана, создана и опробована русским инженером М. О. Доливо-Добровольским в 1889-91 годах. Демонстрация первых двигателей состоялась на Международной электротехнической выставке во Франкфурте на Майне в сентябре 1891 года. На выставке было представлено три трёхфазных двигателя разной мощности. Самый мощный из них имел мощность 1.5 кВт и использовался для приведения во вращение генератора постоянного тока. Конструкция асинхронного двигателя, предложенная Доливо-Добровольским, оказалась очень удачной и является основным видом конструкции этих двигателей до настоящего времени.

За прошедшие годы асинхронные двигатели нашли очень широкое применение в различных отраслях промышленности и сельского хозяйства. Их используют в электроприводе металлорежущих станков, подъёмно-транспортных машин, транспортёров, насосов, вентиляторов. Маломощные двигатели используются в устройствах автоматики.

Широкое применение асинхронных двигателей объясняется их достоинствами по сравнению с другими двигателями: высокая надёжность, возможность работы непосредственно от сети переменного тока, простота обслуживания.

Для расчёта электромеханических параметров погружных электрических двигателей полезно понимать теоретические основы их работы. Теория работы погружных асинхронных двигателей не отличаем от теории применимой к двигателям применяемым на поверхности. Далее кратко изложены основные положения теории. 

Трехфазная цепь является частным случаем многофазных систем электрических цепей, представляющих собой совокупность электрических цепей, в которых действуют синусоидальные ЭДС одинаковой частоты, отличающиеся по фазе одна от другой и создаваемые общим источником энергии.
Переменный ток протекающий по трехфазной цели характеризуется следующими параметрами:

\begin{itemize}
	\item Фазное напряжение $U_A, U_B, U_C $ - напряжение между линейным проводом и нейтралью
	\item Линейное напряжение $U_{AB}, U_{BC}, U_{CA} $ - напряжение между одноименными выводами разных фаз
	\item Фазный ток $I_{phase}$ – ток в фазах двигателя.
	\item Линейный ток $I_{line}$ – ток в линейных проводах.
	\item $ \cos \varphi $ - коэффициент мощности, где $ \varphi$ величина сдвига по фазе между напряжением и током 
\end{itemize}

Подключение двигателя к цепи трехфазного тока может быть выполнено по схеме "звезда" или "треугольник".

Тут нужен рисунок  

Для схемы звезда фазное напряжение меньше линейного в $\sqrt{3}$ раз.

$$ U_{AB} = \sqrt{3} U_{A} $$
$$ I_{phase} = I_{line} $$

Для схемы треугольник 

$$ U_{AB} =  U_{A} $$
$$ I_{line} =\sqrt{3} I_{phase} $$


В погружных двигателях обычно применяет схема подключения звезда. Эта схема обеспечивает более низкое напряжение в линии, что способствует повышению КПД передачи энергии по длинному кабелю. Еще есть причины?
При схеме подключения звезда токи в линии и в фазной обмотке статора двигателя совпадают, поэтому значение тока обозначают $I$ не указывая индекс в явном виде. Поскольку линейное напряжения проще измерить и легче контролировать параметры трехфазного двигателя обычно заданию линейный. в частности номинальное напряжение питания двигателя это линейное напряжение (напряжение между фазами). Далее линейное напряжение будет обозначать без индекса как $U$

Активная электрическая мощность в трехфазной цепи задается выражением 
$$ P= \sqrt{3}U I \cos \varphi$$

Реактивная мощность 
$$ Q= \sqrt{3}U I \sin \varphi$$

Соответственно полная мощность 
$$ S= \sqrt{3}U I $$

\subsection{ Устройство трёхфазной асинхронной машины}
Неподвижная часть машины называется статор, подвижная – ротор. Обмотка статора состоит из трёх отдельных частей, называемых фазами.

При подаче переменного напряжения и тока на обмотки статора внутри статора формируется вращающееся магнитное поле. Частота вращения магнитного поля совпадает с частотой питающего напряжения. 

Магнитный поток $\Phi $ и напряжение подаваемое на статор связаны приближенном соотношением 
$$ U_1 \approx E_1 = 4.44 w_1 k_1 f \Phi $$
где 

 $\Phi$ -  магнитный поток;
 
 $U_1$ -	напряжение в одной фазе статора;
 
 $f$   - частота сети;
 
 $E_1$	- ЭЦН в фазе статора;
 
 $w_1$ - число витков одной фазы обмотки статора;
 
 $k_1$  - обмоточный коэффициент.
   
Из этого выражения следует, что магнитный поток $\Phi $ в асинхронной машине не зависит от её режима работы, а при заданной частоте сети $f$ зависит только от действующего значения приложенного напряжения $U_1$


Для ЭДС ротора можно записать выражение 

$$  E_2 = 4.44 w_2 k_2 f S \Phi $$

где 


$S$ - величина скольжения (проскальзования);

$E_2$	- ЭЦН в фазе ротора;

$w_2$ - число витков одной фазы обмотки ротора;

$k_2$ - обмоточный коэффициент ротора.

ЭДС, наводимая в обмотке ротора, изменяется пропорционально скольжению и в режиме двигателя имеет наибольшее значение в момент пуска в ход.
Для тока ротора в общем случае можно получить такое соотношение

$$  I_2 = \frac{E_2 S}{\sqrt{R_2^2+(S X_2^2)}} $$

где 

$R_2$ -  активное сопротивление обмотки ротора, связанное с потерями на нагрев обмотки;  

$X_2 = 2 \pi f L_2$ - индуктивное сопротивление обмотки неподвижного ротора, связанное с потоком рассеяния;

Отсюда следует, что ток ротора зависит от скольжения и возрастает при его увеличении, но медленнее, чем ЭДС.

Для асинхронного двигателя можно получить следующее выражение для механического момента 

$$ M = \frac{1}{4.44 w_2 k_2 k_T^2 f} \frac{U_1^2 R_2 S}{R_2^2 + (S X_2^2)^2}$$

где 

$k_T = \frac{E_1}{E_2} = \frac{w_1 k_1}{w_2 k_2}$ - коэффициент трансформации асинхронной машины

Из полученного выражения для электромагнитного момента следует, что он сильно зависит от подведённого напряжения ($M \sim U_1^2$). При снижении, например, напряжения на 10\%, электромагнитный момент снизится на 19\% $M \sim (0,9U_1)^2=0.81 U_1^2)$. Это является одним из недостатков асинхронных двигателей. 




\newpage
           % Глава 2 ЭЦН
	\include{text/part4}           % Глава 2 тех режим
	\chapter{Упражнения по работе с пользовательскими функциями \unf}

Освоить работу с расчетными функциями \unf можно выполняя упражнения описанные в данном разделе и изучая устройство тестовых расчетных модулей. Упражнение демонстрируют некоторые подходы к использованию \unf. На основе этих подходов можно создать свои расчетные модули решающие специфические задачи пользователя. 

\section{Расчет PVT свойств}

Расчет физико химических свойств пластовых флюидов лежит в основе всех расчетов систем нефтедобычи. При решении прикладных задач редко возникает необходимость расчета PVT свойств непосредственно, однако понимание принципа их расчета, а особенно зависимости результатов расчета от исходных данных важно.
	
Цель упражнений по расчету PVT свойств:
\begin{itemize}	
	\item 	освоить принципы работы c пользовательскими функций \unf 
	\item 	изучить влияние исходных PVT данных на результаты расчета PVT свойств
	\item 	изучить влияние выбора PVT корреляций на результаты расчета PVT свойств
	\item 	изучить механизм калибровки PVT корреляций на результаты измерений
\end{itemize}
	 
	 
\subsection{Построение простых PVT зависимостей}

Для выполнения упражнения используйте файл "10.PVT.xlsx"

\begin{enumerate}
	\item Запустите файл с надстройкой \unf. Для того чтобы убедиться, что надстройка запущена откройте редактор VBE (Alt+F11). В дереве проектов должен отображаться файл надстройки \mintinline{vb.net}{UniflocVBA_7.xlam}, рис. \ref{ris:VBE_empty}.
	
	\begin{figure}[h!]
		\center{\includegraphics[width=0.5\linewidth]{VBE_empty}}
		\caption{Окно редактора VBE с загруженной надстройкой \unf}
		\label{ris:VBE_empty}
	\end{figure}

	\item Откройте файл с упражнением \texttt{10.PVT.xlsx} (смотри рис. \ref{ris:Ex10_1}).
	
	\begin{figure}[h!]
		\center{\includegraphics[width=0.5\linewidth]{Ex10_1}}
		\caption{Открытый файл с упражнением \texttt{10.PVT.xlsx}}
		\label{ris:Ex10_1}
	\end{figure}
	
	\item Для расчета первого элемента таблицы в ячейках D23:D48 - газосодержания в нефти при давлении 1 атм и температуре 80 °C - введите в ячейку D23 строку
	
	{ \small  \texttt{=PVT\_Rs\_m3m3(B23;C23;gamma\_gas\_;gamma\_oil\_; gamma\_wat\_; Rsb\_; Rp\_; Pb\_; Tres\_; Bob\_; muob\_)}}
	
	Обратите внимание -- при запущенной надстройке достаточно начать вводить в ячейку формулу, например ввести \texttt{=PVT} как Excel откроет выпадающий список с подсказкой, показывающий возможные варианты названий функций (смотри рис. \ref{ris:Ex10_2}). 
	
	В приведенной строке \texttt{B23;C23} - ссылки на соответствующие ячейки,  \texttt{gamma\_gas\_;gamma\_oil\_} - также ссылки на ячейки, которые предварительно были поименованы. 

	\begin{figure}[h!]
		\center{\includegraphics[width=0.5\linewidth]{Ex10_2}}
		\caption{Выпадающий список с подсказками названий функции}
		\label{ris:Ex10_2}
	\end{figure}

	Из выпадающего списка выберите функцию \texttt{=PVT\_Rs\_m3m3(} после чего нажмите кнопку $f_x$ "вставить функцию" слева от строки формул. Это вызовет окно задания параметров функции, в котором будут указаны все параметры, которые необходимо ввести. В этом окно можно ввести необходимые значения параметров или указать ссылки на соответствующие ячейки.

	\begin{figure}[h!]
		\center{\includegraphics[width=0.5\linewidth]{Ex10_3}}
		\caption{Окно ввода аргументов функции}
		\label{ris:Ex10_3}
	\end{figure}

	\item После ввода всех параметров и нажатия кнопки ОК в ячейке должен отобразиться результат расчета. Воспользовавшись инструментом "Влияющие ячейки" на вкладке "Формулы" можно отследить на какие ячейки ссылается введенная формула
	\begin{figure}[h!]
		\center{\includegraphics[width=0.5\linewidth]{Ex10_4}}
		\caption{Результат вызова пользовательской функции с отображение влияющих ячеек}
		\label{ris:Ex10_4}
	\end{figure}

	\item Аналогично заполните все ячейки таблицы  \texttt{D23:D48} вызовами функции \texttt{=PVT\_Rs\_m3m3()} с соответствующими параметрами. Это можно сделать "протянув" ранее введенную функцию в ячейке \texttt{D23}.
	
	Обратите внимание, что при "протягивании" поименованные ячейки оказываются закрепленными, а ссылки на значения давления и температуры съезжают вместе с протягиваемой ячейкой. Результат показан на рисунке \ref{ris:Ex10_5}
	\begin{figure}[h!]
		\center{\includegraphics[width=0.5\linewidth]{Ex10_5}}
		\caption{Результат расчета зависимости газосодержания от давления}
		\label{ris:Ex10_5}
	\end{figure}

	\item По аналогии с зависимостью газосодержания от давления постройте графики зависимости других параметров от давления. Используйте следующие функции для проведения расчатов: 
	
	функция расчета объемного коэффициента нефти
	
	{ \small  \texttt{=PVT\_Bo\_m3m3(B23;C23;gamma\_gas\_;gamma\_oil\_;gamma\_wat\_; Rsb\_; Rp\_; Pb\_;Tres\_;Bob\_;muob\_)}}
	
	функция расчета вязкости нефти при заданных термобарических условиях
	
	{ \small  \texttt{=PVT\_Muo\_cP(B23;C23;gamma\_gas\_;gamma\_oil\_;gamma\_wat\_; Rsb\_; Rp\_; Pb\_;Tres\_;Bob\_;muob\_)}}
	
    функция расчета вязкости газа при заданных термобарических условиях
	
	{ \small  \texttt{=PVT\_Mug\_cP(B23;C23;gamma\_gas\_;gamma\_oil\_;gamma\_wat\_; Rsb\_; Rp\_; Pb\_;Pb\_;Bob\_;muob\_)}}
	
	функция расчета вязкости воды при заданных термобарических условиях
	
	{ \small  \texttt{=PVT\_Muw\_cP(B23;C23;gamma\_gas\_;gamma\_oil\_;gamma\_wat\_; Rsb\_; Rp\_; Pb\_;Tres\_;Bob\_;muob\_)}}
	
	функция расчета плотности газа при заданных термобарических условиях
	
	{ \small  \texttt{=PVT\_Rhog\_kgm3(B23;C23;gamma\_gas\_;gamma\_oil\_;gamma\_wat\_; Rsb\_; Rp\_; Pb\_;Tres\_;Bob\_;muob\_)}}
	
	функция расчета плотности воды при заданных термобарических условиях
	
	{ \small  \texttt{=PVT\_Rhow\_kgm3(B23;C23;gamma\_gas\_;gamma\_oil\_;gamma\_wat\_; Rsb\_; Rp\_; Pb\_;Tres\_;Bob\_;muob\_)}}
	
	функция расчета плотности нефти при заданных термобарических условиях
	
	{ \small  \texttt{=PVT\_Rhoo\_kgm3(B23;C23;gamma\_gas\_;gamma\_oil\_;gamma\_wat\_; Rsb\_; Rp\_; Pb\_;Tres\_;Bob\_;muob\_)}}
	
	Результаты приведены на рисунке \ref{ris:Ex10_6}
	
	\begin{figure}[h!]
		\center{\includegraphics[width=1\linewidth]{Ex10_6}}
		\caption{Результат расчета зависимости свойств пластовых флюидов от давления}
		\label{ris:Ex10_6}
	\end{figure}
	
	\item Ответьте на вопросы по упражнению приведенные в рабочей книге.
	
	\begin{enumerate}
		\item Можно ли глядя на графические зависимости определить параметры нефти? Если да, то какие?
		\item Всегда ли заданное значение давления насыщения совпадает со значением давления насыщения считанным с графиков?
		\item Чему равно значение объемного коэффициента при Р = 1 атма? Есть ли разница между исходным значением и значением определенным по графическими зависимостями?
		\item Как изменятся построенные зависимости если не вводить значения калибровочных параметров - давления насыщения, объемного коэффициента при давлении насыщения, вязкости при давлении насыщения?
		
	\end{enumerate}
 
\end{enumerate}

\section{Расчет производительности скважины}

Модель притока к скважине является достаточно простой и одновременно полезной, позволяя оперативно оценивать добычные возможности скважины. Для индикаторной диаграммы Вогеля зависимость забойного давления от дебита ниже давления насыщения перестает быть линейной.

Для выполнения упражнения необходимо задать:
\begin{enumerate}
	\item PVT свойства флюидов
	\item Параметры работы скважины на установившемся режиме
	\item Пластовое давление
\end{enumerate}


\begin{figure}[h!]
	\center{\includegraphics[width=1\linewidth]{Ex20_1}}
	\caption{Исходные данные для построения индикаторной кривой}
	\label{ris:Ex20_1}
\end{figure}

Коэффициент продуктивности $PI$ скважины рассчитывается в ячейке С25 по замеренным данным  с помощью функции

{ \small  \texttt{=IPR\_PI\_sm3dayatm(qltest\_;Pwftest\_;Pres\_;fw\_;Pb\_)}}

А максимальный дебит $Q_{max}$ при максимальной депрессии с забойным давлением равном нулю

{ \small  \texttt{=IPR\_Qliq\_sm3Day(PI\_;Pres\_;0;fw\_;Pb\_)}}

После задания всех необходимых параметров перейдем к построению индикаторной кривой.

Для расчета забойного давления в зависимости от дебита введите в ячейку D40 строку

{ \small  \texttt{=IPR\_Pwf\_atma(PI\_;Pres\_;C40;fw\_;Pb\_)}}

Для вычисления дебита в зависимости от давления Вы можете воспользоваться функцией 

{ \small  \texttt{=IPR\_Qliq\_sm3Day(PI\_;Pres\_;D40;fw\_;Pb\_)}}

поместив ее в ячейку E40.

\begin{figure}[h!]
	\center{\includegraphics[width=1\linewidth]{Ex20_2}}
	\caption{Результат построения индикаторной кривой}
	\label{ris:Ex20_2}
\end{figure}

Применяя функции, строя дополнительные графики, ответьте на вопросы по упражнению, приведенные в рабочей книге.

	\begin{enumerate}
		\item Как можно оценить продуктивность скважины?
		\item Зависит ли вид индикаторной кривой от газового фактора?
	\end{enumerate}

\section{Набор расчетных модулей анализа скважины}
Пример использования алгоритмов \unf   приведен в файле \texttt{UF7\_calc\_well.xlsm}.

Файл содержит набор расчетных модулей позволяющих провести анализ данных описывающих работу скважины с применением различных методов добычи.


\subsection{Расчетный модуль анализа и настройки PVT свойств}

 % Глава 3 описание упражнений
	\chapter*{Заключение}                       % Заголовок

Заключение возможно будет тут когда то      % Заключение
	\chapter*{Единицы измерений} % Заголовок
\addcontentsline{toc}{chapter}{Единицы измерений}  % Добавляем его в оглавление
\noindent

\section*{Давление}
 
 atm, атм "--- физическая атмосфера 
 
 atma, атма "--- абсолютное значение величины в атмосферах
 
 atmg, атми "--- избыточное (измеренное) значение величины в атмосферах. отличается от абсолютной на величину атмосферного давления (1.01325 атма)

\chapter*{Список сокращений и условных обозначений} % Заголовок
\addcontentsline{toc}{chapter}{Список сокращений и условных обозначений}  % Добавляем его в оглавление
\noindent

$\gamma_g$  - \mintinline{vb.net}{gamma_gas} - удельная плотность газа, по воздуху. 

$\gamma_o$  - \mintinline{vb.net}{gamma_oil} - удельная плотность нефти, по воде.

$\gamma_w$  - \mintinline{vb.net}{gamma_wat}- удельная плотность воды, по воде. 

$R_{sb}$ - \mintinline{vb.net}{Rsb_m3m3} газосодержание при давлении насыщения,  $\text{м}^3/\text{м}^3$. 

$R_p$ - \mintinline{vb.net}{Rp_m3m3}. замерной газовый фактор, $\text{м}^3/\text{м}^3$.

$P_b$ - \mintinline{vb.net}{Pb_atma}. давление насыщения, атма.  

$T_{res}$ - \mintinline{vb.net}{Tres_C} пластовая температура, \textcelsius. 

$B_{ob}$ - \mintinline{vb.net}{Bob_m3m3} объёмный коэффициент нефти,  $\text{м}^3/\text{м}^3$. 

$\mu_{ob}$ - \mintinline{vb.net}{Muob_cP}. вязкость нефти при давлении насыщения, сП. 

$Q_{liq}$ - \mintinline{vb.net}{Qliq_scm3day}. дебит жидкости измеренный на поверхности (приведенный к стандартным условиям), м3/сут. 

$f_{w}$ - \mintinline{vb.net}{fw_perc, fw_fr} объёмная обводненность (fraction of water), проценты или доли единиц. 

$f_{g}$ - \mintinline{vb.net}{fg_perc, fg_fr} объёмная доля газа в потоке (fraction of gas), проценты или доли единиц. 

$PI$ - \mintinline{vb.net}{pi_sm3dayatm} - коэффициент продуктивности скважины, $\text{м}^3$/сут/атм

$\rho_{air}$ - \mintinline{vb.net}{rho_air} - плотность воздуха, относительная плотность газа $\gamma_g$ считается по воздуху $\rho_{air} = 1.22$ кг/$\text{м}^3$        % Список сокращений и условных обозначений
	\chapter*{Словарь терминов}             % Заголовок
\addcontentsline{toc}{chapter}{Словарь терминов}  % Добавляем его в оглавление

\textbf{VBA} "--- Visual Basic for Application язык программрования встроенный в Excel и использованный для написания макросов \unf

\textbf{VBE} "--- Среда разработки для языка VBA. Встроена в Excel

      % Словарь терминов
	\include{text/references}      % Список литературы
	

%%%	\include{text/lists}           % Списки таблиц и изображений (иллюстративный материал)
	
	%%% Настройки для приложений
\appendix
	% Оформление заголовков приложений ближе к ГОСТ:
\setlength{\midchapskip}{20pt}
\renewcommand*{\afterchapternum}{\par\nobreak\vskip \midchapskip}
\renewcommand\thechapter{\Asbuk{chapter}} % Чтобы приложения русскими буквами нумеровались
	
%%%	\include{text/appendix}        % Приложения	
	

\chapter{Автоматически сгенерированное описание}

Далее следует описание расчетных функций \unf автоматически сгенерированное из исходного кода.
Более подробное описание основных функций можно найти в описании выше. Автоматическое описание возможно будет более полным и актуальным пока продолжается разработка.

\section{crv\_fit\_linear}
\putlisting{listings/crv_fit_linear.lst}
\section{crv\_fit\_poly}
\putlisting{listings/crv_fit_poly.lst}
\section{crv\_fit\_spline\_1D}
\putlisting{listings/crv_fit_spline_1D.lst}
\section{crv\_interpolation}
\putlisting{listings/crv_interpolation.lst}
\section{crv\_interpolation\_2D}
\putlisting{listings/crv_interpolation_2D.lst}
\section{crv\_intersection}
\putlisting{listings/crv_intersection.lst}
\section{crv\_parametric\_interpolation}
\putlisting{listings/crv_parametric_interpolation.lst}
\section{crv\_solve}
\putlisting{listings/crv_solve.lst}
\section{Ei}
\putlisting{listings/Ei.lst}
\section{ESP\_calibr\_calc}
\putlisting{listings/ESP_calibr_calc.lst}
\section{ESP\_decode\_string}
\putlisting{listings/ESP_decode_string.lst}
\section{ESP\_dP\_atm}
\putlisting{listings/ESP_dP_atm.lst}
\section{ESP\_eff\_fr}
\putlisting{listings/ESP_eff_fr.lst}
\section{ESP\_encode\_string}
\putlisting{listings/ESP_encode_string.lst}
\section{ESP\_gasseparator\_name}
\putlisting{listings/ESP_gasseparator_name.lst}
\section{ESP\_head\_m}
\putlisting{listings/ESP_head_m.lst}
\section{ESP\_id\_by\_rate}
\putlisting{listings/ESP_id_by_rate.lst}
\section{ESP\_ksep\_gasseparator\_d}
\putlisting{listings/ESP_ksep_gasseparator_d.lst}
\section{esp\_max\_rate\_m3day}
\putlisting{listings/esp_max_rate_m3day.lst}
\section{ESP\_name}
\putlisting{listings/ESP_name.lst}
\section{ESP\_optRate\_m3day}
\putlisting{listings/ESP_optRate_m3day.lst}
\section{ESP\_Power\_W}
\putlisting{listings/ESP_Power_W.lst}
\section{ESP\_p\_atma}
\putlisting{listings/ESP_p_atma.lst}
\section{ESP\_system\_calc}
\putlisting{listings/ESP_system_calc.lst}
\section{E\_1}
\putlisting{listings/E_1.lst}
\section{GLV\_d\_choke\_mm}
\putlisting{listings/GLV_d_choke_mm.lst}
\section{GLV\_IPO\_p\_atma}
\putlisting{listings/GLV_IPO_p_atma.lst}
\section{GLV\_IPO\_p\_close}
\putlisting{listings/GLV_IPO_p_close.lst}
\section{GLV\_IPO\_p\_open}
\putlisting{listings/GLV_IPO_p_open.lst}
\section{GLV\_p\_atma}
\putlisting{listings/GLV_p_atma.lst}
\section{GLV\_p\_bellow\_atma}
\putlisting{listings/GLV_p_bellow_atma.lst}
\section{GLV\_p\_close\_atma}
\putlisting{listings/GLV_p_close_atma.lst}
\section{GLV\_p\_vkr\_atma}
\putlisting{listings/GLV_p_vkr_atma.lst}
\section{GLV\_q\_gas\_sm3day}
\putlisting{listings/GLV_q_gas_sm3day.lst}
\section{GLV\_q\_gas\_vkr\_sm3day}
\putlisting{listings/GLV_q_gas_vkr_sm3day.lst}
\section{GL\_decode\_string}
\putlisting{listings/GL_decode_string.lst}
\section{GL\_encode\_string}
\putlisting{listings/GL_encode_string.lst}
\section{IPR\_PI\_sm3dayatm}
\putlisting{listings/IPR_PI_sm3dayatm.lst}
\section{IPR\_Pwf\_atma}
\putlisting{listings/IPR_Pwf_atma.lst}
\section{IPR\_Qliq\_sm3Day}
\putlisting{listings/IPR_Qliq_sm3Day.lst}
\section{MF\_calibr\_choke\_fr}
\putlisting{listings/MF_calibr_choke_fr.lst}
\section{MF\_calibr\_pipe}
\putlisting{listings/MF_calibr_pipe.lst}
\section{MF\_calibr\_pipeline}
\putlisting{listings/MF_calibr_pipeline.lst}
\section{MF\_CJT\_Katm}
\putlisting{listings/MF_CJT_Katm.lst}
\section{MF\_dpdl\_atmm}
\putlisting{listings/MF_dpdl_atmm.lst}
\section{MF\_fit\_pipe\_m3day}
\putlisting{listings/MF_fit_pipe_m3day.lst}
\section{MF\_gas\_fraction\_d}
\putlisting{listings/MF_gas_fraction_d.lst}
\section{MF\_ksep\_natural\_d}
\putlisting{listings/MF_ksep_natural_d.lst}
\section{MF\_ksep\_total\_d}
\putlisting{listings/MF_ksep_total_d.lst}
\section{MF\_mu\_mix\_cP}
\putlisting{listings/MF_mu_mix_cP.lst}
\section{MF\_p\_choke\_atma}
\putlisting{listings/MF_p_choke_atma.lst}
\section{MF\_p\_gas\_fraction\_atma}
\putlisting{listings/MF_p_gas_fraction_atma.lst}
\section{MF\_p\_pipeline\_atma}
\putlisting{listings/MF_p_pipeline_atma.lst}
\section{MF\_p\_pipe\_atma}
\putlisting{listings/MF_p_pipe_atma.lst}
\section{MF\_qliq\_choke\_sm3day}
\putlisting{listings/MF_qliq_choke_sm3day.lst}
\section{MF\_q\_mix\_rc\_m3day}
\putlisting{listings/MF_q_mix_rc_m3day.lst}
\section{MF\_rho\_mix\_kgm3}
\putlisting{listings/MF_rho_mix_kgm3.lst}
\section{MF\_rp\_gas\_fraction\_m3m3}
\putlisting{listings/MF_rp_gas_fraction_m3m3.lst}
\section{motor\_CosPhi\_d}
\putlisting{listings/motor_CosPhi_d.lst}
\section{motor\_CosPhi\_slip}
\putlisting{listings/motor_CosPhi_slip.lst}
\section{motor\_Eff\_d}
\putlisting{listings/motor_Eff_d.lst}
\section{motor\_Eff\_slip}
\putlisting{listings/motor_Eff_slip.lst}
\section{motor\_I\_A}
\putlisting{listings/motor_I_A.lst}
\section{motor\_I\_slip\_A}
\putlisting{listings/motor_I_slip_A.lst}
\section{motor\_M\_Nm}
\putlisting{listings/motor_M_Nm.lst}
\section{motor\_M\_slip\_Nm}
\putlisting{listings/motor_M_slip_Nm.lst}
\section{motor\_Name}
\putlisting{listings/motor_Name.lst}
\section{motor\_Pnom\_kW}
\putlisting{listings/motor_Pnom_kW.lst}
\section{motor\_S\_d}
\putlisting{listings/motor_S_d.lst}
\section{nodal\_pwf\_atma}
\putlisting{listings/nodal_pwf_atma.lst}
\section{PVT\_Bg\_m3m3}
\putlisting{listings/PVT_Bg_m3m3.lst}
\section{PVT\_Bo\_m3m3}
\putlisting{listings/PVT_Bo_m3m3.lst}
\section{PVT\_Bw\_m3m3}
\putlisting{listings/PVT_Bw_m3m3.lst}
\section{PVT\_decode\_string}
\putlisting{listings/PVT_decode_string.lst}
\section{PVT\_encode\_string}
\putlisting{listings/PVT_encode_string.lst}
\section{PVT\_mu\_gas\_cP}
\putlisting{listings/PVT_mu_gas_cP.lst}
\section{PVT\_mu\_oil\_cP}
\putlisting{listings/PVT_mu_oil_cP.lst}
\section{PVT\_mu\_wat\_cP}
\putlisting{listings/PVT_mu_wat_cP.lst}
\section{PVT\_Pb\_atma}
\putlisting{listings/PVT_Pb_atma.lst}
\section{PVT\_rho\_gas\_kgm3}
\putlisting{listings/PVT_rho_gas_kgm3.lst}
\section{PVT\_rho\_oil\_kgm3}
\putlisting{listings/PVT_rho_oil_kgm3.lst}
\section{PVT\_rho\_wat\_kgm3}
\putlisting{listings/PVT_rho_wat_kgm3.lst}
\section{PVT\_Rs\_m3m3}
\putlisting{listings/PVT_Rs_m3m3.lst}
\section{PVT\_salinity\_ppm}
\putlisting{listings/PVT_salinity_ppm.lst}
\section{PVT\_ST\_liqgas\_Nm}
\putlisting{listings/PVT_ST_liqgas_Nm.lst}
\section{PVT\_ST\_oilgas\_Nm}
\putlisting{listings/PVT_ST_oilgas_Nm.lst}
\section{PVT\_ST\_watgas\_Nm}
\putlisting{listings/PVT_ST_watgas_Nm.lst}
\section{PVT\_Z}
\putlisting{listings/PVT_Z.lst}
\section{transient\_def\_cd}
\putlisting{listings/transient_def_cd.lst}
\section{transient\_def\_cs\_1atm}
\putlisting{listings/transient_def_cs_1atm.lst}
\section{transient\_def\_pd}
\putlisting{listings/transient_def_pd.lst}
\section{transient\_def\_pwf\_atma}
\putlisting{listings/transient_def_pwf_atma.lst}
\section{transient\_def\_td}
\putlisting{listings/transient_def_td.lst}
\section{transient\_def\_t\_day}
\putlisting{listings/transient_def_t_day.lst}
\section{transient\_pd\_radial}
\putlisting{listings/transient_pd_radial.lst}
\section{transient\_pwf\_radial\_atma}
\putlisting{listings/transient_pwf_radial_atma.lst}
\section{wellESP\_plin\_pintake\_atma}
\putlisting{listings/wellESP_plin_pintake_atma.lst}
\section{Well\_Plin\_Pwf\_atma}
\putlisting{listings/Well_Plin_Pwf_atma.lst}
\section{Well\_Pwf\_Plin\_atma}
\putlisting{listings/Well_Pwf_Plin_atma.lst}
      % Список литературы

	
\end{document}
