% файл с настройками документов, для того чтобы транслировать разные настройки по нескольким документам
% rnt 2018

%% Режим черновика
\makeatletter
\@ifundefined{c@draft}{
  \newcounter{draft}
  \setcounter{draft}{0}  % 0 --- чистовик (максимальное соблюдение ГОСТ)
                         % 1 --- черновик (отклонения от ГОСТ, но быстрая сборка итоговых PDF)
}{}
\makeatother


%%% Использование в xelatex и lualatex семейств шрифтов %%%
\makeatletter
\@ifundefined{c@fontfamily}{
  \newcounter{fontfamily}
  \setcounter{fontfamily}{1}  % 0 --- CMU семейство. Используется как fallback;
                              % 1 --- Шрифты от MS (Times New Roman и компания)
                              % 2 --- Семейство Liberation
}{}
\makeatother

%% Библиография


%%% Предкомпиляция tikz рисунков для ускорения работы %%%
\makeatletter
\@ifundefined{c@imgprecompile}{
  \newcounter{imgprecompile}
  \setcounter{imgprecompile}{0}   % 0 --- без предкомпиляции;
                                  % 1 --- пользоваться предварительно скомпилированными pdf вместо генерации заново из tikz
}{}
\makeatother
            % общие настройки шаблона
%%% Проверка используемого TeX-движка %%%
\RequirePackage{ifxetex, ifluatex}
\newif\ifxetexorluatex   % определяем новый условный оператор (http://tex.stackexchange.com/a/47579)
\ifxetex
    \xetexorluatextrue
\else
    \ifluatex
        \xetexorluatextrue
    \else
        \xetexorluatexfalse
    \fi
\fi

\RequirePackage{etoolbox}[2015/08/02]               % Для продвинутой проверки разных условий

%%% Поля и разметка страницы %%%
\usepackage{pdflscape}                              % Для включения альбомных страниц
\usepackage{geometry}                               % Для последующего задания полей

%%% Математические пакеты %%%
\usepackage{amsthm,amsmath,amscd}   % Математические дополнения от AMS
\usepackage{amsfonts,amssymb}       % Математические дополнения от AMS
\usepackage{mathtools}              % Добавляет окружение multlined
\usepackage{unicode-math}           % использование Unicode-шрифтов для формул

%%%% Установки для размера шрифта 14 pt %%%%
%% Формирование переменных и констант для сравнения (один раз для всех подключаемых файлов)%%
%% должно располагаться до вызова пакета fontspec или polyglossia, потому что они сбивают его работу
\newlength{\curtextsize}
\newlength{\bigtextsize}
\setlength{\bigtextsize}{13.9pt}

\makeatletter
%\show\f@size                                       % неплохо для отслеживания, но вызывает стопорение процесса, если документ компилируется без команды  -interaction=nonstopmode 
\setlength{\curtextsize}{\f@size pt}
\makeatother

%%% Кодировки и шрифты %%%

\usepackage{polyglossia}[2014/05/21]            % Поддержка многоязычности (fontspec подгружается автоматически)


%%% Оформление абзацев %%%
\usepackage{indentfirst}                            % Красная строка

%%% Цвета %%%
\usepackage[dvipsnames, table, hyperref, cmyk]{xcolor} % Совместимо с tikz. Конвертация всех цветов в cmyk заложена как удовлетворение возможного требования типографий. Возможно конвертирование и в rgb.

%%% Таблицы %%%
\usepackage{longtable,ltcaption}                    % Длинные таблицы
\usepackage{multirow,makecell}                      % Улучшенное форматирование таблиц
\usepackage{pbox}

%%% Общее форматирование
\usepackage{soulutf8}                               % Поддержка переносоустойчивых подчёркиваний и зачёркиваний
\usepackage{icomma}                                 % Запятая в десятичных дробях

%%% Оптимизация расстановки переносов и длины последней строки абзаца
\ifluatex
    \ifnumequal{\value{draft}}{1}{% Черновик
        \usepackage[hyphenation, lastparline, nosingleletter, homeoarchy,
        rivers, draft]{impnattypo}
    }{% Чистовик
        \usepackage[hyphenation, lastparline, nosingleletter]{impnattypo}
    }
\else
    \usepackage[hyphenation, lastparline]{impnattypo}
\fi

%%% Гиперссылки %%%
\usepackage{hyperref}[2012/11/06]

%%% Изображения %%%
\usepackage{graphicx}[2014/04/25]                   % Подключаем пакет работы с графикой

%%% Списки %%%
\usepackage{enumitem}

%%% Счётчики %%%
\usepackage[figure,table]{totalcount}               % Счётчик рисунков и таблиц
\usepackage{totcount}                               % Пакет создания счётчиков на основе последнего номера подсчитываемого элемента (может требовать дважды компилировать документ)
\usepackage{totpages}                               % Счётчик страниц, совместимый с hyperref (ссылается на номер последней страницы). Желательно ставить последним пакетом в преамбуле

%%% Продвинутое управление групповыми ссылками (пока только формулами) %%%
\usepackage{cleveref}                              % cleveref корректно считывает язык из настроек polyglossia

\creflabelformat{equation}{#2#1#3}                  % Формат по умолчанию ставил круглые скобки вокруг каждого номера ссылки, теперь просто номера ссылок без какого-либо дополнительного оформления
\crefrangelabelformat{equation}{#3#1#4\cyrdash#5#2#6}   % Интервалы в русском языке принято делать через тире, если иное не оговорено


\ifnumequal{\value{draft}}{1}{% Черновик
    \usepackage[firstpage]{draftwatermark}
    \SetWatermarkText{DRAFT}
    \SetWatermarkFontSize{14pt}
    \SetWatermarkScale{15}
    \SetWatermarkAngle{45}
}{}

%%% Цитата, не приводимая в автореферате:
% возможно, актуальна только для biblatex
%\newcommand{\citeinsynopsis}[1]{\ifsynopsis\else ~\cite{#1} \fi}


%%% Прикладные пакеты %%% 
%\usepackage{calc}               % Пакет для расчётов параметров, например длины

%%% Для добавления Стр. над номерами страниц в оглавлении
%%% http://tex.stackexchange.com/a/306950
\usepackage{afterpage}

\usepackage{tikz}                   % Продвинутый пакет векторной графики
\usetikzlibrary{chains}             % Для примера tikz рисунка
\usetikzlibrary{shapes.geometric}   % Для примера tikz рисунка
\usetikzlibrary{shapes.symbols}     % Для примера tikz рисунка
\usetikzlibrary{arrows}             % Для примера tikz рисунка
\ifnumequal{\value{imgprecompile}}{1}{% Только если у нас включена предкомпиляция
	\usetikzlibrary{external}   % подключение возможности предкомпиляции
	\tikzexternalize[prefix=Dissertation/images/] % activate! % здесь можно указать отдельную папку для скомпилированных файлов
	\ifxetex
	\tikzset{external/up to date check={diff}}
	\fi
}{}


\usepackage{tabu, tabulary}  %таблицы с автоматически подбирающейся шириной столбцов
\usepackage{fr-longtable}    %ради \endlasthead



% Русская традиция начертания греческих букв
\usepackage{upgreek} % прямые греческие ради русской традиции

%%% Микротипографика
%\ifnumequal{\value{draft}}{0}{% Только если у нас режим чистовика
%    \usepackage[final, babel, shrink=45]{microtype}[2016/05/14] % улучшает представление букв и слов в строках, может помочь при наличии отдельно висящих слов
%}{}

% Отметка о версии черновика на каждой странице
% Чтобы работало надо в своей локальной копии по инструкции
% https://www.ctan.org/pkg/gitinfo2 создать небходимые файлы в папке
% ./git/hooks
% If you’re familiar with tweaking git, you can probably work it out for
% yourself. If not, I suggest you follow these steps:
% 1. First, you need a git repository and working tree. For this example,
% let’s suppose that the root of the working tree is in ~/compsci
% 2. Copy the file post-xxx-sample.txt (which is in the same folder of
% your TEX distribution as this pdf) into the git hooks directory in your
% working copy. In our example case, you should end up with a file called
% ~/compsci/.git/hooks/post-checkout
% 3. If you’re using a unix-like system, don’t forget to make the file executable.
% Just how you do this is outside the scope of this manual, but one
% possible way is with commands such as this:
% chmod g+x post-checkout.
% 4. Test your setup with “git checkout master” (or another suitable branch
% name). This should generate copies of gitHeadInfo.gin in the directories
% you intended.
% 5. Now make two more copies of this file in the same directory (hooks),
% calling them post-commit and post-merge, and you’re done. As before,
% users of unix-like systems should ensure these files are marked as
% executable.
\ifnumequal{\value{draft}}{1}{% Черновик
	\IfFileExists{.git/gitHeadInfo.gin}{                                        
		\usepackage[mark,pcount]{gitinfo2}
		\renewcommand{\gitMark}{rev.\gitAbbrevHash\quad\gitCommitterEmail\quad\gitAuthorIsoDate}
		\renewcommand{\gitMarkFormat}{\rmfamily\color{Gray}\small\bfseries}
	}{}
}{}


% для раскраски листингов кода
\usepackage{minted}
\usemintedstyle{vs}
\usepackage{tcolorbox}

\usepackage{pgfplotstable}
\usepackage{pgfplots}
\pgfplotsset{compat=1.12}

\tcbuselibrary{breakable,skins,minted}

\renewcommand{\listingscaption}{Листинг}

\newcommand{\putlisting}[1]{
	\tcbinputlisting{
		listing file=#1,
		minted language=vb.net,
		minted options={breaklines,fontsize=\footnotesize},% <-- put other minted options inside the brackets
		breakable,enhanced,% <-- put other tcolorbox options here
		listing only
	}
}


         % Пакеты общие для диссертации и автореферата


\input{../setup/common/setup_template}      % Упрощённые настройки шаблона

\input{../setup/common/names}         % Новые переменные, для всего проекта

\input{../setup/common/data}             % Основные сведения
\input{../setup/common/fonts}            % Определение шрифтов (частичное)
\input{../setup/common/styles}           % Стили общие для диссертации и автореферата
\input{../setup/common/disstyles}  % Стили для диссертации
% для вертикального центрирования ячеек в tabulary
\def\zz{\ifx\[$\else\aftergroup\zzz\fi}
%$ \] % <-- чиним подсветку синтаксиса в некоторых редакторах
\def\zzz{\setbox0\lastbox
\dimen0\dimexpr\extrarowheight + \ht0-\dp0\relax
\setbox0\hbox{\raise-.5\dimen0\box0}%
\ht0=\dimexpr\ht0+\extrarowheight\relax
\dp0=\dimexpr\dp0+\extrarowheight\relax 
\box0
}




%Общие счётчики окружений листингов
%http://tex.stackexchange.com/questions/145546/how-to-make-figure-and-listing-share-their-counter
% Если смешивать плавающие и не плавающие окружения, то могут быть проблемы с нумерацией
\makeatletter
\AtBeginDocument{%
    \let\c@ListingEnv\c@lstlisting
    \let\theListingEnv\thelstlisting
    \let\ftype@lstlisting\ftype@ListingEnv % give the floats the same precedence
}
\makeatother

% значок С++ — используйте команду \cpp
\newcommand{\cpp}{%
    C\nolinebreak\hspace{-.05em}%
    \raisebox{.2ex}{+}\nolinebreak\hspace{-.10em}%
    \raisebox{.2ex}{+}%
}

%%%  Чересстрочное форматирование таблиц
%% http://tex.stackexchange.com/questions/278362/apply-italic-formatting-to-every-other-row
\newcounter{rowcnt}
\newcommand\altshape{\ifnumodd{\value{rowcnt}}{\color{red}}{\vspace*{-1ex}\itshape}}
% \AtBeginEnvironment{tabular}{\setcounter{rowcnt}{1}}
% \AtEndEnvironment{tabular}{\setcounter{rowcnt}{0}}

%%% Ради примера во второй главе
\let\originalepsilon\epsilon
\let\originalphi\phi
\let\originalkappa\kappa
\let\originalle\le
\let\originalleq\leq
\let\originalge\ge
\let\originalgeq\geq
\let\originalemptyset\emptyset
\let\originaltan\tan
\let\originalcot\cot
\let\originalcsc\csc

%%% Русская традиция начертания математических знаков
\renewcommand{\le}{\ensuremath{\leqslant}}
\renewcommand{\leq}{\ensuremath{\leqslant}}
\renewcommand{\ge}{\ensuremath{\geqslant}}
\renewcommand{\geq}{\ensuremath{\geqslant}}
\renewcommand{\emptyset}{\varnothing}

%%% Русская традиция начертания математических функций (на случай копирования из зарубежных источников)
\renewcommand{\tan}{\operatorname{tg}}
\renewcommand{\cot}{\operatorname{ctg}}
\renewcommand{\csc}{\operatorname{cosec}}

%%% Русская традиция начертания греческих букв (греческие буквы вертикальные, через пакет upgreek)
\renewcommand{\epsilon}{\ensuremath{\upvarepsilon}}   %  русская традиция записи
\renewcommand{\phi}{\ensuremath{\upvarphi}}
%\renewcommand{\kappa}{\ensuremath{\varkappa}}
\renewcommand{\alpha}{\upalpha}
\renewcommand{\beta}{\upbeta}
\renewcommand{\gamma}{\upgamma}
\renewcommand{\delta}{\updelta}
\renewcommand{\varepsilon}{\upvarepsilon}
\renewcommand{\zeta}{\upzeta}
\renewcommand{\eta}{\upeta}
\renewcommand{\theta}{\uptheta}
\renewcommand{\vartheta}{\upvartheta}
\renewcommand{\iota}{\upiota}
\renewcommand{\kappa}{\upkappa}
\renewcommand{\lambda}{\uplambda}
\renewcommand{\mu}{\upmu}
\renewcommand{\nu}{\upnu}
\renewcommand{\xi}{\upxi}
\renewcommand{\pi}{\uppi}
\renewcommand{\varpi}{\upvarpi}
\renewcommand{\rho}{\uprho}
%\renewcommand{\varrho}{\upvarrho}
\renewcommand{\sigma}{\upsigma}
%\renewcommand{\varsigma}{\upvarsigma}
\renewcommand{\tau}{\uptau}
\renewcommand{\upsilon}{\upupsilon}
\renewcommand{\varphi}{\upvarphi}
\renewcommand{\chi}{\upchi}
\renewcommand{\psi}{\uppsi}
\renewcommand{\omega}{\upomega}
 % Стили для специфических пользовательских задач

%%% Библиография %%%
\input{../biblio/biblatex}     % Реализация пакетом biblatex через движок biber

%%% Управление компиляцией отдельных частей диссертации %%%
% Необходимо сначала иметь полностью скомпилированный документ, чтобы все
% промежуточные файлы были в наличии
% Затем, для вывода отдельных частей можно воспользоваться командой \includeonly
% Ниже примеры использования команды:
%
%\includeonly{Dissertation/part2}
%\includeonly{Dissertation/contents,Dissertation/appendix,Dissertation/conclusion}
%
% Если все команды закомментированы, то документ будет выведен в PDF файл полностью



%\usepackage[xetex]{graphicx}


%\usepackage{polyglossia}  
%polyglossia — пакет для многоязычной верстки с в системе XeLaTeX  
%https://ru.wikibooks.org/wiki/LaTeX/polyglossia

%\setdefaultlanguage[spelling=modern]{russian}
%\setotherlanguage{english}
%\setkeys{russian}{babelshorthands=true}

%\setmainfont{Times New Roman}
%\setromanfont{Times New Roman} 
%\setsansfont{Arial}
%\setmonofont{Courier New}

%\newfontfamily{\cyrillicfont}{Times New Roman} 
%\newfontfamily{\cyrillicfontrm}{Times New Roman}
%\newfontfamily{\cyrillicfonttt}{Courier New}
%\newfontfamily{\cyrillicfontsf}{Arial}

%\addto\captionsrussian{%
%	\renewcommand{\figurename}{Рис.}%
%	\renewcommand{\tablename}{Табл.}%
%}

%\usepackage{amsmath,amsfonts,amssymb,amsthm,mathtools} % AMS
%\usepackage{icomma} % "Умная" запятая: $0,2$ --- число, $0, 2$ --- перечисление
%\usepackage{unicode-math} 
%\setmathfont{Asana-Math.otf}



%\usepackage{indentfirst}
%\frenchspacing
%\usepackage{hyperref}
%\usepackage{csquotes} 	% ещё одна штука для цитат

%\input{biblio/biblatex}     % Реализация пакетом biblatex через движок biber

%\usepackage[backend=biber,style=numeric,autocite=footnote,sorting=none]{biblatex}
%\usepackage[% 
%backend=biber, %подключение пакета biber (тоже нужен)
%bibstyle=gost-numeric, %подключение одного из четырех главных стилей biblatex-gost 
%citestyle=numeric-comp, %подключение стиля стиля (а вот!) 
%language=auto, %указание сортировки языков
%babel=other, %указание языков
%sorting=ntvy, %тип сортировки в библиографии
%doi=false, 
%eprint=false, 
%isbn=false, 
%dashed=false, 
%url=false %все false выключают отображение полей, заполненных в библиографической базе, но не актуальных для %печатного листа
%]{biblatex}
%\bibliography{cyrill} %библиографическая база
%\usepackage{bibentry} %пакет, отвечающий за отображение ссылок при использовании BibTex 

%\addbibresource{bib/biblio.bib}
%\bibliographystyle{bib/utf8gost705u}


